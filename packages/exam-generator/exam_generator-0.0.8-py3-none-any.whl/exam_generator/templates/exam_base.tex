% https://www.overleaf.com/read/yyzgvwtqrfny

%%%%%%%%%%%%%%%%%%%%%%%%%%%%%%%%%%%%%%%%%%%%%%%%%%%%%%%%%%%%%%%%%%%%%%%%%%%%%%%%%
% START EXAM GENERATOR TEMPLATE. THIS DOCUMENT HAS BEEN AUTOMATICALLY GENERATED.
%%%%%%%%%%%%%%%%%%%%%%%%%%%%%%%%%%%%%%%%%%%%%%%%%%%%%%%%%%%%%%%%%%%%%%%%%%%%%%%%%


\documentclass[addpoints]{exam}
\usepackage{bbm}


\input{ {{dirs.public_git_dir_relative_to_latex}}/shared/02465shared_preamble}


\usepackage{tikz}
%\usetikzlibrary{shapes,arrows,positioning}
\usetikzlibrary{automata,arrows,positioning,calc}
\usetikzlibrary{arrows.meta}
% Exam-class specific stuff.
\usepackage{color}
\usepackage{comment}
\shadedsolutions
\definecolor{SolutionColor}{rgb}{1.,.85,.85}

\printanswers

\CorrectChoiceEmphasis{\color{red}}

\usepackage{rotating}
\usepackage{array}
\usepackage{multirow}
\usepackage{amsmath}
\usepackage{textcomp}
\usepackage{amssymb}
\usepackage{color}
\usepackage{graphicx}
\usepackage{booktabs}
\usepackage{enumerate}
\usepackage{url}
\usepackage{todonotes}
\usepackage[yyyymmdd,hhmmss]{datetime}
\usepackage{enumitem}

\usepackage{xr}

\externaldocument{ {{dirs.public_git_dir_relative_to_latex}}/Notes/Latex/02465_Notes}



\newcommand*\nref[1]{\cite[\cref{{'#1'|bracket}}]{{'herlau'|bracket}}}

\newcommand{\yoursolution}{ \redt{YOUR SOLUTION HERE} }
\newcommand{\stub}{ \begin{solutionstub} \redt{YOUR SOLUTION HERE} \end{solutionstub} }
\newtoggle{show_solutions}\toggletrue{show_solutions}
\togglefalse{show_solutions} \newtoggle{show_stubs}\toggletrue{show_stubs}
\newcommand{\projectnum}{3}


%% Boxes for solutions (eventually).
\usepackage{amsmath}
\usepackage{empheq}
\usepackage{xcolor}
\definecolor{lightgreen}{HTML}{90EE90}
\newcommand{\sstub}[1]{\begin{empheq}[box={\fboxsep=6pt\fbox}]{align}{ #1}\end{empheq} }


\pagestyle{headandfoot}
\runningheadrule
\newcommand{\continuedmessage}{
\ifcontinuation{Question \ContinuedQuestion\ continues\ldots}{}%
}
\firstpageheader{ {{header_left}} }{ {{header_center}} }{ {{date}} }
\runningheader{\continuedmessage}
{ {{centerlabel}} }
{Page \thepage\ of \numpages}
\footer{}
{}
{\ifincomplete{Question \IncompleteQuestion\ continues
on the next page\ldots}{}}

 \pointpoints{$\%$}{$\%$}



\PassOptionsToPackage{svgnames,x11names,dvipsnames}{xcolor}
%\usepackage[most]{tcolorbox}
\newtcbox{\qbox}[1][]{enhanced,
	box align=base,
	nobeforeafter,
	colback=yellow!20,
	colframe=red!20,
	size=small,
	left=1pt,
	right=1pt,
	boxsep=2pt,
	#1}


\begin{document}



% -------------------------------------------------------------
% This code creates the text before the first question
% -------------------------------------------------------------
\begin{center}
\fbox{\fbox{\parbox{5.5in}{
\begin{description}[noitemsep,topsep=0pt]
\item[Midterm date:Written examination date:]  {{date}}
\item[Course title:] {{course_title}}
\item[Course number:] 02465
\item[Aids allowed:] All aids allowed
\item[MidtermExam duration:] {{duration}}
% Kursusnavn Introduktion til Intelligente Systemer
% Kursusnummer 02461
% Hjælpemidler Alle hjælpemidler er tilladt
% Varighed 2 timer
% Vægtning Alle delopgaver vægtes ens
\end{description}
\textbf{Weighting:} The exam is divided into 3 parts:
\begin{itemize}[noitemsep,topsep=0pt]
\item Multiple-Choice questions
\item Conceptual questions
\item Programming questions
\end{itemize}
The overall scores of each part contribute roughly equally towards the overall result. Each question in each part contribute equally towards the score of that part.

\noindent \textbf{Part I: } Questions 1-{{N_MC}} are multiple choice. The score of a correct answer is 3 points. The score of an incorrect answer is $-1$ points. The score of option $E$ or blank is $0$ points.\\
\noindent \textbf{Part II and part III: } Each completed sub-task contribute towards your score. \\
\noindent \textbf{Preparing your handin:} The three parts are prepared as follows:
\begin{description}[noitemsep,topsep=0pt]
    \item[Part I:] Edit the file \bai{{mc_answer_path|bracket}}. Don't include calculations. Only answer with \pyi{'A'}, \pyi{'B'}, \pyi{'C'}, \pyi{'D'}, \pyi{'E'}.
    \item[Part II:] Create a PDF file with your answers and justifications.
    \item[Part III:] Program your answer in the \pyi{.py}-files indicated in the question and run \bai{{gradepy|bracket}} to generate your \bai{.token}-file.
\end{description}
\noindent \textbf{Handing in: } To hand in, you should upload the files:
\begin{itemize}
\item The \bai{{mc_answer_path|bracket}}-file with your answer to part I
\item The \bai{.pdf}-file with your answers to part II
\item The \bai{.token}-file with your answers to part III

\item The \bai{{t|bracket}} source file containing your solution

\end{itemize}
\noindent \textbf{Note on part II: }
The main quantities asked for are highlighted as \qbox{$f(x)$}. Answer unambigiously, concisely, and if applicable with algebraic simplifications.
Your final result must be clearly indicated at the end of your answer: \qbox{$f(x) = 3\sin(x)$}. To get credit, you must state the relevant theory and equations, and all relevant calculations must be included.
  Credit is not given for answers with missing or erroneous justifications.\\
\noindent \textbf{Note on part III:} To get started, move the folder \bai{{midterm_base|bracket}} from the \bai{.zip} file to the corresponding location in your existing exercise directory.
The \bai{.py} source files must be \textbf{reproducible} and \textbf{readable} so that someone else can run and fully understand your solution.
You can freely use the \pyi{irlc}-toolbox (including solutions) and the packages we have used in the course,
but you \textbf{must} include additional code you write during the exam or have prepared beforehand in the source files listed above. The source files must not be renamed.
The \bai{.token} file contains your results and must be up to date with your source files, i.e., generate it just prior to handin. In the case they differ, the \bai{.token} file takes precedence.
Credit is given for correct implementations defined by the problem description. The points in the \bai{.token} file name is computed from the public tests, and might not correspond to overall correctness.
\clearpage



%Fill in your answers in the provided boxes.  \newline
%\noindent
%\textbf{Scoring:} The score of each question is indicated in the margin in a box. The score of the exam is the sum of the individual scores. I.e., a completely correct exam will give a total score of $100\%$, and a blank exam will give a score of $0\%$.

%\textbf{Scoring of the multiple-choice questions: } Questions 1-9 are multiple choice. The score of a correctly filled in questions is as indicated in the margin ($3.7\%$). The score of a wrong answer is $\frac{-3.7}{3} = -1.2\%$. The score of option $E$, or a blank answer, is $0\%$. This mean random guessing will on average net a score of $0\%$.


%Alle spørgsmål skal besvares med et entydigt resultat, som skal angives med understregning sidst i hver
%besvarelse. Besvarelsen skal altid underbygges med relevante betragtninger og/eller beregninger. Det skal
%klart fremgå hvilke teorier og formler der tages udgangspunkt i, og alle relevante mellemregninger skal
%medtages.
}  } }


\end{center}

%\qformat{\textbf{Question \thequestion : \thequestiontitle\dotfill\thepoints} }
\qformat{\textbf{\thequestiontitle\dotfill} }

%\vspace{5mm}

%\makebox[\textwidth]{Name:\enspace\hrulefill}

%\vspace{5mm}

%\makebox[\textwidth]{Student ID (Format: \texttt{s123456}):\enspace\hrulefill}
% -------------------------------------------------------------



\clearpage
\begin{questions}


	
		{{q}}
	

\end{questions}

\clearpage


\bibliographystyle{alpha}


\bibliography{ {{dirs.public_git_dir_relative_to_latex}}/shared/pensum}




\noindent \emph{This line concludes the exam. Document build: \today, \currenttime.}


%\inputminted[]{python}

\end{document}
