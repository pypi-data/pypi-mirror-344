\documentclass{article}

\usepackage{arxiv}

\usepackage[utf8]{inputenc} % allow utf-8 input
\usepackage[T1]{fontenc}    % use 8-bit T1 fonts
\usepackage[hidelinks]{hyperref}       % hyperlinks
\usepackage{url}            % simple URL typesetting
\usepackage{booktabs}       % professional-quality tables
\usepackage{amsmath,amssymb,amsthm}
\usepackage{amsfonts}       % blackboard math symbols
\usepackage{float}
\usepackage{nicefrac}       % compact symbols for 1/2, etc.
\usepackage{microtype}      % microtypography
\usepackage{mathrsfs}
\usepackage{graphicx}
\usepackage{doi}
\usepackage{acronym}
\usepackage{listings}
\usepackage{tikz}
\usepackage[dvipsnames]{xcolor}

\usetikzlibrary{trees}

\newacro{abm}[ABM]{Agent-Based Model}
\newacro{cabm}[CABM]{Cellular Agent-Based Model}
\newacro{ca}[CA]{Cellular Automaton}
\newacroplural{ca}[CA]{Cellular Automata}

\newcommand{\todo}[1]{\colorbox{WildStrawberry}{\textcolor{white}{#1}}}

\title{Bayesian Optimization for parameter estimation of Cell-Based Agent-Based Models}

%\date{September 9, 1985}	% Here you can change the date presented in the paper title
%\date{} 					% Or removing it

\author{
    \href{https://orcid.org/0009-0001-0613-7978}{
        \includegraphics[scale=0.06]{orcid.pdf}
        \hspace{1mm}Jonas Pleyer
    }
    \thanks{
        \href{https://jonas.pleyer.org}{jonas.pleyer.org},
        \href{https://cellular-raza.com}{cellular-raza.com}
    }\\
	Freiburg Center for Data-Analysis and Modeling\\
	University of Freiburg\\
	\texttt{jonas.pleyer@fdm.uni-freiburg.de} \\
	%% examples of more authors
	\And
    Polina Gaindrik\\
	Freiburg Center for Data-Analysis and Modeling\\
	University of Freiburg\\
	\And
	\href{https://orcid.org/0000-0002-6371-4495}{
        \includegraphics[scale=0.06]{orcid.pdf}
        \hspace{1mm}Christian Fleck
    }\\
	Freiburg Center for Data-Analysis and Modeling\\
	University of Freiburg
}

% Uncomment to remove the date
%\date{}

% Uncomment to override  the `A preprint' in the header
\renewcommand{\headeright}{Preprint}
%\renewcommand{\undertitle}{Technical Report}
\renewcommand{\shorttitle}{
    Bayesian Optimization for parameter estimation of Cell-Based Agent-Based Models
}

\usepackage{enumitem}
\setlist{nolistsep}

%%% Add PDF metadata to help others organize their library
%%% Once the PDF is generated, you can check the metadata with
%%% $ pdfinfo template.pdf
\hypersetup{
pdftitle={Mathematical Foundations of Cellular Agent-Based Models},
pdfsubject={q-bio.NC, q-bio.QM},
pdfauthor={Jonas Pleyer, Christian Fleck},
pdfkeywords={},
}

% Define definition, example, lemma, proof and theorem.
\newtheorem{definition}{Definition}[section]
\newtheorem{example}[definition]{Example}
\newtheorem{lemma}[definition]{Lemma}
\newtheorem{corollary}[definition]{Corollary}
\newtheorem{theorem}[definition]{Theorem}

% Change numbering of equations
% \numberwithin{equation}{section}

% MAKE TITLES IN THEOREMS BOLD
\makeatletter
\def\th@plain{%
  \thm@notefont{}% same as heading font
  \itshape % body font
}
\def\th@definition{%
  \thm@notefont{}% same as heading font
  \normalfont % body font
}
\makeatother

\begin{document}
\maketitle

%###################################################################################################
\begin{abstract}
\end{abstract}


% keywords can be removed
\keywords{Agent-Based \and Biology \and Bayesian Optimization}

%###################################################################################################
\section{Introduction}
\label{section:introduction}
First notion of bayesian optimization~\cite{Kushner1964}, then in~\cite{Mockus1978}.

Mention AI and hyperparameter-tuning

\subsection*{Existing Literature}
\cite{Jrgensen2022}
\begin{itemize}
    \item construct light-weight surrogate models or estimate posterior directly
        \todo{can these surrogate models always be constructed?}
    \item Good Inference possible with low number of simulations
    \item estimating parameters requirs specific choice of method
\end{itemize}

Maybe cite these:
\cite{Lima2021,Duswald2024}

\begin{itemize}
    \item \cite{Stephan2024} "Agent-based approaches for biological modeling in oncology: A
        literature review" (it reads very badly tbh, many mistakes in plain sight, literature review
        not by hand but automation simply querying search engines, not scientific at all imho)
    \item \cite{An2016} use surrogate models for ABM Optimization (only uses population-based
        measures; no spatial stuff; why use an ABM in the first place?)
\end{itemize}

\subsubsection*{Bayesian Optimization}
\begin{itemize}
    \item Requires less function evaluations
    \item Can optimize nonlinear functions
\end{itemize}

%###################################################################################################
\section{Target System}
\label{section:target-system}
\begin{itemize}
    \item Make clear that black-box optimization is required
    \item Should have some application for real-world scenarios
\end{itemize}

%---------------------------------------------------------------------------------------------------
\subsection{Bacterial Rods}
\label{subsection:bacterial-rods}
\begin{itemize}
    \item Optimize parameters of indidivual agents; fit to data
    \item Fit agents via positions
    \item Fit agents via image comparison (see \texttt{cr\_mech\_coli})
    \item neighbor interaction $\rightarrow$ not diff.-able
\end{itemize}

\begin{figure}[H]
    \centering
    \includegraphics[width=0.48\textwidth]{figures/cr_mech_coli/bacterial-rods-0000000025.png}%
    \hspace{0.04\textwidth}%
    \includegraphics[width=0.48\textwidth]{figures/cr_mech_coli/bacterial-rods-0000007200.png}\\%
    \vspace{1em}
    \includegraphics
        [width=0.32\textwidth]{figures/cr_mech_coli/09395645494836445480/masks/000000200.png}%
    \hspace{0.01\textwidth}%
    \includegraphics
        [width=0.32\textwidth]{figures/cr_mech_coli/09395645494836445480/masks/000000600.png}%
    \hspace{0.01\textwidth}%
    \includegraphics
        [width=0.32\textwidth]{figures/cr_mech_coli/09395645494836445480/masks/000001000.png}\\%
    \includegraphics[width=0.49\textwidth]{figures/cr_mech_coli/estim-param/microscopic-images-0.png}%
    \includegraphics[width=0.49\textwidth]{figures/cr_mech_coli/estim-param/microscopic-images-1.png}%
    \caption{\todo{caption}}
    \label{fig:bacterial-rods-sim}
\end{figure}

%---------------------------------------------------------------------------------------------------
\subsection{Bacterial Branching}
\label{subsection:bacterial-branching}
\begin{itemize}
    \item Calculate fractal dimension of branching pattern
    \item Optimize parameters to obtain desired fractal dimension in branching pattern
    \item Really expensive optimization due to simulation and fractal dimension.
        Takes in the order of minutes (up to $30$min) for one simulation.
\end{itemize}

\begin{figure}[H]
    \centering
    \includegraphics[width=0.32\textwidth]{figures/crb_sim_branching/cells_at_iter_0000000200.png}%
    \hspace{0.01\textwidth}%
    \includegraphics[width=0.32\textwidth]{figures/crb_sim_branching/cells_at_iter_0000003400.png}%
    \hspace{0.01\textwidth}%
    \includegraphics[width=0.32\textwidth]{figures/crb_sim_branching/cells_at_iter_0000006600.png}%
    \caption{\todo{caption}}
    \label{fig:bacterial-branching-sim}
\end{figure}

\begin{figure}[H]
    \centering
    \includegraphics
        [width=0.32\textwidth]
        {figures/crb_sim_branching/diffusion-80/discretization-nvoxels-000022.png}%
    \hspace{0.01\textwidth}%
    \includegraphics
        [width=0.32\textwidth]
        {figures/crb_sim_branching/diffusion-80/discretization-nvoxels-000080.png}%
    \hspace{0.01\textwidth}%
    \includegraphics
        [width=0.32\textwidth]
        {figures/crb_sim_branching/diffusion-80/discretization-nvoxels-000666.png}\\%
    \includegraphics
        [width=0.5\textwidth]
        {figures/crb_sim_branching/fractal-dim-box-size-scaling.pdf}%
    \includegraphics
        [width=0.5\textwidth]
        {figures/crb_sim_branching/fractal-dim-over-time.pdf}\\%
    \includegraphics
        [width=0.5\textwidth]
        {figures/crb_sim_branching/runtime-sim-branching.pdf}%
    \includegraphics
        [width=0.5\textwidth]
        {figures/crb_sim_branching/fractal-dim-vs-diffusion-constant.pdf}%
    \caption{
        (A-C) Growth of a bacterial Colony from $5$ up to $6886$ agents.
        (D-F) Last simulation snapshot under different discretizations.
        (G) Scaling of number of Boxes with box size.
        The linear regression (LR) represents the data very nicely.
        This means that the model of a self-similar fractal is well satisfied.
        (H) Temporal change in the fractal dimension and colony size.
    }
    \label{fig:fractal-dim}
\end{figure}

\begin{itemize}
    \item Use curve in \ref{fig:fractal-dim} (H) to calculate fractal dimension very early on
    \item extrapolate curve (with error) to obtain final fractal dimension
    \item Use this procedure to explore parameter space much more quickly
    \item This needs some type of notion of "certainty" for the results since results at earlier
        time points $t_1$ are not the final results at $t_\text{fin}$.
    \item \todo{Runtime plot of simulation time}
\end{itemize}

%...................................................................................................
\subsubsection{Calculating the fractal dimension}
\begin{enumerate}
    \item Take all points of bacteria $x_i$ within the simulation domain
    \item Discretize the space at arbitrary (up to size of one cell) lattice size $\Delta x$
    \item Assign each grid-point the value $1$ if a cell is inside or $0$ otherwise
    \item Calculate which cells contain living cells and
    \item Calculate the scaling behaviour as $\Delta x\rightarrow0$ starting at a large (possibly
        trivially large) size for $\Delta x$
\end{enumerate}

%###################################################################################################
\section{Optimization}
\label{section:optimization}

Problems of form
\begin{equation}
    \text{max}_{x\in X} f(x)
\end{equation}

\begin{enumerate}
    \item need prior
    \item evaluate $f(x_i)$ for $x_i\in\text{prior}$
    \item form posterior distribution 
    \item construct acquisition function ("infill sampling criteria"); determines next query point
\end{enumerate}

Two methods for defining prior/posterior distributions
\begin{enumerate}
    \item "kriging"
    \item "Parzen-Tree Estimator"
\end{enumerate}

Main citation: \cite{Jones1998}
\begin{itemize}
    \item $f(x)$ difficult to evaluate
    \item $f(x)$ continuous
    \item no derivatives are evaluated
\end{itemize}

\begin{figure}[H]
    \centering
    \includegraphics[width=0.5\textwidth]{figures/optim-cost-function-temp.jpg}%
    \includegraphics[width=0.5\textwidth]{figures/optim-acquisition-function-comparison.jpg}
    \caption{\todo{caption}}
    \label{fig:optim-comparison-cost-function}
\end{figure}

%---------------------------------------------------------------------------------------------------
\subsection{Acquisition Functions}
\begin{itemize}
    \item Probability of Improvement
    \item Expected Improvement
    \item Bayesian Expected Losses
    \item Upper Confidence Bounds (UCB) or Lower Confidence Bounds (LCB)
    \item Thompson Sampling
\end{itemize}
Combinations of these are also possible.
All of them are trade-offs between exploration and exploitation to minimize the evaluations of
$f(x)$.

%###################################################################################################
\section{Results}
\label{section:results}

%###################################################################################################
\section{Discussion}
\label{subsection:discussion}

\bibliographystyle{IEEEtran}
\bibliography{references}

%###################################################################################################
\section{Supplementary Material}
\label{section:supplementary-material}

%---------------------------------------------------------------------------------------------------
\subsection{Example}
\label{subsection:supplement-example}

\end{document}
