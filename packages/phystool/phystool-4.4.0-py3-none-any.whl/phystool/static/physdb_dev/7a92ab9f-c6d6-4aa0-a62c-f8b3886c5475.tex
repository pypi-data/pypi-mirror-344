\begin{theory}[title=Intégration à git]
	Lors de la première utilisation de git (accessible via le menu
	\underline{G}it), le programme tente d'initialiser un nouveau répertoire
	git. Ce répertoire doit être mis en lien avec un \emph{remote} sur
	BitBucket (ou autre plateforme similaire). Afin de n'écraser aucune donnée,
	le \emph{remote} sélectionné doit être totalement vide, sans quoi,
	l'initialisation est annulée. L'url de ce \emph{remote} doit être de la
	forme \texttt{git@bitbucket.org:username/repository\_name.git}. Une fois la
	première initialisation réussie, il ne sera plus possible de changer l'url
	du remote dans \verb|physnoob| (même s'il est toujours possible de la
	changer manuellement via le terminal). 
\end{theory}
