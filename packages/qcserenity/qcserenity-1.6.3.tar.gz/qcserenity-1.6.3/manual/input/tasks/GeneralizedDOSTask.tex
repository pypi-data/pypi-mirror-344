\subsection{Task: Generalized Direct Orbital Selection\label{task:gdos}}
This task performs a generalized orbital selection for multiple structures along a reaction coordinate.
This task does not localize any orbitals! A generalized variant of the algorithm of
the active space selection task (see Section~\ref{sec:activeSpaceTask}) is employed.
\subsubsection{Example Input}
\begin{lstlisting}
 +task gdos
   act reactant
   act product
   env r1
   env r2
   env r3
   env p1
   env p2
   env p3
   similarityLocThreshold {1e-1 1e-2}
   similarityKinEnergyThreshold {1e-1 1e-2}
 -task
\end{lstlisting}
\subsubsection{Basic Keywords}
\begin{description}
 \item [\texttt{Name}]\hfill \\
 Aliases for this task are \ttt{GENERALIZEDDOS} and \ttt{GDOS}.
 \item [\texttt{Supersystems}]\hfill \\
 Accepts multiple supersystems (\ttt{super}) which are used for the orbital comparison.
 \item [\texttt{ActiveSystems}]\hfill \\
 Accepts multiple active systems which are the supersystem for the orbital comparison.
 \item [\texttt{EnvironmentSystems}]\hfill \\
 Accepts multiple environment systems which are used as the subsystem into which the
 supersystems are partitioned. The oder of the subsystems is important. The
 first set of subsystems is used for the first active system, etc., as given in the
 example. Within each set the subsystems importance to the relative energies between
 the supersystems decreases with increasing subsystem index.
 \item [\texttt{similarityLocThreshold}]\hfill \\
 The threshold for the comparison of partial charges. The default is ${0.05}$.
 \item [\texttt{similarityKinEnergyThreshold}]\hfill \\
 The threshold for the comparison of the orbital kinetic energy. The default is ${0.05}$.
 \item [\texttt{populationAlgorithm}]\hfill \\
 The algorithm used to calculate partial charges. The default are shell-wise intrinsic atomic orbital (IAO)
 charges (\ttt{IAOShell}). Other useful options are Mulliken charges (\ttt{MULLIKEN}) and atom-wise IAO
 charges (\ttt{IAO}).
 \item [\texttt{mapVirtuals}]\hfill \\
 If true, the virtual orbitals are considered in the orbital mapping. By default \ttt{false}.
 \item [\texttt{bestMatchMapping}]\hfill \\
 If true, the selection thresholds are optimized to provide a qualitative orbital map, \emph{i.e.},
 the thresholds are chosen such that they minimize the number of unmappable orbitals
 under the constraint that they are smaller than \texttt{scoreStart}. By default \ttt{false}.
\end{description}
\subsubsection{Advanced Keywords}
\begin{description}
    \item [\texttt{localizationThreshold}]\hfill \\
    The atom-wise Mulliken population threshold for the assignment of atoms to the subsystems. This is done
    after the supersystem orbital set has been separated into active and environment orbitals. If the total
    population of the active orbitals on a given atoms exceeds this threshold it is assigned to the active
    system. The default is $0.8$ au.
    \item [\texttt{usePiBias}]\hfill \\
    Use a selection threshold scaling based on the number of significant, orbital-wise partial charges for
    each orbital. This is used to reduce the over-selection of $\pi$-orbitals. The default is \ttt{false}.
    \item [\texttt{biasThreshold}]\hfill \\
    The threshold for the evaluation of the number of significant partial charges for. The default is
    \ttt{0.01}.
    \item [\texttt{biasAverage}]\hfill \\
    The averaging-parameter for the threshold scaling used for \ttt{usePiBias true}.
    The default is $12.0$.
    \item [\texttt{prioFirst}]\hfill \\
    Prioritize the first subsystem of each subsystem set for the atom-assignment after
    the orbital partitioning.
    \item [\texttt{writeScores}]\hfill \\
    If true, the scores at which each orbital is selected is written to file.
    The orbitals are not partitioned into subsystems.
    This is achieved with a large number of on a logarithmic scale tightly packed DOS thresholds.
    By default \ttt{false}.
    \item [\texttt{scoreStart}]\hfill \\
    The start of the DOS-thresholds to be scanned for \texttt{writeScores}.
    By default $0.1$.
    \item [\texttt{scoreEnd}]\hfill \\
    The end of the DOS-thresholds to be scanned for \texttt{writeScores}.
    By default 1e-4.
    \item [\texttt{nTest}]\hfill \\
    The number of thresholds scanned for \texttt{writeScores}. By default $1000$.
    \item [\texttt{writeGroupsToFile}]\hfill \\
    If true, a file is created containing the orbital set map between structures. By default \ttt{false}.
    \item [\texttt{checkDegeneracies}]\hfill \\
    If true, the orbitals are checked if they are very similar with respect to the comparison criteria. By default \ttt{true}.
    \item [\texttt{degeneracyFactor}]\hfill \\
    Threshold scaling for the degeneracy check. By default \ttt{1.0}.
 \end{description}
