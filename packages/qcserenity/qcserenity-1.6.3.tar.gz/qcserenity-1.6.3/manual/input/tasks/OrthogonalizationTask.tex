\subsection{Task: Orthogonalization}\label{task: ortho}
This task orthogonalizes orbitals.
\subsubsection{Example Input}
\begin{lstlisting}
	+task Ortho
	act water
	orthogonalizationScheme Loewdin
	-task
\end{lstlisting}

\subsubsection{Basic Keywords}
\begin{description}
	\item [\texttt{name}]\hfill \\
	Aliases for this task are \ttt{OrthogonalizationTask} and \ttt{Ortho}.
	\item [\texttt{Supersystems}]\hfill \\
	The first Supersystem is the system holding the orbitals resulting from the orthogonalization, if not specified the orthogonal orbitals are stored in a system
	called by the name of the first active system + "Ortho".
	\item [\texttt{activeSystem}]\hfill \\
	The active systems listed here are combined to a supersystem (see \ref{task: addition}) and the orbitals
	of this supersystem are orthogonalized and stored in the super system.
	\item [\texttt{orthogonalizationScheme}]\hfill \\
	The orthogonalization scheme used. Possible options are:\\
	\ttt{NONE}: No orthogonalization is performed.\\
	\ttt{LOEWDIN}: L\"owdin's symmetric orthogonalization\cite{lowdin1950, lowdin1970}.\\
	\ttt{PIPEK}: Pipek's iterative orthogonalization scheme\cite{pipek1985}.\\
	\ttt{BROER}: Broer's orthogonalization scheme based on corresponding orbitals\cite{broer1993}.\\
	\item [\texttt{maxIterations}]\hfill \\
	Maximum number of iterations performed in the Pipek scheme.
\end{description}
