\chapter{Text Input}

The input of \serenity has two main types of sections, the \ttt{system} blocks and the \ttt{task} blocks.
A chart depicting how \serenity handles input can be seen in Figure~\ref{fig:input}.
While the chart is already quite technical, it displays how to read the sectioned input.
Starting from the right, the main job of the program is to run tasks in a certain order.
These tasks can be related \textit{via} the systems they use, \textit{i.e.} a \ttt{MP2Task} will be used.
orbital data from a previous \ttt{SCFTask} if both of them are working with the same system.
Accordingly, the first part of the input has to be the system definitions which is followed by
a list of tasks to complete with these systems.
In a minimal case, the systems are just a geometry and a set of settings (such as charge, spin, basis set, and so on).
During the run these systems will be populated with more (calculated) data, such as orbitals, energies and electron densities.
This additional data can also be read from previous runs (see the following section).
Any feature that \serenity has is implemented as a task (or a part of a task).
For a list of available features/tasks, see Section~\ref{sec:tasks}.
\begin{figure}[h!]
\label{fig:input}
\includegraphics[width=0.95\textwidth]{./figs/SerenityInput.pdf}
\caption{Input flow of \serenity.}
\end{figure}\\
As stated above, the text input of \serenity is structured into blocks containing the keywords.
A keyword consists of a name and a value and is always given inside of a block.
Each keyword-value pair has to be given in one separate line.
As for the blocks, there are two main types, those are \ttt{system} and \ttt{task}.
A block in the {\serenity} text input is started by a \ttt{+<blockname>} and ended with a
\ttt{-<blockname>}.
The main blocks of a simple input might thus look like this:
\begin{lstlisting}
+system
 name water
 geometry ./water.xyz
 method dft
 +dft
  functional PBE
 -dft
-system

+task scf
  system water
-task
\end{lstlisting}
While there are many settings and thus many keywords, all of them are defaulted, and
each run of the program will create a file containing all settings (even the default ones)
for restart purposes.\\
The input accepts comments just like shell scripts.
Thus, everything preceded by a hash is not parsed.
This holds for entire lines but also for parts of a line.
\begin{lstlisting}
# not parsed
+system # also not parsed
 name test
 geometry test.xyz
 # you guessed it, this is also not parsed
-system
\end{lstlisting}
The entire input (with the exception of system names and path names) is case-insensitive.
Keywords could thus look like this:
\begin{lstlisting}
  gradients true
  funcTIONAL PbE0
  maxCycles 123
  NaMe water # note that the value of name is case sensitive
\end{lstlisting}
A list of values for a keyword can be parsed by enclosing them into curly brackets:
\begin{lstlisting}
  orbitals {1 2 3 4}
\end{lstlisting}
A list of settings (keywords) inside the \ttt{system} and \ttt{task} blocks will be
presented in the following two sections.

\newpage

% ================
%   System Input
% ================

\section{Systems}\label{sec:system}
Each system is defined by one system block which contains general keywords and can contain one layer of sub-blocks.
All relevant keywords are discussed in this section.
\subsection{General Keywords}\label{sec:system:general}
The keywords given directly inside the system block and not within one of the nested blocks are presented
in this section.\\
A regular system will need both the \ttt{geometry} and also the \ttt{name} keyword.
Further keywords can be given if their values need to differ from the defaults.
A loaded system only needs the \ttt{load} keyword and a \ttt{name}.
All other keywords will be deduced from the stored settings.
\subsubsection{Example Input}
\begin{lstlisting}
  +system
    name methyl
    geometry methyl.xyz
    method HF
    charge 0
    spin 1
  -system
 \end{lstlisting}
\subsubsection{Basic Keywords}
\begin{description}
 \item [\texttt{name}]\hfill \\
   The system name.
 \item [\texttt{geometry}]\hfill \\
   The path to the geometry (.xyz). This or \ttt{load} has to be given.
   \item [\texttt{path}]\hfill \\
   The path to store HDF5 files.
   \item [\texttt{load}]\hfill \\
   The path to load HDF5 and settings files from. If this is given all other settings are replaced by the loaded settings. \serenity will search 
   for a directory with \ttt{name} at the given path.
   \item [\texttt{charge}]\hfill \\
   The charge of the system. Default is $0$.
   \item [\texttt{spin}]\hfill \\
   The spin of the system. Given as alpha spin electron excess. Default is $0$.
 \item [\texttt{scfmode}]\hfill \\
   The orbital mode of the SCF. Default is \ttt{RESTRICTED}. If \ttt{UNRESTRICTED} is chosen, an unrestricted calculation is enforced.
  \item [\texttt{method}]\hfill \\
  The general electronic structure method for this system. By default, \ttt{HF} (Hartree--Fock) is chosen. The other option is \ttt{DFT} (density-functional theory). Note that \ttt{method HF} means that any density fitting will be ignored. To actually perform Hartree--Fock calculations with density fitting, use \ttt{method HF} and \ttt{functional HF} in the \ttt{DFT} block.
\end{description}

\subsubsection{Advanced Keywords}
\begin{description}
  \item [\texttt{ignorecharge}]\hfill \\
  Adds an electron in case the sum of charge and spin is odd. Default is \ttt{false}.
  \item [\texttt{externalGridPotential}]\hfill \\
  The path to load a potential from. Similar to external charges (see Sec.~\ref{sec:system:extchrg}), this potential is added to the system's core Hamiltonian. The file content is expected to be row-wise gridpoints with the coordinates and potential values given in atomic units, with the first line being ignored.
  \begin{lstlisting}
  #   x         y         z           w              pot   
  <x-coord> <y-coord> <z-coord> <grid-weight> <potential-value>
  <x-coord> <y-coord> <z-coord> <grid-weight> <potential-value>
  ...
  \end{lstlisting}
  Note that this is consistent with the format of the \ttt{ExportGridTask} (see Sec.~\ref{sec:tasks:ExportGridTask}).
\end{description}

\subsection{SCF Block}\label{sec:system:scf}
All settings available pertaining the SCF and its convergence are given in this sub-block of the system block.
\subsubsection{Example Input}
\begin{lstlisting}
  +system
    name methyl
    geometry methyl.xyz
    method HF
    charge 0
    spin 1
    +scf
      maxCycles 100
      damping dynamic
    -scf
  -system
 \end{lstlisting}
\subsubsection{Basic Keywords}
\begin{description}
 \item [\texttt{initialguess}]\hfill \\
   The initial guess for the electronic structure. Possible options are:\\
   \ttt{HCORE}: $h_{core}$ guess, i.e. ignoring electron-electron interactions.\\
   \ttt{EHT}: extended Hueckel theory guess.\\
   \ttt{ATOM\_SCF}: Attempts to load precalculated BHLYP/MINAO densities from disk and projects onto the basis set used in the calculation. If not found for a particular atom type
   or other problems occur, uses \ttt{ATOM\_SCF\_INPLACE} as a fall-back solution.\\
   \ttt{ATOM\_SCF\_INPLACE}: Generate the atomic densities on the fly with atom-wise SCFs in the basis of the calculation. Always starts from an Hcore guess with spherical densities.\\
   \ttt{SAP}: superposition of atomic potentials.\\
   The default is \ttt{ATOM\_SCF}. For atoms using effective core potentials, the \ttt{ATOM\_SCF\_INPLACE}
   guess is used as the default.
 \item [\texttt{damping}]\hfill \\
   The mode of damping of the new Fock matrix with the last one during the SCF.\\
   \ttt{NONE}: no damping.\\
   \ttt{SERIES}: damping value decreases in each step.\\
   \ttt{STATIC}: static damping value.\\
   \ttt{DYNAMIC}: dynamic damping value adjustment (see Ref.~\cite{cances2000can}).\\
   The default is \ttt{SERIES}.
   \item [\texttt{maxCycles}]\hfill \\
   Maximum number of SCF cycles. Default is $100$.
   \item [\texttt{energyThreshold}]\hfill \\
   Convergence criterion for the energy. Default is \ttt{5e-8}.
   \item [\texttt{rmsdThreshold}]\hfill \\
   Convergence criterion for the density matrix. Default is \ttt{1e-8}.
  \item [\texttt{diisThreshold}]\hfill \\
   The convergence criterion for the DIIS ([F,P]) error. Default is \ttt{5e-7}. Note that if two out of these three convergence criteria are fulfilled, the SCF is considered converged.
   \item[\texttt{allowNotConverged}]\hfill \\
   If the maximum number of SCF cycles is reached, Serenity will continue even with non-converged orbitals. By default \ttt{false}.
   \item [\texttt{minimumLevelshift}]\hfill \\
   Minimum diagonal level-shift to be used. Default is $0.0$. Typical values are $0.3$ or $0.1$. Setting this to values larger than $0$ will
   help non-converging SCF procedures. However, it increases the number of iterations.
   \item [\texttt{rohf}]\hfill \\
   Can be used to invoke a restricted open-shell HF or KS calculation (ground-state). 
   Available are two flavors: \ttt{SUHF} \cite{Andrews1991} and \ttt{CUHF} \cite{Tsuchimochi2010}.
   Both are implemented as constrained UHF procedures. SUHF also requires a parameter (\texttt{suhfLambda}) which interpolates between
   UHF ($\lambda = 0$) and ROHF ($\lambda = \infty$). As a result, \ttt{SUHF} does not yield the \textit{exact} ROHF wavefunction with
   zero spin contamination in practice. \ttt{SUHF} may also struggle with convergence issues for $\lambda > 10$ and \ttt{CUHF} should
   basically always be preferred. Defaults to \ttt{NONE}.
   \item [\texttt{suhfLambda}]\hfill\\
   Scaling parameter for the SUHF procedure. Defaults to \ttt{0.01}.
   \item [\texttt{degeneracyThreshold}]\hfill\\
   Defaults to \ttt{0.0}. If set to something other than zero, fractionally occupies orbitals whose difference in eigenvalues is below this double.
\end{description}
\subsubsection{Advanced Keywords}
\begin{description}
    \item [\texttt{writeRestart}]\hfill \\
    Interval to write checkpoints during SCF. Default is $5$.
   \item [\texttt{seriesDampingStart}]\hfill \\
    Start value for the series damping. Default is $0.7$.
    \item [\texttt{seriesDampingEnd}]\hfill \\
    End value for the series damping. Default is $0.2$.
    \item [\texttt{seriesDampingStep}]\hfill \\
    Increment value for the series damping. Default is $0.05$.
    \item [\texttt{seriesDampingInitialSteps}]\hfill \\
    Number of initial steps without increment. Default is $2$.
    \item [\texttt{staticDampingFactor}]\hfill \\
    Damping value for the static damping. Default is $0.7$.
    \item [\texttt{endDampErr}]\hfill \\
    $\text{[F,P]}$ value at which the damping ends. Default is $0.05$.
    If the SCF has problems converging it may help to stop the damping at a lower error. However, this can
    increase the number of SCF iterations needed significantly.
    \item [\texttt{diisFlush}]\hfill \\
    Number of cycles after which the DIIS starts anew without previously stored data.
    \item [\texttt{diisStartError}]\hfill \\
    $\text{[F,P]}$ value at which the DIIS starts. Default is $0.05$.
    \item [\texttt{diisMaxStore}]\hfill \\
    The number of Fock Matrices stored in the DIIS. Default is $10$. If oscillations in the SCF occur, it can
    help to change this number. Typical values are between $5$ and $15$.
    \item [\texttt{useLevelshift}]\hfill \\
    Use a level-shift for the diagonal elements of the virtual--virtual Fock matrix block. Default is \ttt{true}.
    \item [\texttt{useOffDiagLevelshift}]\hfill \\
    Use a scaling for the virtual--occupied Fock matrix block. Default is \ttt{false}.
    \item [\texttt{canOrthThreshold}]\hfill \\
    Minimum eigenvalue of the overlap matrix to be tolerated. Default is 1e-7. This eliminates linear dependencies
    in the basis set. The default threshold is quite conservative. Linear dependencies can lead to numerical instability
    for procedures that depend explicitly on the orbital coefficients, e.g. coupled cluster.
    \item [\texttt{useADIIS}]\hfill \\
    Use the A-DIIS procedure \cite{hu2010} at the beginning of the SCF. Default is \ttt{false}. This interpolation scheme
    includes an energy minimization criterion into the Fock matrix update. The A-DIIS is supposed to be stable even for a
    bad initial guess. The A-DIIS is turned off automatically when the DIIS procedure starts.
    \item [\texttt{breakUnrestrictedSymmetry}]\hfill\\
    Defaults to \ttt{true} and breaks the symmetry of $\alpha$ and $\beta$ orbitals in an unrestricted initial guess if their number is identical.
 \end{description}

\subsection{Basis Block}\label{sec:system:basis}
All settings available pertaining the usage and generation of basis sets.
The basis sets can be found in \texttt{data/basis}. These are not exclusive. Any folder with \textsc{Turbomole}
style basis sets can be given and every file can be added to it, given that the file names are spelled in
uppercase.

\subsubsection{Example Input}
\begin{lstlisting}
  +system
    name methyl
    geometry methyl.xyz
    method HF
    charge 0
    spin 1
    +basis
      label def2-SVP
    -basis
  -system
 \end{lstlisting}
\subsubsection{Basic Keywords}
\begin{description}
 \item [\texttt{label}]\hfill \\
 Basis set name. \serenity will search for a file with the given label in uppercase at the \ttt{basisLibPath}. Default is \ttt{6-31GS}.

\end{description}
\subsubsection{Advanced Keywords}
\begin{description}
    \item [\texttt{makeSphericalBasis}]\hfill \\
    Use spherical harmonics basis. Default is \ttt{true}. If \ttt{false}, a Cartesian basis set is constructed.
    \item [\texttt{basisLibPath}]\hfill \\
    Basis set file location (default taken from environment variable).
    \item [\texttt{auxJLabel}]\hfill \\
    Auxiliary basis set name for Coulomb interactions (RI). Default is \ttt{RI\_J\_Weigend}.
    \item [\texttt{auxJKLabel}]\hfill \\
    Auxiliary basis set name for Coulomb and exchange interactions (RI). Default is none.
    \item [\texttt{auxCLabel}]\hfill \\
    Auxiliary basis set name for Correlation interactions (RI). If none is given, \serenity will search for a
    basis-set file for the given basis with a filename \ttt{label}-RI-C.
    \item [\texttt{firstECP}]\hfill \\
    Atomic number threshold for the use of ECPs if available in the basis-set file. Default is $37$ (starting from the fifth period).
    It is possible to manipulate/add the ECP definition to any basis-set file. For an example see the DEF2-SVP basis-set file.
    \item [\texttt{integralThreshold}]\hfill \\
    Threshold for the prescreening in integral evaluations. Default it is calculated as $1\cdot 10^{-8}/(3 M)$, where $M$ is
    the number of Cartesian basis functions.
    \item [\texttt{integralIncrementThresholdStart}]\hfill \\
    Initial integral prescreening threshold. Default is 1e-8.
    \item [\texttt{integralIncrementThresholdEnd}]\hfill \\
    Final integral prescreening threshold. Default it is equal to \texttt{integralThreshold}.
    \item [\texttt{incrementalSteps}]\hfill \\
    SCF-step interval to fully calculate the Fock matrix instead of only the Fock matrix increment due to increments in the density matrix. Default is $5$.
    \item [\texttt{densFitJ}]\hfill \\
    The density fitting approach to be used for Coulomb contributions. By default, the \ttt{RI} approximation is used. Furthermore, different approaches using the Cholesky decomposition can be chosen, namely the generic Cholesky decomposition \ttt{CD}, the atomic Cholesky decomposition \ttt{ACD} and the atomic-compact Cholesky decomposition \ttt{ACCD}. The full four center integrals may be used by selecting \ttt{NONE} for \ttt{densFitJ}.
    \item [\texttt{densFitK}]\hfill \\
    The density fitting approach to be used for exchange contributions. By default, no density fitting is used. Besides the \ttt{RI} approximation, different approaches using the Cholesky decomposition, namely the generic Cholesky decomposition \ttt{CD}, the atomic Cholesky decomposition \ttt{ACD} and the atomic-compact Cholesky decomposition \ttt{ACCD}, can be chosen.
    \item [\texttt{densFitLRK}]\hfill \\
    The density fitting approach to be used for long-range exchange contributions. By default, no density fitting is used. Besides the \ttt{RI} approximation, different approaches using the Cholesky decomposition, namely the generic Cholesky decomposition \ttt{CD}, the atomic Cholesky decomposition \ttt{ACD} and the atomic-compact Cholesky decomposition \ttt{ACCD}, can be chosen.
    \item [\texttt{densFitCorr}]\hfill \\
    The density fitting approach to be used to calculate correlation related contributions. By default, the \ttt{RI} approximation is used. Furthermore, different approaches using the Cholesky decomposition can be chosen, namely the generic Cholesky decomposition \ttt{CD}, the atomic Cholesky decomposition \ttt{ACD} and the atomic-compact Cholesky decomposition \ttt{ACCD}. The full four center integrals may be used by selecting \ttt{NONE} for \ttt{densFitCorr}.
    \item [\texttt{cdThreshold}]\hfill\\
    Decomposition threshold for all Cholesky decomposition based methods. The default is \ttt{1e-6}.
    \item [\texttt{extendSphericalACDShells}]\hfill\\
    The scheme to generate complete spherical basis function products represented as a spherical function. \ttt{NONE} uses the pure product of the spherical basis functions. With the default setting \ttt{SIMPLE} a single additional basis function is added to account for missing contributions. Using \ttt{COMLETE} the full product is approximated.
    \item [\texttt{secondCD}]\hfill\\
    The decomposition threshold for the secondary decomposition. This threshold is used if an ACD or ACCD auxiliary basis set is generated for a spherical basis set. The default value is \ttt{1e-8}.
    \item [\texttt{cdOffset}]\hfill\\
    Offset value used if an ACD or ACCD auxiliary basis set is generated for spherical basis sets. The default value is \ttt{1e-2}.
    \item[\texttt{intCondition}]
    Threshold to determine whether a four center integral should be cached or not.
    The caching criterion is based on the number of primitive  functions entering each basis function and their angular
    momentum. In general, integrals over highly contracted basis functions with low angular momentum are more likely to
    be cached. By default, \ttt{-1} (no integral caching is used). For small to medium-sized molecules
    (\emph{e.g.} up to 3000 basis functions) a threshold of \ttt{4} gives a balanced speed/memory-usage. The larger the
    threshold the more selective will the integral caching be.
 \end{description}

\subsection{Grid Block}\label{sec:system:grid}
All settings available to tune the generation of integration grids, which are used e.g. to evaluate exchange--correlation or nonadditive kinetic energy contributions.
\subsubsection{Example Input}
\begin{lstlisting}
  +system
    name methyl
    geometry methyl.xyz
    method DFT
    charge 0
    spin 1
    +grid
      accuracy 5
      smallGridAccuracy 3
    -grid
  -system
 \end{lstlisting}
\subsubsection{Basic Keywords}
\begin{description}
  \item [\texttt{accuracy}]\hfill \\
  Overall accuracy of the integration grid. Range: 1 to 7, with 7 being most accurate.\\ Default is $4$.
 \item [\texttt{smallGridAccuracy}]\hfill \\
 Accuracy for intermediate grids (e.g. a small grid can be used during an SCF, but once converged the energy is evaluated again on the larger grid). Range: 1 to 7, with 7 being most accurate. Default is $2$.
\end{description}
\subsubsection{Advanced Keywords}
\begin{description}
    \item [\texttt{gridType}]\hfill \\
    Procedure how to combine atomic grids. Default is \ttt{SSF}.
    All options are:\\
    \ttt{BECKE}: the fuzzy cells \cite{beck1988}.\\ 
    \ttt{SSF}: the algorithm by Stratmann, Scuseria and Frisch \cite{stra1996}.\\ 
    \ttt{VORONOI}: sharp cuts between atoms.
    \item [\texttt{radialGridType}]\hfill \\
    Procedure how to construct the radial part of the grid. Default is \ttt{AHLRICHS}.
    All options are:\\
    \ttt{BECKE}: \cite{beck1988, Gill2003} \\
    \ttt{HANDY}: \cite{Murray1993, Gill2003} \\
    \ttt{AHLRICHS}: \cite{Treutler1995, krack1998adaptive} \\
    \ttt{KNOWLES}: \cite{Mura1996, Gill2003} \\
    \ttt{EQUI}: equidistant points
    \item [\texttt{sphericalGridType}]\hfill \\
    Procedure how to construct the spherical part of the grid. Currently the only option and default is 
    the point distribution according to the work of Lebedev (\ttt{LEBEDEV}).
    \item [\texttt{gridPointSorting}]\hfill \\
    Enable or disable grid point sorting in order to enhance grid evaluation efficiency. Note that it should be
    set to \ttt{false} if using the \ttt{withAtomInfo} option within the \ttt{ExportGridTask}. Default is \ttt{true}.
    \item [\texttt{blocksize}]\hfill \\
    Size of blocked grid data  for parallelisation. Default is $128$.
    \item [\texttt{blockAveThreshold}]\hfill \\
    Threshold for average weights of blocks of data. Setting this to zero will increase accuracy and computational time of the numerical 
    integration for DFT type calculations. Default is 1e-11.
    \item [\texttt{basFuncRadialThreshold}]\hfill \\
    Threshold for the evaluation of basis functions on grid points. Setting this to zero will increase accuracy and computational time of the numerical 
    integration for DFT type calculations. Default is 1e-9.
    \item [\texttt{weightThreshold}]\hfill \\
    Threshold for grid point reduction. Default is 1e-14.
    \item [\texttt{smoothing}]\hfill \\
    Smoothing parameters for grid cells. Default is 3. This is the number of recursions on Becke's smoothing 
    function if \ttt{BECKE} is chosen as \ttt{gridType}, see Ref.~\cite{beck1988}.
 \end{description}


\subsection{DFT Block}\label{sec:system:dft}
The \ttt{DFT} block contains all options pertaining only DFT calculations.
Note that changes in this block do not automatically enable the \ttt{method dft}
keyword from the general system settings.
This keyword is still required in order to use \ttt{DFT} instead of the default \ttt{HF}.
\subsubsection{Example Input}
\begin{lstlisting}
  +system
    name methyl 
    geometry methyl.xyz
    method DFT 
    charge 0
    spin 1
    +dft
      functional PBE0
    -dft
  -system
 \end{lstlisting}
\subsubsection{Basic Keywords}
\begin{description}
  \item [\texttt{functional}]\hfill \\
  The exchange--correlation energy functional to use. A full list of functionals is given in the table below.
  The default is \ttt{BP86}.
 \item [\texttt{dispersion}]\hfill \\
 The version of Grimme's dispersion correction. The default is \ttt{NONE}. The options are:
 No correction (\ttt{NONE}), the third set of parameters (with zero damping, also called D3(0)) (\ttt{D3}),
 the third set of parameters (with zero damping, also called D3(0)) and 3 center correction term (\ttt{D3ABC}),
 the third set of parameters with Becke-Johnson damping (\ttt{D3BJ}),
 or the third set of parameters with Becke-Johnson damping and 3 center correction term (\ttt{D3BJABC}).
\end{description}

The following two tables list all possible exchange--correlation energy functionals and kinetic energy functionals.
The latter may be used for embedding calculations :
\begin{table}[H]\small \centering \begin{tabular}{|>{\ttfamily}c|l|>{\ttfamily}c|l|} \hline
\multicolumn{4}{|c|}{\textbf{Kinetic Energy Functionals}} \\ \hline
Functional & \multicolumn{1}{c|}{Notes} & Functional & \multicolumn{1}{c|}{Notes} \\ \hline
NONE     & No functional will be used. && \\ \hline
\hline \multicolumn{4}{|c|}{LDA} \\ \hline
TF       & & &\\ \hline
\hline \multicolumn{4}{|c|}{GGA} \\ \hline
PW91K    &                       &PBE2KS   & sDFT optimized PBE2K\\ \hline
LLP91K   &                       &PBE3K    & \\ \hline
LLP91KS  & sDFT optimized LLP91K &PBE4K    & \\ \hline
PBE2K    &                       &E2000K   & \\ \hline
\end{tabular}\end{table}
\begin{table}[H]\small \centering \begin{tabular}{|>{\ttfamily}c|l|>{\ttfamily}c|l|} \hline
\multicolumn{4}{|c|}{\textbf{Exchange--Correlation Energy Functionals}} \\ \hline
Functional & \multicolumn{1}{c|}{ Notes} & Functional & \multicolumn{1}{c|}{ Notes} \\ \hline
NONE     & No functional will be used. & & \\ \hline
\hline \multicolumn{4}{|c|}{LDA} \\ \hline
SLATER   & & LDA       & \\ \hline
VWN3     & & LDAERF    & \\ \hline
VWN5     & & LDAERF\_JT& \\ \hline
\hline \multicolumn{4}{|c|}{GGA} \\ \hline
OLYP     & &B97      & \\ \hline
BLYP     & &B97\_1   & \\ \hline
PBE      & &B97\_2   & \\ \hline
BP86     & &B97\_D   & \\ \hline
KT1      & &KT3      & \\ \hline
KT2      & &PW91     & \\ \hline
\hline \multicolumn{4}{|c|}{Hybrid} \\ \hline
PBE0     & & B3P86    & As implemented in Turbomole \\ \hline
BHLYP    & & B3P86\_G & As implemented in Gaussian \\ \hline
B3LYP    & As implemented in Turbomole & BPW91    & \\ \hline
B3LYP\_G & As implemented in Gaussian & HF       & Performs a HF calculation.\\ \hline
\hline \multicolumn{4}{|c|}{Double Hybrid} \\ \hline
B2PLYP   & &ROB2PLYP & \\ \hline
B2GPPLYP & &B2PIPLYP & \\ \hline
DSDBLYP  & &B2PPW91  & \\ \hline
DSDPBEP86& &DUT      & \\ \hline
B2KPLYP  & &PUT      & \\ \hline
B2TPLYP  & &&\\ \hline
\hline \multicolumn{4}{|c|}{Range-Separated Hybrid} \\ \hline
CAMB3LYP   &   & WB97      & Instabilities might occur in \texttt{libxc}. \\ \hline
LCBLYP     &   & WB97X     & See \texttt{WB97}. \\ \hline
LCBLYP\_47 &   & WB97X\_V  & See \texttt{WB97}. \\ \hline
LCBLYP\_100&   & WB97X\_D  & See \texttt{WB97}. \\ \hline
\hline \multicolumn{4}{|c|}{Model Potential} \\ \hline
SAOP     & & &\\ \hline
\end{tabular}\end{table}

\subsection{CUSTOMFUNC Block}\label{sec:system:customfunc}
This block allows to customize density functionals. As soon as the \texttt{basicFunctionals} list contains an entry, the custom functional is active, \textit{i. e.}
the custom functional takes precedence over a functional defined in the DFT block.

\subsubsection{Basic Keywords}
\begin{description}
  \item [\texttt{impl}]\hfill \\
  The external library from which the basic functionals are taken. Default is \ttt{Libxc}, the alternative is \ttt{XCFun}.
  \item [\texttt{basicFunctionals}]\hfill \\
  A list of basic functionals that together make up the custom functional. Note that all of them have to be available in the same functional library. Default is an empty list (\ttt{\{\}}).
  \item [\texttt{mixingFactors}]\hfill \\
  A list of doubles by which the corresponding basic functionals are weighted. Default is a list containing a \ttt{$1.0$}.
  \item [\texttt{hfExchangeRatio}]\hfill \\
  A double defining the amount of Hartree--Fock exchange. Default is \ttt{$0.0$}.
  \item [\texttt{hfCorrelRatio}]\hfill \\
  A double defining the amount of MP2 correlation. Default is \ttt{$0.0$}.
  \item [\texttt{lrExchangeRatio}]\hfill \\
  A double denoting the amount of long-range Hartree--Fock exchange. Default is \ttt{$0.0$}.
  \item [\texttt{mu}]\hfill \\
  The range-separation parameter. Default is \ttt{$0.0$} meaning no range-separation.
  \item [\texttt{ssScaling}]\hfill \\
  Same-spin scaling parameter. Default is \ttt{$1.0$}.
  \item [\texttt{osScaling}]\hfill \\
  Opposite-spin scaling parameter. Default is \ttt{$1.0$}.
\end{description}

\subsubsection{Example Input}
A B3LYP variant with a higher amount of Hartree--Fock exchange.
\begin{lstlisting}
  +system
    name methyl
    geometry methyl.xyz
    method DFT
    charge 0
    spin 1
    +customfunc
      basicfunctionals {x_slater x_b88 c_lyp c_vwn}
      mixingfactors {0.65 0.72 0.81 0.19}
      hfExchangeRatio 0.35
    -customfunc
  -system
 \end{lstlisting}

\subsection{PCM Block}\label{sec:system:pcm}

The PCM block contains all options for implicit solvent models. Note that this block can
be used not only in the system definition but in the task definition, too. If set for
a system, the PCM settings will apply to the system. For tasks, the PCM settings will apply
to the task, which may affect multiple systems. For the embedding step of FDE-type
calculations any system-specific PCM-settings are ignored and only the task specific settings
are applied.

GEPOL-type \cite{Pascual-Ahuir1987} (solvent-excluded and solvent-accessible-surface) and
a Delley-type \cite{Delley2006} surface are available. Note, that the GEPOL-type surfaces
show discontinuities at the intersection between spheres surrounding atoms. The Delley-type
surface does not show these problems.

The parameters for the tabulated solvents are taken from
\url{{https://pcmsolver.readthedocs.io}}.
Note that this data is provided for convenience and may contain errors.

A detailed description of the implementation of the Delley-type surface is given in Sec.~\ref{sec:MolcSurface}.
Analytical gradients are implemented for single systems and CPCM only.

For single systems the calculation of a cavity formation energy using the Pierotti--Claverie scaled particle
theory formula \cite{langlet1988improvements} is possible. For the calculation of this energy a Van-der-Waals
type surface (DELLEY-type surface, Van-der-Waals-radii) is used.


\subsubsection{Example Input}
\begin{lstlisting}
  +system
    name water  
    geometry water.xyz
    method HF  
    charge 0
    spin 0
    +pcm 
      use true
      solverType CPCM 
      solvent ETHANOL
    -pcm 
  -system
 \end{lstlisting}
\subsubsection{Basic Keywords}
\begin{description}
  \item [\texttt{use}]\hfill \\
  If true, the implicit solvent model is used. The default is \ttt{false}.
 \item [\texttt{solverType}]\hfill \\
 The type of solver continuum model to be used. \ttt{CPCM} selects the continuum-like PCM and 
 \ttt{IEFPCM} selects the integral equation formalism variant of PCM. \ttt{CPCM} is the default.
 \item [\texttt{solvent}]\hfill \\
 Select a predefined and tabulated solvent. The default is \ttt{WATER}. All options are given in the table below.
 The manual solvent definition can be triggered by selecting \ttt{EXPLICIT} as the solvent. If done so, the keywords 
 \ttt{eps} (static permittivity) and \ttt{probeRadius} (solvent probe radius) have to be set manually.
 \item [\texttt{cavityFormation}]\hfill \\
 Enable the calculation of a cavity formation energy using scaled particle theory. The default
 is \ttt{false}.
\end{description}
\subsubsection{Advanced Keywords}
\begin{description}
    \item [\texttt{scaling}]\hfill \\
    If true, the atom-radii used for the cavity construction are scaled by a factor of $1.2$. This is commonly done 
    when using the \texttt{BONDI} (VdW radii) of the atoms for cavity construction. For the \ttt{UFF} radii set it is 
    advised to turn this scaling off. The default is \ttt{true}.
    \item [\texttt{cavity}]\hfill \\
    The type of cavity used with the PCM. The default is a Delley-type cavity (\ttt{DELLEY}). Other options are 
    a primitive version of the GEPOL cavity which is available as a solvent excluded surface \ttt{GEPOL\_SES} and 
    solvent accessible surface \ttt{GEPOL\_SAS}.
    \item [\texttt{radiiType}]\hfill \\
    The atomic radii-set to be used in the cavity construction. The default is the Bondi-Mantina radii set \ttt{BONDI}.
    Note that not for all atoms radii are tabulated in this set. For the \ttt{UFF} radii set all atoms are supported.
    If the \texttt{UFF} radii set is used, it is advised to select \ttt{scaling} as \ttt{false}.
    \item [\texttt{minDistance}]\hfill \\
    Minimum distance tolerated between surface grid points in Bohr. The default is $0.2$. This prevents singularities
    for extremely close points.
    \item [\texttt{minRadius}]\hfill \\
    Minimal radius in Bohr for additional spheres not centered on atoms (GEPOL-type surfaces only). The default is $0.377$ au.
    \item [\texttt{correction}]\hfill \\
    Correction $k$ for the apparent surface charge scaling factor in the CPCM. The default is $0.0$. In the related 
    COSMO solvation model $k$ is chosen as $0.5$. This has little effect on solvent with a high static permittivity
    but may be important for others.
    \item [\texttt{probeRadius}]\hfill \\
    Radius in Bohr of the spherical probe approximating a solvent molecule. The default is $1.0$. This is only used if
    the explicit solvent definition is used by selecting \ttt{solvent=EXPLICIT}. Otherwise, the tabulated
    data is used.
    \item [\texttt{eps}]\hfill \\
    The static dielectric permittivity of the medium. The default is $1.0$. This is only used if 
    the explicit solvent definition is used by selecting \ttt{solvent=EXPLICIT}. Otherwise, the tabulated
    data is used.  
    \item [\texttt{patchLevel}]\hfill \\
    Wavelet cavity mesh patch level for GEPOL cavities. The default is $0$. This increases the resolution of the 
    GEPOL-type cavity at the intersection between spheres.
    \item [\texttt{overlapFactor}]\hfill \\
    Maximum ratio of a new sphere to be allowed to be covered within the already present ones. This is only used for 
    GEPOL-type surfaces. The default is $0.7$.  
    \item [\texttt{cacheSize}]\hfill \\
    Maximum number of two center integrals to be stored in memory. The default is $128$.
    \item [\texttt{lLarge}]\hfill \\
    Angular momentum used for the spherical grid construction for non-hydrogen atoms for DELLEY-type surfaces. The default
    is $7$, which corresponds to 110 spherical grid points for each  non-hydrogen atom.
    \item [\texttt{lSmall}]\hfill \\
    Angular momentum used for the spherical grid construction for hydrogen atoms for DELLEY-type surfaces. The default
    is $4$, which corresponds to 50 spherical grid points for each hydrogen atom.
    \item [\texttt{alpha}]\hfill \\
    The sharpness parameter for the molecular surface model function for DELLEY-type surfaces. The default
    is $50$, which corresponds to fairly sharp surfaces. Since the model employs an exponential 
    molecular surface model function, results are relatively stable with variation of the parameter. The surface 
    will typically become obese for values smaller than $10$, which leads to a lowered dielectric interaction with 
    the continuum.
    \item [\texttt{projectionCutOff}]\hfill \\
    The cut-off for the projection to the molecular surface for DELLEY-type surfaces. Each atom-wise grid 
    point is projected to the molecular surface. If the molecular surface is too far away, the grid point is 
    assumed to be unimportant and covered within the molecule. The value is given in atomic units. Its default is 
    $5.0$.
    \item [\texttt{oneCavity}]\hfill \\
    Only cavity points connected to the point with the most extreme
    x-coordinate are kept after cavity construction. Two points are connected if they are within
    $connectivityFactor\times probeRadius$ from another. This can be used to eliminate cavity points within a 
    molecular cluster. The default is \ttt{false}.
    \item [\texttt{connectivityFactor}]\hfill \\
    The scaling factor for point connection. The default is $2.0$. Increase this for low
    resolution cavities. This is only used for \ttt{oneCavity=true}.
    \item [\texttt{numberDensity}]\hfill \\
    The number density (in a.u.) of the solvent. The default is $1.0$. This is only used if 
    \ttt{solvent=EXPLICIT} and \ttt{cavityFormation=true} are given.
    \item [\texttt{temperature}]\hfill \\
    The temperature used in the scaled particle theory treatment of the cavity formation. The
    default is $298.15$. This is only used if \ttt{cavityFormation=true} is given.
    \item [\texttt{cavityProbeRadius}]\hfill \\
    The probe radius used for the calculation of the cavity formation energy. The default is $5.0$.
    This is only used if \ttt{solvent=EXPLICIT} and \ttt{cavityFormation=true} are given. 
    \item [\texttt{saveCharges}]\hfill \\
	Switch to determine whether we want to save the PCM charges. The default is \ttt{false}.
 \end{description}

\begin{table}[H]\small \centering \begin{tabular}{|>{\ttfamily}c|l|} \hline 
\multicolumn{2}{|c|}{\textbf{PCM Solvents}} \\ \hline
Solvent & \multicolumn{1}{c|}{ Keyword} \\ \hline 
Water               & WATER, H2O \\ \hline 
Propylenecarbonate  & PROPYLENECARBONATE, C4H6O3 \\ \hline 
Dimethylsulfoxide   & DIMETHYLSULFOXIDE, DMSO, C2H6OS \\ \hline 
Nitromethane        & NITROMETHANE, CH3NO2 \\ \hline 
Acetonitrile        & ACETONITRILE, CH3CN \\ \hline 
Methanol            & METHANOL, MEOH, CH3OH \\ \hline 
Ethanol             & ETHANOL, ETOH, CH3CH2OH  \\ \hline 
Acetone             & ACETONE, C2H6CO \\ \hline 
1,2-Dichlorethane   & 1,2-DICHLORETHANE, DICHLORETHANE, C2H4CL2  \\ \hline 
Methylenechloride   & METHYLENECHLORIDE, CH2CL2  \\ \hline
Tetrahydrofurane    & TETRAHYDROFURANE, THF, C4H8O  \\ \hline 
Aniline             & ANILINE, C6H5NH2 \\ \hline 
Chlorobenzene       & CHLOROBENZENE, C6H5CL \\ \hline 
Chloroform          & CHLOROFORM, CHCL3 \\ \hline 
Toluene             & TOLUENE, C6H5CH3 \\ \hline 
1,4-Dioxane         & DIOXANE, C4H8O2, 1,4-DIOXANE \\ \hline 
Benzene             & BENZENE, C6H6 \\ \hline 
Carbontetrachloride & CARBONTETRACHLORIDE, CCL4 \\ \hline 
Cyclohexane         & CYCLOHEXANE, C6H12 \\ \hline 
N-Heptane           & N-HEPTANE, C7H16 \\ \hline 
Explicit            & EXPLICIT (triggers explicit solvent definition) \\ \hline 
\end{tabular}\end{table}

\subsection{EField Block}\label{sec:system:efield}
The \ttt{EField} block contains options regarding external electric fields. Currently,
two strategies for using electric fields can be chosen. First, a molecular system is placed between 
two circular plates consisting of opposing point charges which will be referred to as numerical
electric field. The algorithm used for constructing these capacitor plates loosely resembles the
method used in TITAN~\cite{stuyver2020}.
Second the so-called analytical electric fields are produced by incorporating
dipole integrals into the core Hamiltonian.
Parameters of this algorithm can be adjusted here.
\subsubsection{Example Input}
Numerical efield:
\begin{lstlisting}
  +system
    name water 
    geometry water.xyz
    +efield
      use true
      pos1 {0.0 0.0 0.0}
      pos2 {0.0 0.0 1.0}
      distance 50
      nRings 50
      fieldStrength 1e-3
      nameOutput efieldplate
    -efield
  -system
 \end{lstlisting}
Analytical efield:
\begin{lstlisting}
  +system
    name water 
    geometry water.xyz
   +efield
     use true
     analytical true
     pos1 {0.0 0.0 0.0}
     pos2 {0.0 0.0 1.0}
     fieldStrength 1e-3
   -efield
 -system
\end{lstlisting}
\subsubsection{Basic Keywords}
\begin{description}
  \item [\texttt{use}]\hfill \\
  A boolean to trigger the electric field usage.
  The default is \ttt{false}.
  \item [\texttt{analytical}]\hfill \\
  A boolean to trigger the analytical electric field usage. If chosen \ttt{false}
  the numerical field is used.
  The default is \ttt{false}.
 \item [\texttt{pos1}]\hfill \\
  The electric field will point from \ttt{pos1} to \ttt{pos2}, \emph{i.e.}
  the positively charged plates will appear on the side of \ttt{pos1} and 
  the negatively charged plates will appear on the side of \ttt{pos2}.
  The default for \ttt{pos1} is \ttt{\{0.0 0.0 0.0\}}.
 \item [\texttt{pos2}]\hfill \\
  See \ttt{pos1}.
  The default for \ttt{pos2} is \ttt{\{0.0 0.0 1.0\}}.
 \item [\texttt{distance}]\hfill \\
 Specifies the distance between the center point of the capacitor plates and 
 the position in the middle between \ttt{pos1} and \ttt{pos2} in \AA. The default is \ttt{50.0}.
 \item [\texttt{nRings}]\hfill \\
 The capacitor plates are organized in rings made up of point charges. This keyword specifies the
 number of rings to construct. The default is \ttt{50}.
 \item [\texttt{radius}]\hfill \\
 The radius between two adjacent rings in \AA. The default is \ttt{1.0}.
 \item [\texttt{fieldStrength}]\hfill \\
 The strength of the electric field in atomic units. Note that 1 a.u. is
 approximately equivalent to $5.14\cdot 10^{11}~\mathrm{V/m}$. The default is \ttt{1e-3}.
 \item [\texttt{nameOutput}]\hfill \\
 If specified, the electric field plates will be written to disk as an xyz-file. Negative
 charges will be printed as boron atoms (B), positive charges as xenon atoms (Xe).
\end{description}

\subsection{EXTCHARGES Block}\label{sec:system:extchrg}
The \ttt{EXTCHARGES} block contains options for using external charges. At the moment, only a path
to a file containing the charge positions and values can be defined. These charges will be included in
the system's core Hamiltonian. Furthermore, the energy from the interaction with these charges will be added to
the system. The interaction between the external charges is not included.
\subsubsection{Example Input}
\begin{lstlisting}
  +system
    name water
    geometry water.xyz
   +extcharges
     externalChargesFile point-charges.pc
   -extcharges
 -system
\end{lstlisting}
Charges and coordinates must be provided row-wise in the file. All coordinates must be given in Angstrom and charges
in atomic units.
\begin{lstlisting}
<total number of charges>
<charge-value> <x-coordinate> <y-coordinate> <z-coordinate>
<charge-value> <x-coordinate> <y-coordinate> <z-coordinate>
...
\end{lstlisting}
Example file:
\begin{lstlisting}
45678
-0.573704   1.513000  -5.452000   496.216000
 0.290609   2.398000  -5.884000   496.044000
 0.290609   1.863000  -4.590000   495.850000
-0.573704   4.253000   2.507000   503.760000
...
\end{lstlisting}
\subsubsection{Basic Keywords}
\begin{description}
  \item [\texttt{externalChargesFile}]\hfill \\
  The file path to the external charge file. If empty, no external charges are used. By default an empty string.
\end{description}


% ================
%    Task Input
% ================

\clearpage
\section{Tasks}
\label{sec:tasks}
The task block accepts different options depending on the task to be run.
These task specific options, as well as the options common to all tasks will be listed and explained
in the following section.\\
\\
The general form of a task block is the following.
Each task block is opened by a \ttt{+task} statement followed by the name of the
task to be run and is ended by a \ttt{-task} statement.\\
Furthermore, each task block accepts three types of systems. Supersystem (keywords: \ttt{super}),
active  systems (keywords: \ttt{system}, \ttt{active}, \ttt{act}) and environment systems
(\ttt{environment}, \ttt{env}) their meaning changes depending
on the task type. Additionally, a print level (keywords: \ttt{printLevel}) and timings
print level (keyword: \ttt{timingsPrintLevel}) are available
for every task. The effect of the print levels will depend on the task. More information
is given in Section~\ref{sec:tasksPrintLevels}.
The exact meaning of active versus environment system in the context of each task will be 
given in its respective section specifically.
A minimal task block for a SCF calculation could look like this:
\begin{lstlisting}
+task scf
  system water
-task
\end{lstlisting}
Note that some tasks may use a block structure similar to the system settings.
The blocks are opened and closed in the same way as before. The input for an
FDE calculation for two water molecules with the Huzinaga operator and the
PBE functional for the exchange--correlation interaction may look like this:
\begin{lstlisting}
+task FDE
  act waterA
  env waterB
  +emb
    naddXCFunc PBE
    embeddingMode HUZINAGA
  -emb
-task
\end{lstlisting}
Possible blocks are listed in the respective task documentation and their
structure shown in the following sections starting with~\ref{sec:scb:emb}.

\subsection{General Keywords}
\label{sec:tasksPrintLevels}
The following keywords are present, and can be changed in all tasks, regardless
of the other keywords present.
\begin{description}
  \item [\texttt{printLevel}]\hfill \\
Changing the print level may result in additional output which is indented by
a set of characters in order to allow for easy searches. Furthermore, the number
of eigenvalues printed to the output after converging a SCF calculation depends
on the print level.
\begin{table}[H]\small \centering \begin{tabular}{|>{\ttfamily}c|>{\ttfamily}c|>{\ttfamily}c|l|}\hline
  \multicolumn{1}{|c|}{Numerical value} & & \multicolumn{1}{c|}{Indent}&\multicolumn{1}{c|}{Number of printed eigenvalues}\\\hline
  0 & \texttt{MINIMUM}          &       & HOMO-LUMO gap    \\\hline
  1 & \texttt{NORMAL }[Default] &       & HOMO-9 to LUMO+9 \\\hline
  2 & \texttt{VERBOSE}          & ==V== & 0 to LUMO+9      \\\hline
  3 & \texttt{DEBUGGING}        & ==D== & ALL              \\\hline
\end{tabular}
\end{table}
\item [\texttt{timingsPrintLevel}]\hfill \\
   The amount of timings that will be printed to the output. This setting is closer to a
   debug/developer setting, yet it may also be interesting for an experienced user.
   The default value is $1$, a value of $0$ will remove many timings, higher
   values than $1$ may add additional timings.
\end{description}
\subsubsection{Example Input}
\begin{lstlisting}
  +task SCF
    actsystem water
    printLevel debugging
    timingsPrintLevel 0
  -task
\end{lstlisting}


\subsection{Common Sub-Block: Embedding Settings}
\label{sec:scb:emb}
The \texttt{EMB} block contains all settings for embedding calculations. Be aware: some defaults may vary between the tasks.
\subsubsection{Example Inputs}
\begin{lstlisting}
  +task fat
    system ammonia1
    system ammonia2
    +emb
      naddXcFunc PW91
      embeddingmode NADDFUNC
    -emb
  -task
\end{lstlisting}
\begin{lstlisting}
  +task fat
    system ammonia1
    system ammonia2
    +emb
      embeddingmode NADDFUNC
    -emb
    +customnaddxc
      basicFunctionals {x_slater}
      mixingFactors {0.8}
      hfExchangeRatio 0.2
    -customnaddxc
  -task
\end{lstlisting}
\subsubsection{Basic Keywords}
\begin{description}
  \item [\texttt{embeddingMode}]\hfill \\
    The type of embedding to be done. Possible options are:
    \ttt{NONE}, no treatment of Pauli-repulsion.
    \ttt{NADD\_FUNC}, the standard subsystem DFT formalism, employing two non-additive density functionals.
    This option is the default in most cases.
    \ttt{LEVELSHIFT}, a projector based embedding using only one non-additive density functional for exchange
    and correlation. This method was introduced by Manby and Miller.
    \ttt{RECONSTRUCTION}, embedding using potential reconstruction techniques.
    Furthermore, there are three more projector based embedding techniques that use only one non-additive
    density functional for exchange and correlation. These can be used with the keywords \ttt{HUZINAGA} and
    \ttt{FERMI\_SHIFTED\_HUZINAGA} or \ttt{FERMI}, as well as \ttt{HOFFMANN}.
    Related to projector based embedding techniques, absolutely localized molecular orbitals can be used with
    the keyword \ttt{ALMO}.
    Note that depending on the actual task, the selection of embedding modes can be limited. As an example,
    the calculation of nuclear gradients is only possible for \ttt{NADD\_FUNC} type calculations. Finally, \ttt{LOEWDIN} employs evaluations of the non-additive kinetic energy from non-orthogonal molecular orbitals of subsystems and making use of the Neumann expansion of the inverse overlap matrix. The latter option is in experimental stage and, therefore, is not recommended.
  \item [\texttt{naddXCFunc}]\hfill \\
    The exchange--correlation functional to be used in the evaluation of the non-additive term arising in the
    subsystem DFT formalism. The list of available functionals can be found in Section~\ref{sec:system:dft}.
    Note that double hybrid functionals are not fully supported yet. This can also be customized by invoking the \ttt{+customnaddxc} block (wich is a \ttt{customfunc}-type block, see Sec.~\ref{sec:system:customfunc}) in addition to the \ttt{emb} block, which will be prioritized over \ttt{naddXCFunc}.
  \item [\texttt{naddKinFunc}]\hfill \\
    The kinetic energy functional to be used in the evaluation of the non-additive term arising in the
    subsystem DFT formalism. The list of available functionals can be found in Section~\ref{sec:system:dft}. This can also be customized by invoking the \ttt{+customnaddkin} block (wich is a \ttt{customfunc}-type block, see Sec.~\ref{sec:system:customfunc}) in addition to the \ttt{emb} block, which will be prioritized over \ttt{naddKinFunc}.
  \item [\texttt{levelShiftParameter}]\hfill \\
    When using \ttt{embeddingMode LEVELSHIFT} this parameter determines the magnitude by which the environment
    orbitals are shifted. The default value is \ttt{1.0e+6}.
  \item [\texttt{dispersion}]\hfill \\
    The type of dispersion interaction correction to employ for the calculation of interactions between the
    embedded systems. For the different options, see Sec.~\ref{sec:system:dft}. By default \ttt{None}.
  \item [\texttt{partialChargesForCoulombInt}]\hfill \\
    Approximate Coulomb interactions between subsystems by representing a subsystem using partial charges.
  \item [\texttt{chargeModel}]\hfill \\
    Model of partial charges to be used for representing a subsystem (see Sec.~\ref{sec:PopAnalysisTask}).
  \item [\texttt{loewdinOrder}]\hfill \\
    An integer value setting up the truncation order of the Neumann series in the \ttt{LOEWDIN} embedding. Note that only terms up to and including the third order are implemented.
\end{description}
\subsubsection{Advanced Keywords}
\begin{description}
  \item [\texttt{calculateMP2Correction}]\hfill \\
    A flag to turn off the evaluation of the MP2 correction for double hybrid functionals. By default \ttt{True}.
  \item [\texttt{fullMP2Coupling}]\hfill \\
    If true, the MP2 contribution of the non--additive exchange-correlation energy for double hybrids captures
    the effect of environment orbital pairs on the active-pair amplitudes. By default \ttt{False}.
  \item [\texttt{longRangeNaddKinFunc}]\hfill \\
    Long-range non--additive kinetic energy for hybrid methods. Only used if \ttt{embeddingMode} is one of
    the projection based methods. This option is intended to mainly be used in conjunction with
    \ttt{embeddingMode LEVELSHIFT}. By default, this option is disabled\\ (\ttt{longRangeNaddKinFunc NONE}).
    For a list of possible functionals see the kinetic energy functionals in Section~\ref{sec:system:dft}. This can also be customized by invoking the \ttt{+customlongrangenaddkin} block (wich is a \ttt{customfunc}-type block, see Sec.~\ref{sec:system:customfunc}) in addition to the \ttt{emb} block, which will be prioritized over\\ \ttt{longRangeNaddKinFunc}.
  \item [\texttt{borderAtomThreshold}]\hfill \\
    The Mulliken population threshold used to determine if an orbital is considered ``distant'' or not.
    Only used in hybrid functional/projection schemes. By default \ttt{0.02}.
  \item [\texttt{basisFunctionRatio}]\hfill \\
    The minimum ratio of retained basis function shells needed in order to consider an atom to be not
    ``distant''. Only used in hybrid functional/projection schemes. By default \ttt{0.0}.
  \item [\texttt{truncateProjector}]\hfill \\
    A flag to truncate the projection operator in bottom-up calculations. Only used if any kind of projection
    technique is used. By default \ttt{False}.
  \item [\texttt{projecTruncThresh}]\hfill \\
    Total overlap threshold for the truncation of the projection operator. Only used if \\
    \ttt{truncateProjector true} and hence only useful in any kind of projection technique.
    By default \ttt{1.0e+1}.
  \item [\texttt{fermiShift}]\hfill \\
    An optional shift for the Huzinaga operator. Only used if\\ \ttt{embeddingMode FERMI\_SHIFTED\_HUZINAGA} is
    used. By default \ttt{1.0}.
  \item [\texttt{carterCycles}]\hfill \\
    Number of update cycles for the Zhang-Carter potential reconstruction. Only relevant if
    \ttt{embeddingMode RECONSTRUCTION} is used.
  \item [\texttt{potentialBasis}]\hfill \\
    The label of the basis set used for the potential reconstruction. Only relevant if\\
    \ttt{embeddingMode RECONSTRUCTION} is used.
  \item [\texttt{singValThreshold}]\hfill \\
    Threshold for the singular value decomposition in potential reconstruction. Only relevant if
    \ttt{embeddingMode RECONSTRUCTION} is used.
  \item [\texttt{lbDamping}]\hfill \\
    Damping for the density update in the van Leeuwen--Barends potential reconstruction. Only relevant if
    \ttt{embeddingMode RECONSTRUCTION} is used.
  \item [\texttt{lbCycles}]\hfill \\
    Number of update cycles for the van Leeuwen--Barends potential reconstruction. Only relevant if
    \ttt{embeddingMode RECONSTRUCTION} is used.
  \item [\texttt{smoothFactor}]\hfill \\
    The smoothing factor for the potential reconstruction. Only relevant if\\
    \ttt{embeddingMode RECONSTRUCTION} is used.
  \item [\texttt{naddXCFuncList}]\hfill \\
    A list of non-additive exchange--correlation functionals used for mixed embedding, where some subsystems are embedded two non-additive density functionals and others via projection. By default empty. Two elements are expected, with the first being the functional for the exactly embedded subsystems and the second entry as the functional for the approximately embedded subsystems.
  \item [\texttt{naddKinFuncList}]\hfill \\
    A list of non-additive kinetic energy functionals. By default empty.
  \item [\texttt{embeddingModeList}]\hfill \\
    A list of embeddingModes, allowing each subsystem to use a different embeddingMode in a mixed embedding scheme. By default empty. This list is expected to contain as many elements as there are systems for a specific tasks.
  \item [\texttt{loewdinWeights}]\hfill \\
    A vector of scaling factors used for Loewdin-based kinetic energy correction terms. The first value corresponds to the first-order term, the second value is related with the second term, etc.
\end{description}

\subsection{Common Sub-Block: Local Correlation Settings}
\label{sec:scb:lc}
The \texttt{LC} block contains all settings for calculations based on local-correlation schemes.
Some defaults may vary between the tasks and the pair natural orbital-(PNO-)macro flag
(\ttt{pnoSettings}) used (\ttt{TIGHT}, \ttt{NORMAL} or \ttt{LOOSE}).
All defaults set by the PNO-macro flag \ttt{pnoSettings} can be overwritten.

\subsubsection{Example Input}
\begin{lstlisting}
  +task MP2
    system water  
    mp2Type local
    +lc 
      pnoSettings TIGHT    #the PNO-macro flag
      pnoThreshold 3.3e-8  #overrides the value given by the macro
      useFrozenCore true
    -lc 
  -task
 \end{lstlisting}
\subsubsection{Basic Keywords}
\begin{description}
  \item [\texttt{pnoSettings}]\hfill \\
  The PNO-macro flag. This flag can be used to manipulate several settings at once. The default value is
  \ttt{NORMAL}. The other options are \ttt{LOOSE} and \ttt{TIGHT} with looser and tighter thresholds
  respectively. The actual values associated with this setting are given below and are different for
  coupled-cluster and MP2.
 \item [\texttt{useFrozenCore}]\hfill \\
 If true, the frozen core approximation is used. Frozen core orbitals are determined by the orbital energy
 threshold \texttt{energyCutOff}. By default, no frozen-core approximation (\ttt{false}) is used. Note
 that using a frozen-core approximation requires that the core and valence orbitals have been localized
 independently.
 \item [\texttt{linearScalingSigmaVector}]\hfill \\
 If true, the integrals required for the sigma vector construction in DLPNO-CCSD are calculated directly from
 integrals calculated during the main integral transformation. This requires slightly more memory, however,
 the amplitude optimization is accelerated by more than one order of magnitude. Note that this scheme is an
 additional approximation. It is sensible to set this to true for larger systems. Default is \ttt{true}.
\end{description}
\subsubsection{Advanced Keywords}
\begin{description}
    \item [\texttt{projectedEnvironment}]\hfill \\
    Apply an additional energy shift (1e+6 Hartree) for environment orbitals.
    This is turned off by default (\ttt{false}).
    \item [\texttt{useBPAlgorithm}]\hfill \\
    If true, the Boughton--Pulay algorithm is used for the selection of projected atomic orbitals. The size of the
    virtual domains is controlled by the threshold (\ttt{completenessThreshold}). This is turned off by default
    (\ttt{false}) in favor of a differential-overlap-integral based selection. The latter is argued to be more robust.
    \item [\texttt{completenessThreshold}]\hfill \\
    The Boughton--Pulay completeness threshold. The default is $0.02$. Lowering this threshold will lead to
    increased virtual domains if the Boughton--Pulay algorithm is used.
    \item [\texttt{doiPairThreshold}]\hfill \\
    Differential-overlap-integral pair-prescreening threshold. In combination with the pair energy threshold
    (\ttt{ccsdPairThreshold}), this threshold controls the number of orbital pairs.
    The default is 1e-5.
    \item [\texttt{doiPAOThreshold}]\hfill \\
    The differential-overlap-integral threshold for the projected atomic orbital (PAO) selection. Lowering this
    threshold leads to larger virtual PAO domains. The default depends on the PNO macro flag,
    \ttt{pnoSetting}. Setting this manually overrides the value given by the macro setting.
    \item [\texttt{collinearDipoleScaling}]\hfill \\
    The scaling factor of \ttt{ccsdPairThreshold} for the very-distant pair selection.
    The default is $0.01$.
    \item [\texttt{ccsdPairThreshold}]\hfill \\
    Pair energy threshold for close-pairs. The default depends on the PNO macro flag, \ttt{pnoSetting}.
    Setting this manually overrides the value given by the macro setting.
    \item [\texttt{triplesSCMP2Scaling}]\hfill \\
    The scaling factor of \ttt{ccsdPairThreshold} for the pair selection which are used for the triples
    selection in coupled cluster. The default is $0.1$.
    \item [\texttt{pnoThreshold}]\hfill \\
    Pair natural orbital occupation number selection threshold. The default depends on the PNO macro flag,
    \ttt{pnoSettings}. Setting this manually overrides the value given by the macro setting.
    \item [\texttt{tnoThreshold}]\hfill \\
    Triples natural orbital occupation number selection threshold. The default is 1e-9.
    \item [\texttt{singlesPNOFactor}]\hfill \\
    The scaling factor for \ttt{pnoThreshold} for the singles virtual domain construction. The default
    is $0.03$. This recovers the singles contributions nearly completely.
    \item [\texttt{pnoCoreScaling}]\hfill \\
    The scaling factor for the \ttt{pnoThreshold} for orbital pairs containing core orbitals. The
    default is $0.01$. Calculating the electron correlation correctly for core-like orbitals requires
    significantly stricter thresholds.
    \item [\texttt{orbitalToShellThreshold}]\hfill \\
    The orbital coefficients are prescreened based on their absolute value. If they are smaller than this
    threshold, integrals associated with them are neglected. The default depends on the PNO macro flag,
    \ttt{pnoSettings}. Setting this manually overrides the value given by the macro setting.
    \item [\texttt{mullikenThreshold}]\hfill \\
    The Mulliken population threshold which is used to determine the pair-wise fitting domains. Lowering this
    threshold leads to increased accuracy of the integral transformation but will increase the computational
    cost significantly. The default depends on the PNO macro flag,
    \ttt{pnoSettings}. Setting this manually overrides the value given by the macro setting.
    \item [\texttt{crudeDomainFactor}]\hfill \\
    The scaling factor for \ttt{mullikenThreshold} for the construction of the initial,
    crude fitting domains which are used for orbital pair prescreening. The default is $10.0$.
    \item [\texttt{crudeStrongTripFactor}]\hfill \\
    The scaling factor for \ttt{mullikenThreshold} for the construction of the strong triples
    fitting domains. The default is $10.0$. A strong triple is an orbital triple constructed from
    close orbital pairs only.
    \item [\texttt{crudeWeakTripFactor}]\hfill \\
    The scaling factor for \ttt{mullikenThreshold} for the construction of the weak triples
    fitting domains. The default is $100.0$.
    \item [\texttt{fockMatrixPrescreeningThreshold}]\hfill \\
    The Fock matrix prescreening threshold for local MP2. The default is 1e-5.
    \item [\texttt{doiNetThreshold}]\hfill \\
    Orbital coefficient prescreening threshold for differential-overlap-integral calculation. The default
    is 1e-7.
    \item [\texttt{paoOrthogonalizationThreshold}]\hfill \\
    The canonical orthogonalization threshold for non-redundant projected atomic orbital (PAO) construction.
    The initial PAO domains often contain linear dependencies. This is resolved by canonical orthogonalization of
    the PAO--PAO overlap matrix. The default for the smallest tolerated overlap matrix eigenvalues is
    1e-6.
    \item [\texttt{paoNormalizationThreshold}]\hfill \\
    Renormalization threshold for PAOs. By default 1e-6.
    \item [\texttt{maximumMemoryRatio}]\hfill \\
    Maximum ratio of the available memory used for integral storage. The default is $0.8$.
    \item [\texttt{dumpIntegrals}]\hfill \\
    If true, all pair-wise integrals are written to file. Warning: this may take a lot of disk space.
    The default is \ttt{false}.
    \item [\texttt{diisStartResidual}]\hfill \\
    Starting residual threshold for the DIIS. The default is $1.0$.
    \item [\texttt{dampingFactor}]\hfill \\
    The initial damping factor for amplitude update. The default is $0.4$.
    \item [\texttt{dampingChange}]\hfill \\
    The damping factor is lowered by this value in each iteration. The default is $0.1$.
    \item [\texttt{finalDamping}]\hfill \\
    The final damping factor for the amplitude update. The default is $0.0$.
    \item [\texttt{diisMaxStore}]\hfill \\
    The maximum number of DIIS vectors stored during amplitude optimization. The default is $10$.
    \item [\texttt{setFaiZero}]\hfill \\
    Force the $F_{ai}$ (occupied--virtual) block of the Fock matrix to be zero. The default is \ttt{true}.
    \item [\texttt{useTriplesCoreScaling}]\hfill \\
    If true, the \texttt{pnoCoreScaling} factor is used for the triples natural orbital truncation for
    triples containing core orbitals. The default is \ttt{false}.
    \item [\texttt{method}]\hfill \\
    Specify the local-correlation method (\ttt{DLPNO-CCSD/DLPNO-CCSD(T0)/DLPNO-MP2} or \ttt{NONE}). This is only used if the method
    is not specified via a task-keyword. By default \ttt{DLPNO-CCSD}.
    \item [\texttt{topDownReconstruction}]\hfill \\
    Enforce top-down \textit{ansatz} for potential reconstruction.
    \item [\texttt{extendedDomainScaling}]\hfill \\
    You may include additional pairs as close pairs in the sparse map / extended
    domain construction. By default turned off/default $\ttt{1.0}$.
    \item [\texttt{enforeceHFFockian}]\hfill \\
    Enforce the use of the HF Fock operator. By default \ttt{false}.
    \item [\texttt{reuseFockMatrix}]\hfill \\
    If true, \textsc{Serenity} will try to read the Fock matrix from disk or from memory.
    This avoids an additional Fock matrix construction. By default \ttt{true}.
    \item [\texttt{lowMemory}]\hfill \\
    If true, \textsc{Serenity} will limit the number of 3-center integrals stored in memory
    and recalculate integrals more often. By default \ttt{false}.
    \item [\texttt{useProjectedOccupiedOrbitals}]\hfill \\
    If false, projected orbitals that are occupied by the environment are removed from the virtual orbital space. By default \ttt{false}.
 \end{description}

The settings used in local correlation methods based on the PNO-approximation can
be adjusted using the macro-flags \ttt{LOOSE}, \ttt{NORMAL} or \ttt{TIGHT}.
These flags affect multiple parameters. The values set by the flags will vary between
local coupled cluster and MP2 calculations. The affected parameters and values for
MP2/coupled cluster are given below.

\begin{table}[H]\small \centering \begin{tabular}{|>{\ttfamily}c|>{\ttfamily}c|>{\ttfamily}c|l|}\hline
   & \multicolumn{1}{c|}{\texttt{LOOSE}}& \multicolumn{1}{c|}{\texttt{NORMAL}}& \multicolumn{1}{c|}{\texttt{TIGHT}}\\\hline
   \multicolumn{4}{|c|}{\texttt{Local Coupled Cluster}}  \\\hline
  \texttt{ccsdPairThreshold}       &  1e-3  & 1e-4     & 1e-5    \\\hline
  \texttt{pnoThreshold}            &  1e-6  & 3.33e-7  & 1e-7    \\\hline
  \texttt{orbitalToShellThreshold} &  1e-3  & 1e-3     & 1e-3    \\\hline
  \texttt{mullikenThreshold}       &  1e-3  & 1e-3     & 1e-4    \\\hline
  \texttt{doiPAOThreshold}         &  1e-2  & 1e-2     & 1e-3    \\\hline
  \multicolumn{4}{|c|}{\texttt{Local MP2}}  \\\hline
  \texttt{ccsdPairThreshold}       &  1e-3     &  1e-4 & 1e-5    \\\hline
  \texttt{pnoThreshold}            &  3.33e-7  &  1e-8 & 1e-9    \\\hline
  \texttt{orbitalToShellThreshold} &  1e-3     &  1e-3 & 1e-4    \\\hline
  \texttt{mullikenThreshold}       &  1e-4     &  1e-4 & 1e-5    \\\hline
  \texttt{doiPAOThreshold}         &  1e-2     &  1e-3 & 1e-3    \\\hline
\end{tabular}
\end{table}

%###############################%
%             Tasks             %
%###############################%

% in alphabetical order, insert new tasks in the correct location
% please use the format given in template.tex when adding a new task
\subsection{Task: Active-Space Selection\label{sec:activeSpaceTask}}
This tasks performs an occupied orbital selection for the calculation of relative
energies along a reaction path. The atoms of all supersystems have to be identical
and ordered in the same way. If active and environment systems are provided via the
input, the results of the selection are saved in these systems. The charge, atoms, basis
set and spin may be changed. If no environment and/or active systems are provided
via the input, systems are constructed and written to disk with the names
supersystem name + ``\_act''/``\_env'' for active and environment system, respectively.
For information regarding the methodology we refer to the original publications~\cite{Bensberg2019a,Bensberg2020}.

\subsubsection{Example Input}
\begin{lstlisting}
  +task AS
    super reactant  
    super transition
    super product
    similarityLocThreshold 1e-2
    similarityKinEnergyThreshold 1e-2
    +loc
      locType IBO
    -loc
  -task
 \end{lstlisting}
\subsubsection{Basic Keywords}
\begin{description}
 \item [\texttt{Name}]\hfill \\
 Aliases for this task are \ttt{ACTIVESPACESELECTIONTASK}, \ttt{ACTIVESPACETASK} and \ttt{AS}.  
 \item [\texttt{Supersystems}]\hfill \\
 Accepts multiple supersystems (\ttt{super}) which are used for the orbital comparison.   
 \item [\texttt{ActiveSystems}]\hfill \\
 Accepts multiple active systems to which the active orbitals are parsed. The first
 active system is assumed to correspond to the first super system, the second to the second
 supersystem etc. 
 \item [\texttt{EnvironmentSystems}]\hfill \\
 Accepts multiple environment systems to which the environment orbitals are parsed. The first
 environment system is assumed to correspond to the first supersystem. 
 \item [\texttt{sub-blocks}]\hfill \\
 Orbital localization task (\ttt{loc}) settings are added via sub-blocks in the task settings.
 \item [\texttt{similarityLocThreshold}]\hfill \\
 The threshold for the comparison of partial charges. The default is \ttt{0.05}.
 \item [\texttt{similarityKinEnergyThreshold}]\hfill \\
 The threshold for the comparison of the orbital kinetic energy. The default is \ttt{0.05}.
 \item [\texttt{populationAlgorithm}]\hfill \\
 The algorithm used to calculate partial charges. The default are shell-wise intrinsic atomic orbital (IAO)
 charges (\ttt{IAOShell}). Other useful options are Mulliken charges (\ttt{MULLIKEN}) and atom-wise IAO
 charges (\ttt{IAO}).
 \item [\texttt{alignPiOrbitals}]\hfill \\
 Align the orbitals of all supersystems to the orbitals of the first supersystem (after localization). This
 is useful to avoid degenerate localization minima especially for $\pi$-systems. Default is \ttt{false}.
\end{description}
\subsubsection{Advanced Keywords}
\begin{description}
    \item [\texttt{localizationThreshold}]\hfill \\
    The atom-wise Mulliken population threshold for the assignment of atoms to the subsystems. This is done
    after the supersystem orbital set has been separated into active and environment orbitals. If the total
    population of the active orbitals on a given atoms exceeds this threshold it is assigned to the active
    system. The default is \ttt{0.8} au.
    \item [\texttt{load}]\hfill \\
    If true, the SCF procedure is skipped and the tasks tries to load the supersystem results. The default
    is \ttt{false}.
    \item [\texttt{usePiBias}]\hfill \\
    Use a selection threshold scaling based on the number of significant, orbital-wise partial charges for
    each orbital. This is used to reduce the over-selection of $\pi$-orbitals. The default is \ttt{false}.
    \item [\texttt{biasThreshold}]\hfill \\
    The threshold for the evaluation of the number of significant partial charges for. The default is
    \ttt{0.01}.
    \item [\texttt{biasAverage}]\hfill \\
    The averaging-parameter for the threshold scaling used for \ttt{usePiBias true}.
    The default is \ttt{12.0}.
    \item [\texttt{skipLocalization}]\hfill \\
    If \ttt{true}, the localization step is skipped. This can be useful if the orbitals were already
    localized. The default is \ttt{false}.
 \end{description}

\subsection{Task: Basis-Set Truncation\label{task:truncation}}
Performs a truncation of the basis set of the given system. Only basis functions centered
on dummy atoms are truncated. All others are considered to be the core basis of the
system. Note that this task can manipulate the geometry of the given system such that 
dummy atoms without any basis functions will not survive. This effectively truncates any associated 
fitting basis.

{\color{red}IMPORTANT:} There will be no orbitals available for the system after executing this
task! Only the density matrix is available. This task should always be followed by an SCF-like step!
\subsubsection{Example Input}
\begin{lstlisting}
  +task BST
    act water
    truncAlgorithm NETPOPULATION
  -task
\end{lstlisting}
\subsubsection{Basic Keywords}
\begin{description}
  \item [\texttt{name}]\hfill \\
    Aliases for this task are \ttt{BASISSETTRUNCATIONTASK}, \ttt{BasisSetTask} and \ttt{BST}.
  \item [\texttt{activeSystems}]\hfill \\
    Will use the first active system and truncate the basis set based on the density of the given system.
  \item [\texttt{truncAlgorithm}]\hfill \\
    The algorithm used for basis set truncation. By default the Mulliken-net-population
    algorithm (\ttt{NETPOPULATION}) is taken. This algorithm eliminates basis functions from the active
    system where the sum of all squared occupied orbital coefficients is lower than \ttt{netThreshold}.
    Alternatively the primitive net-population algorithm (\ttt{PRIMITIVENETPOP}) can be requested. This
    algorithm reduces the number of basis functions by the factor \ttt{truncationFactor}.
  \item[\texttt{netThreshold}] \hfill \\
    Mulliken-net-population up to which an basis function centred on dummy atoms is not truncated, requires
    \ttt{truncAlgorithm}=\ttt{NETPOPULATION}. By default \ttt{1e-4}.
  \item[\texttt{truncationFactor}] \hfill \\
    The ratio of basis functions on dummy atoms used in addition ordered by importance in their MnP, requires
    \ttt{truncAlgorithm}=\ttt{PRIMITIVENETPOP}. By default \ttt{0.0}.
\end{description}
\subsection{Task: Broken--Symmetry\label{task:brokensymmetry}}
The Broken--Symmetry task performs broken--symmetry (BS) calculations using KS-DFT
or sDFT. During a BS calculation, the phenomenological parameter $J_\text{AB}$ is calculated
which describes the magnetic interaction between two spin sites. The corresponding 
Heisenberg-Dirac-van Vleck (HDvV) Hamiltonian is:
\begin{equation}
	\mathcal{H}_\text{HDvV} = -2J_\text{AB} \hat{S}_\text{A} \hat{S}_\text{B}.
\end{equation}
Here, $\hat{S}$ is the spin localized on spin site A or B. A positive value of $J_{AB}$ describes
a ferromagnetically coupled system while a negative value specifies an antiferromagnetic coupling.
The BS state in a regular BS (KS)-DFT calculation is constructed from the high spin (HS) 
state via the Localized Natural Orbitals (LNO) approach\cite{shoji2014}. In case of sDFT
calculations the BS state is constructed directly from the subsystems, see also Ref. \cite{massolle2020}.

Several formalisms for the calculation of $J_{AB}$ were proposed and are implemented. In the
following they will be called $J(1)_{AB}$\cite{ginsberg1980, noodleman1981, noodleman1986},
$J(2)_{AB}$\cite{bencini1986}, $J(3)_{AB}$\cite{yamaguchi1986, soda2000} and $J(4)_{AB}$\cite{noodleman1981}. The corresponding equations are:
\begin{align}
	J(1)_{AB} &= \frac{E_\text{BS}-E_\text{HS}}{S_\text{max}^2}\\
	J(2)_{AB} &= \frac{E_\text{BS}-E_\text{HS}}{S_\text{max} (S_\text{max} +1)}\\
	J(3)_{AB} &= \frac{E_\text{BS}-E_\text{HS}}{\langle \hat{S}^2\rangle_\text{HS} - \langle \hat{S}^2\rangle_\text{BS}}\\
	J(4)_{AB} &= \frac{E_\text{BS}-E_\text{HS}}{1 + S_\text{AB}^2},
\end{align}
with $E_\text{BS}$ and $E_\text{HS}$ as the energy of the BS or HS state, $S_\text{max}$
the spin of the HS state, $\langle \hat{S}^2\rangle$ the $S^2$ expectation value and $S_\text{AB}$ the overlap
of the magnetic orbitals. $J(1)_{AB}$ describes the weak and $J(2)_{AB}$ the
strong interaction limit. $J(3)_{AB}$ and $J(4)_{AB}$ describe the whole interaction range.

\subsubsection{Example BS-DFT Input}
\begin{lstlisting}
+task BS
  act TTTAdimer
  nA 1
  nB 1
-task
\end{lstlisting}

\subsubsection{Example BS-sDFT Input}
\begin{lstlisting}
+task BS
  act TTTAmon1
  act TTTAmon2
  embeddingScheme FDE
-task
\end{lstlisting}

\subsubsection{Basic Keywords}
\begin{description}
	\item [\texttt{name}]\hfill \\
	Aliases for this task are \ttt{BrokenSymmetryTask} and \ttt{BS}.
	\item [\texttt{activeSystems}]\hfill \\
	The active system is the high spin systems of the broken--symmetry calculation. It can
	be defined as one super system or two subsystems can be provided which resemble the two
	spin sites.
	\item [\texttt{environmentSystems}]\hfill \\
	If specified, the environment system is the broken--symmetry system which is loaded
	from disk. This can be used to try a different energy evaluation without performing a 
	new scf.
	\item[\texttt{sub-blocks}]\hfill \\
	The embedding (\ttt{emb}) and PCM (\ttt{pcm}) settings are added via sub-block 
	in the task settings. Prominent
	settings in the embedding block that are relevant for this task, and their defaults are
	:\ttt{naddXCFunc=PW91}, \ttt{embeddingMode=NADD\_FUNC}.
\subsubsection{BS-DFT Keywords}
	\item [\texttt{nA}]\hfill \\
	Number of unpaired electrons on the first spin site. By default (\ttt{1}).
	\item [\texttt{nB}]\hfill \\
	Number of unpaired electrons on the second spin site. By default (\ttt{1}).
\subsubsection{Advanced BS-DFT Keywords}
	\item [\texttt{noThreshold}]\hfill \\
	Threshold for the assignment of the NO orbitals as SONO, UONO and DONO. By default (\ttt{0.2}).
	\item [\texttt{locType}]\hfill \\
	The localization algorithm applied for localizing the SONOs. By default the
	Pipek--Mezey algorithm is taken (\ttt{PIPEK\_MEZEY}).
\subsubsection{BS-sDFT Keywords}
	\item [\texttt{embeddingScheme}]\hfill \\
	The embedding scheme used for the sDFT calculation. Possible options are: \\
    \ttt{NONE}: A standard BS-DFT calculation is performed where the two subsystems 
    represent one spin site.\\
	\ttt{ISOLATED}. The isolated (spin) densities (without any spin polarization!) of the 
	subsystems are used for an FDE like energy evaluation.\\ 
	\ttt{FDE}. A parallel FDE task is performed for the high spin and broken--symmetry
	system.\\ 
	\ttt{FAT} A FaT task is performed for the high spin and broken--symmetry
	system.\\
	By default: \ttt{NONE}.
	\item [\texttt{evalTsOrtho}]\hfill \\
	If enabled, the non-additive kinetic energy is evaluated from orthogonalized subsystem 
	orbitals. If \ttt{orthogonalizationScheme = NONE} the density matrix is corrected for
	the non-orthogonality of the MOs, use \ttt{evalAllOrtho} with \ttt{orthogonalizationScheme = NONE} 
	if $T_s^\text{nadd}=0$ should be calculated. By default: \ttt{False}.
	\item [\texttt{evalAllOrtho}]\hfill \\
	If enabled, all energy contributions are evaluated from orthogonalized subsystem orbitals.
	By default: \ttt{False}.
	\item [\texttt{orthogonalizationScheme}]\hfill \\ The orthogonalization scheme used
	for the construction of orthogonal supersystem orbitals if \ttt{evalTsOrtho} or \ttt{evalAllOrtho} are enabled.
	By default: \ttt{LOEWDIN}.
	\item [\texttt{maxCycles}]\hfill \\ The maximum number of FaT iterations.
	By default: \ttt{50}.
	\item [\texttt{convThresh}]\hfill \\ Convergence criterion for the absolute change
	of the density matrices w.r.t FaT cycles.
	By default: \ttt{1.0e-6}.
\end{description}

\subsection{Task: Coupled Cluster\label{sec:coupledClusterTask}}
This task performs coupled cluster calculations, possibly using canonical and local coupled cluster.
Additionally, this task also performs embedded coupled cluster calculations. Effects of an optional
PCM are taken only into account on the level of the orbitals/Fock matrix.

\subsubsection{Example Input}
\textbf{Canonical coupled cluster:}
\begin{lstlisting}
 +task CC
   system water
   level CCSD
 -task
\end{lstlisting}
\textbf{Local coupled cluster:}

{\color{red}IMPORTANT: } The orbital localization step for local coupled cluster is not carried out
within this task! Please make sure that the orbitals have been localized before using the local
correlation version of this task (see Sec.~\ref{task:localization}).
\begin{lstlisting}
 +task CC
   system water
   level DLPNO-CCSD
   +lc
     pno_settings tight
   -lc
 -task
\end{lstlisting}
\subsubsection{Basic Keywords}
\begin{description}
 \item [\texttt{name}]\hfill \\
   Aliases for this task are \ttt{CoupledClusterTask} and \ttt{CC}.
 \item [\texttt{activeSystems}]\hfill \\
   Accepts a single active system that will be used in the actual coupled cluster calculation.
 \item [\texttt{environmentSystems}]\hfill \\
   Accepts multiple environment systems that are used in embedded coupled cluster calculations.
 \item [\texttt{sub-blocks}]\hfill \\
   The embedding (\ttt{emb}), local correlation (\ttt{lc}) and PCM (\ttt{pcm}) settings are added via sub-blocks in the task settings.
   Prominent settings in the embedding block that are relevant for this task, and their defaults are:
   \ttt{naddXCFunc=BP86}, \ttt{embeddingMode=LEVELSHIFT}.
   Similarly, \ttt{pnoSettings=NORMAL} is a commonly changed default in the local correlation settings.
 \item [\texttt{level}]\hfill \\
   The type of coupled cluster calculation to perform. By default a (canonical) \ttt{CCSD} calculation is performed
   Alternatively canonical \ttt{CCSD(T)} and local \ttt{DLPNO-CCSD}, \ttt{DLPNO-CCSD(T0)} calculations can be
   requested with this keyword.
 \item [\texttt{maxCycles}]\hfill \\
   The maximum number of cycles allowed until convergence is expected by the coupled cluster iterations.
   By default \ttt{100} cycles are allowed.
 \item [\texttt{normThreshold}]\hfill \\
   The threshold at which the coupled cluster iterations are considered converged. By default \ttt{1.0e-5}.
\end{description}
\subsubsection{Advanced Keywords}
\begin{description}
  \item [\texttt{writePairEnergies}]\hfill \\
  Write the pair energies to a file with name: systemName\_pairEnergies\_CCSD.dat. This only works for
  DLPNO-CCSD/DLPNO-CCSD(T0). By default \ttt{false}.
\end{description}

\subsection{Task: DFT-embedded Local Correlation Calculations}
This tasks performs a DFT-embedded local correlation (DLPNO-coupled cluster or MP2) calculation.
This task only calls other tasks. It servers as an input helper. It calls the freeze-and-thaw/FDE task
(see Section~\ref{sec:FAT} and \ref{sec:FDE}) and local correlation task
(see Section~\ref{task:localCorrelation}).
\subsubsection{Example Input}
\begin{lstlisting}
 +task DFTEMB
   act waterA
   env waterB
   +emb
     naddXCFunc PBE
   -emb
   +lc
     method DLPNO-CCSD(T0)
   -lc
 -task
\end{lstlisting}
\subsubsection{Basic Keywords}
\begin{description}
  \item [\texttt{name}]\hfill \\
    Aliases for this task are \ttt{DFTEMBEDDEDLOCALCORRELATIONTASK}, \ttt{DFTEMBEDDING}, \ttt{DFTEMB},  and \ttt{DFTEMBLC}.
  \item [\texttt{activeSystems}]\hfill \\
    Accepts a single active system that is used for the local correlation calculation.
  \item [\texttt{environmentSystems}]\hfill \\
    Accepts a list of environment systems.
  \item [\texttt{supersystems}]\hfill \\
    Accepts a single supersystem which is used as a supersystem during the calculation.
    If none is given, a supersystem is constructed on-the-fly. The supersystem geometry and
    electronic structure will be overwritten during the calculation.
 \item [\texttt{sub-blocks}]\hfill \\
  The Embedding (\ttt{emb}) settings, local correlation (\ttt{lc}) settings,
  orbital localization task (\ttt{loc}) settings, system splitting task settings (\ttt{split})
  system addition task settings (\ttt{add}) and basis set truncation task settings (\ttt{trunc})
  are added via sub-blocks in the task settings.
  By default \ttt{splitValenceAndCore = true} is set for the localization task settings,
  \ttt{truncAlgorithm = NONE} for the basis set truncation task settings,
  \ttt{addOccupiedOrbitals = false} for the system addition task settings,
  \ttt{enforeceHFFockian = true} for the local correlation settings,
  and \ttt{embeddingMode = FERMI} for the embedding settings.
 \item [\texttt{runFaT}]\hfill \\
 If true, all subsystems are relaxed in a freeze-and-thaw procedure. By default \ttt{false}.
 \item [\texttt{fromSupersystem}]\hfill \\
 If true, the supersystem orbitals are calculated and partitioned into subsystems. By default \ttt{true}.
\end{description}

\subsubsection{Advanced Keywords}
\begin{description}
  \item [\texttt{maxCycles}]\hfill \\
  The maximum number of cycles allowed until convergence is expected by the coupled cluster iterations.
  By default \ttt{100} cycles are allowed.
  \item [\texttt{normThreshold}]\hfill \\
  The threshold at which the coupled cluster/MP2 iterations are considered converged. By default \ttt{1.0e-5}.
  \item [\texttt{writePairEnergies}]\hfill \\
  Write the pair energies to a file with name: systemName\_pairEnergies\_CCSD.dat or systemName\_pairEnergies\_MP2.dat
\end{description}

\subsection{Task: Dispersion Correction}
This task evaluates different types of dispersion corrections for the given system.
\subsubsection{Example Input}
\begin{lstlisting}
 +task Dispersion
   system water
   dispType D3BJ
   functional PBE0
   gradient True
 -task
\end{lstlisting}
\subsubsection{Basic Keywords}
\begin{description}
  \item [\texttt{name}]\hfill \\
    Aliases for this task are \ttt{DispersionCorrectionTask}, \ttt{Dispersion},  and \ttt{Disp}.
  \item [\texttt{activeSystems}]\hfill \\
    Accepts a single active system and calculates its dispersion correction.
  \item [\texttt{dispType}]\hfill \\
    The type of dispersion correction to calculate. By default \ttt{D3BJ} is chosen.
    For a list of valid options see the \ttt{dispersion} keyword in Section~\ref{sec:system:dft}.
  \item [\texttt{functional}]\hfill \\
    Any exchange correlation functional for which parameters are available, by default the system functional is used (\ttt{NONE}). A functional given in this task has priority over the system functional.
  \item [\texttt{gradient}]\hfill \\
    Boolean switch, if true, will additionally calculate the cartesian nuclear gradient corrections.
  \item [\texttt{hessian}]\hfill \\
    Boolean switch, if true, will additionally calculate the nuclear Hessian corrections.
\end{description}

\subsection{Task: DOS-Coupled Cluster}
This task combines the localization task, the generalized direct orbital selection (GDOS)
task, and the wavefunction embedding task. It allows the user to calculate relative energies
with a multi-level DLPNO-CC approach using the orbital sets generated by the GDOS.
\subsubsection{Example Input}
\begin{lstlisting}
+task DOSCC
  act reactant
  act product
  dosSettings normal
  +lc0
    method DLPNO-CCSD(T0)
    pnoSettings normal
    useFrozenCore true
  -lc0
  +lc1
    method DLPNO-CCSD
    pnoSettings loose
    useFrozenCore true
  -lc1
-task
\end{lstlisting}
\subsubsection{Basic Keywords}
\begin{description}
\item [\texttt{name}]\hfill \\
Aliases for this task are \ttt{DOSCC} and \ttt{DOSCCTask}.
\item [\texttt{activeSystems}]\hfill \\
Accepts a set of active systems as supersystems. Note that at least two systems need to be supplied.
All systems have to fulfill the criteria for the direct orbital selection (see task~\ref{sec:activeSpaceTask}).
\item [\texttt{sub-blocks}]\hfill \\
Local correlation (\ttt{lc}), localization task settings (\ttt{loc}). and non-blocked settings
of the wavefunction embedding task (\ttt{wfemb}) are added via sub-blocks in the task settings.
The local correlation settings are added as a list of settings. For each fragment (environment) system there is
a local correlation settings object that determines the settings employed for the orbitals of this system. The
task assumes that systems/settings with a lower index are tighter.
By default \ttt{splitValenceAndCore = true} is set for the localization task settings.
\item [\texttt{alignOrbitals}]\hfill \\
If true, the orbitals of each system are aligned with respect to the first system orbitals. By default \ttt{true}.
\item [\texttt{dosSettings}]\hfill \\
Macro flag to select multiple DOS thresholds (see task~\ref{task:gdos}, \ttt{similarityLocThreshold}
and \ttt{similarityKinEnergyThreshold}). The flag \ttt{LOOSE} corresponds to the threshold triple \ttt{\{1e-1, 1e-2, 1e-3\}},
\ttt{NORMAL} to \ttt{\{5e-2, 5e-3, 1e-3\}}, \ttt{TIGHT} to  \ttt{\{2e-2, 2e-3, 8e-4\}}, \ttt{VERYTIGHT} to \ttt{\{8e-3, 1e-3, 1e-4\}},
and \ttt{EXTREME} to \ttt{\{5e-3, 5e-4, 1e-4\}}.
By default \ttt{NORMAL}. The thresholds set by the flag may be overridden manually.
\item [\texttt{printGroupAnalysis}]\hfill \\
If true, print information on orbitals pairs. By default \ttt{true}.
\item[\texttt{pairCutoff}] \hfill \\
Vector of cutoffs to determine what pair belongs to which settings based on energy differences. The default is \ttt{\{1e-4, 0.0\}}. The vector is assumed to have the tighter settings at smaller indices -  corresponding to higher pair cutoffs.
\item[\texttt{normThreshold}] \hfill \\
Maximum entry left in the residual matrix / vector in order to assume that the amplitude optimization in a DLPNO-CCSD calculation is converged. \ttt{1e-5} by default. See section~\ref{sec:coupledClusterTask} for further information.
\item[\texttt{maxCycles}] \hfill \\
Maximum number of iterations allowed in the amplitude optimization in a DLPNO-CCSD calculation. \ttt{100} by default. See section~\ref{sec:coupledClusterTask} for further information.
\item[\texttt{strictPairEnergyThreshold}] \hfill \\
If true, choose the tightest pair threshold given for the CCSD pair truncation. \ttt{True} by default.
\item[\texttt{strictTriples}] \hfill \\
If true, the triples construction of the orbital pair will be determined based on the tightest settings of the group rather than the orbital pair's settings. \ttt{True} by default.
\item[\texttt{skipCrudePresPairSelected}] \hfill \\
If true, the crude SC-MP2 prescreening step in a DLPNO-CCSD calculation will be skipped. \ttt{False} by default.  
 
\end{description}

\subsection{Task: Energy-Decomposition Analysis}
This task performs an energy-decomposition analysis as proposed by Morokuma\cite{Morokuma1971}.
\subsubsection{Example Input}
\begin{lstlisting}
 +task EDA
   system water_monomer_a
   system water_monomer_b
 -task
\end{lstlisting}
\subsubsection{Basic Keywords}
\begin{description}
  \item [\texttt{name}]\hfill \\
    Aliases for this task are \ttt{EDATask},  and \ttt{EDA}.
  \item [\texttt{activeSystems}]\hfill \\
    Requires two systems. Will use the two systems and calculate the Morokuma EDA of the supersystem 
    generated by these two systems.
\end{description}

\subsection{Task: Electronic-Structure Copy}
This task copies the first active system to all environment systems by rotating the
orbitals to their internal frames. All systems should have the same atoms ordered
in the same way.
\subsubsection{Example Input}
\begin{lstlisting}
 +task ESC
  act waterAct
  env waterEnv1
  env waterEnv2
  env waterEnv3
 -task
\end{lstlisting}

\subsubsection{Basic Keywords}
\begin{description}
\item [\texttt{name}]\hfill \\
  Aliases for this task are \ttt{ELECTRONICSTRUCTURECOPYTASK}, \ttt{COPY}, and \ttt{ESC}.
\item [\texttt{activeSystems}]\hfill \\
  The source system to be used.
\item [\texttt{environmentSystems}]\hfill \\
  The target systems to which the orbitals are to be copied.
\item [\texttt{atomFrameIndices}]\hfill \\
  The atom indices used for the internal frame construction.
  If none are given, the first three atoms are used.
\item [\texttt{orthogonalize}]\hfill \\
  Orthogonalize the orbitals after copying using a
  Löwdin orthogonalization. \ttt{false} by default.
\item [\texttt{copyCharges}]\hfill \\
  Change the charges/spins of the environment systems
  to the charge/spin of the active system. \ttt{true} by default.
\end{description}

\subsection{Task: Electron Transfer}

This task calculates adiabatic wave functions from the linear combination of any number of quasi-diabatic states.
Those wave functions are used to calculate several charge-transfer properties. Within a two-state approach, the
analytical electronic coupling and  excitation energy are calculated. For multiple quasi-diabatic states, the Hamilton
matrix in the quasi-diabatic basis as well as the energy levels of the adiabatic wave functions are calculated.
Additionally, spin-density distributions can be calculated and printed to cube files. Moreover, atom-wise spin
populations via a Becke-Population analysis can be obtained. Basically, this task combines the FDE-ET, FDE-diab,
and ALMO multi-state DFT formalisms. This task is only implemented for the \ttt{UNRESTRICTED} case, therefore,
prior FaT calculations must also be carried out in spin-unrestricted mode.
\subsubsection{Example Input}
\textbf{FDE-ET:}
\begin{lstlisting}
+task fdeet
  #quasi-diabatic state 1
  act sys1-state1
  act sys2-state1
  #quasi-diabatic state 2
  act sys1-state2
  act sys2-state2
  states {2 2}
  couple {1 2}
  spindensity true
-task
\end{lstlisting}
\textbf{ALMO multi-state DFT:}
{\color{red}IMPORTANT:} The states were calculated with the \ttt{FreezeAndThawTask} and \ttt{embeddingMode ALMO}. If the 
resulting subsystems are added to one system with the \ttt{SystemAdditionTask}, the input simplifies to:
\begin{lstlisting}
+task et
  act state1
  act state2
-task
\end{lstlisting}

\subsubsection{Basic Keywords}
\begin{description}
  \item [\texttt{name}]\hfill \\
  Aliases for this task are \ttt{ELECTRONTRANSFERTASK}, \ttt{ET} and \ttt{FDEET}.
  \item [\texttt{activeSystems}]\hfill \\
  The systems that construct the quasi-diabatic states (they are usually obtained from separate FaT calculations).
  \item [\texttt{states}]\hfill \\
  This vector input defines the quasi-diabatic states. Herein, the number of elements in the vector defines how many
  quasi-diabatic states are used, while each index specifies the number of subsystems contained in this state. All states must have the same number 
  of systems. Here, the ordering of the \ttt{activeSystems} is important! As an example, \ttt{states\{2 2\}} specifies a linear combination of 2 quasi-diabatic states, where the first state is composed of the first two \ttt{activeSystems} and the second state is composed of the next two \ttt{activeSystems}.
  By default, if no vector is given, each system specifies one quasi-diabatic state.
  \item [\texttt{couple}]\hfill \\
  This vector input specifies which of the specified \ttt{states} shall be used in the linear combination.
  Therefore, \ttt{couple\{1 2\}} can be understood as using states 1 and 2. Additionally, \ttt{couple\{1 2 ; 1 3\}}
  can be understood as two separate FDE-ET runs for which in the first run states 1 and 2 are used, whereas in the
  second run states 1 and 3 are used. Using 3 or 4 quasi-diabatic states can be accomplished by specifying
  \ttt{couple\{1 2 3\}} or \ttt{couple\{1 2 3 4\}}, respectively.
  By default, if no vector is given, all states are coupled.
  \item [\texttt{spindensity}]\hfill \\
  Specifies if the spin density of the adiabatic states shall be calculated and printed to cube file. By default \ttt{false}.
  \item [\texttt{spinpopulation}]\hfill \\
  Specifies if atom-wise spin populations are calculated. If \ttt{true}, the spin-density is calculated and subsequently
  used in a Becke population analysis. See also the keyword \ttt{population}. By default \ttt{false}.
  \item [\texttt{population}]\hfill \\
  This vector input specifies which adiabatic states are used for the population analysis if \ttt{spinpopulation true}. By default \ttt{\{0\}}. This corresponds to the calculation of the spin populations for the first adiabatic state.
\end{description}
\subsubsection{Advanced Keywords}
\begin{description}
  \item [\texttt{disjoint}]\hfill \\
  This vector input specifies the usage of the disjoint approximation for FDE-ET/Diab. Therefore, \ttt{disjoint \{1 2\}} corresponds to joining
  the systems 1 and 2 in each state while all other systems are not coupled. By default \ttt{\{\}}.
  \item [\texttt{printContributions}]\hfill \\
  Specifies whether the real-space representations of the diabatic transition-density matrices shall be printed to cube
  files or not. By default \ttt{false}.
  \item [\texttt{diskMode}]\hfill \\
  Specifies whether the transition-density matrices are written to HDF5 file or kept in memory. This is recommended for FDE-ET/diab calculations in which the available memory is insufficient.
  This is often the case when many quasi-diabatic states are coupled of which each state holds a very large transition-density matrix. By default \ttt{false}.
  \item [\texttt{useHFCoupling}]\hfill \\
  Specifies whether the off-diagonal elements of the Hamilton matrix are calculated alternatively using HF exchange contributions. By default \ttt{false}.
  \item [\texttt{coupleAdiabaticStates}]\hfill \\
  Specifies whether the adiabatic states of different runs (see keyword \ttt{couple}) should be coupled again. By default \ttt{false}.
  \item [\texttt{configurationWeights}]\hfill \\
  Specifies weights of the adiabatic states that are coupled. Must have the same dimensions as \ttt{couple}. Requires \ttt{coupleAdiabaticStates}.

\end{description}

\subsection{Task: Energy Evaluation}\label{task: energy eval}
This task performs an energy evaluation for the given systems with the settings
and density currently assigned to the system. It is possible to re-evaluate the density
of systems with a different (nadd) XC functional and to use orthogonalized subsystem 
orbitals for the evaluation of $T_s^\text{nadd}$ or all energy contributions.
\subsubsection{Example Input}
\begin{lstlisting}
  +task ENERGY
    act water
    +lc
      PNOSETTINGS TIGHT
    -lc
  -task
\end{lstlisting}

\subsubsection{Basic Keywords}
\begin{description}
  \item [\texttt{name}]\hfill \\
  Aliases for this task are \ttt{EVALUATEENERGYTASK}, \ttt{ENERGYTASK}, \ttt{ENERGY}, and \ttt{E}.
  \item [\texttt{activeSystems}]\hfill \\
   Evaluates the energy for the active systems in a KS-DFT (one active system)
   or FDE (more than one active system) manner, using its density and settings.
   \item [\texttt{Supersystems}]\hfill \\
   If \ttt{evalTsOrtho} or \ttt{evalAllOrtho} is used the supersystem with the orthogonalized orbitals can be 
   stored here.
   \item [\texttt{sub-blocks}]\hfill \\
   The local correlation (\ttt{lc}) and embedding (\ttt{emb}) settings are added via sub-blocks in the task settings.
   Prominent settings in the local correlation block that are relevant for this task, and their defaults are:
    \ttt{PNOSETTINGS=TIGHT} and \ttt{method=DLPNO\_MP2}.
    Prominent settings in the embedding block that are relevant for this task, and their defaults are: \ttt{embeddingMode=NADD\_FUNC}
    and \ttt{naddXCFunc=PW91}.
    \item [\texttt{mp2Type}]\hfill \\
    MP2-type used for the evaluation of the correlation energy of double-hybrid functionals, see also \ref{task: mp2}. By default \ttt{DF}.
    \item [\texttt{maxResidual}]\hfill \\
    Maximum residual threshold for local MP2. By default \ttt{1e-5}.
    \item [\texttt{maxCycles}]\hfill \\
    Maximum number of iterations before canceling the amplitude optimization in local MP2. By default \ttt{100}.
    \item [\texttt{useDifferentXCFunc}]\hfill \\
    If \ttt{true} the XC-functional specified in \ttt{XCfunctional} (or the one in the \ttt{+customfunc} block) is used for the energy evaluation instead of the
    one specified in the system block. By default \ttt{false}.
    \item [\texttt{XCfunctional}]\hfill \\
    The XC-functional used for the evaluation of the electron density, if \ttt{useDifferentXCFunc=true}.
    By default \ttt{BP86}. This can also be customized by invoking a \ttt{+customfunc} block (see Sec.~\ref{sec:system:customfunc}). The custom functional, if specified, has higher priority than the \ttt{XCfunctional}, but it also requires \ttt{useDifferentXCFunc=true}.
	\item [\texttt{evalTsOrtho}]\hfill \\
    If enabled, the non-additive kinetic energy is evaluated from orthogonalized subsystem 
    orbitals. If \ttt{orthogonalizationScheme=NONE} the density matrix is corrected for
    the non-orthogonality of the MOs, use \ttt{evalAllOrtho} with \ttt{orthogonalizationScheme=NONE} 
    if $T_s^\text{nadd}=0$ should be calculated. By default \ttt{false}.
    \item [\texttt{evalAllOrtho}]\hfill \\
    If enabled, all energy contributions are evaluated from orthogonalized subsystem orbitals.
    By default \ttt{false}.
    \item [\texttt{orthogonalizationScheme}]\hfill \\ The orthogonalization scheme used
    for the construction of orthogonal supersystem orbitals if \ttt{evalTsOrtho} or \ttt{evalAllOrtho} are enabled.
    By default \ttt{LOEWDIN}.
\end{description}

\subsection{Task: Export Cavity}\label{sec:tasks:ExportCavityTask}
This task creates and exports a molecular surface of a given system with all necessary quantities in .h5 files.
This will create a CavityData.h5  file and in case \ttt{cavityFormation} in the PCM block is set to \ttt{true}, a VDWCavityData.h5 file will also be created.
In order to save the necessary charges of the solvent models, the keyword \ttt{saveCharges} in the PCM block must be switched to true. This will generate the file PCMChargesData.h5.
\subsubsection{Example Input}
\begin{lstlisting}
 +system
  name water
  geometry h2o.xyz
  method dft

  +pcm
   use true
   solvent WATER
   solverType CPCM
   saveCharges true
  -pcm

 -system

 +task SCF
  act water
 -task

 +task EXPORTCAVITYTASK
  act water
  fdecavity false
 -task
\end{lstlisting}

\subsubsection{Basic Keywords}
\begin{description}
\item [\texttt{name}]\hfill \\
  Aliases for this task are \ttt{EXPORTCAVITYTASK} and \ttt{EXPORTCAVITY}.
\item [\texttt{activeSystems}]\hfill \\
  The system whose molecular surface is to be exported.
\item [\texttt{fdecavity}]\hfill \\
  Manual switch to toggle between exporting a supersystem cavity. \ttt{false} by default.
\end{description}

\subsection{Task: Export Grid}\label{sec:tasks:ExportGridTask}
This task creates and exports a grid of a given system with weights in text form.
\subsubsection{Example Input}
\begin{lstlisting}
 +task EXPORTGRIDTASK
  act water
 -task
\end{lstlisting}

\subsubsection{Basic Keywords}
\begin{description}
\item [\texttt{name}]\hfill \\
  Aliases for this task are \ttt{EXPORTGRIDTASK}, \ttt{GRIDTASK}, and \ttt{GRID}.
\item [\texttt{activeSystems}]\hfill \\
  The system whose grid is to be exported.
\item [\texttt{withAtomInfo}]\hfill \\
  Switch to print an additional column referencing the
atom each point belongs to (Beck/Voronoi Cell). \ttt{false} by default.
\end{description}

\subsection{Task: FCI Dump File Writer}\label{sec:tasks:FCIDumpFileWriter}
This tasks writes Molpro-style FCI dump file to disk. The integrals are calculated using the RI approximation and local
integral fitting.

An FCI dump file provides all two electron and one electron integrals in MO basis.
The orbital index counting starts at 1. The third and fourth index for the one
particle integrals are zero. The file start with (exactly) 4 lines of header, providing
the number of orbitals, electrons, the number of excess alpha electron (MS2), orbital
symmetry, and system symmetry. After the header the integrals are encoded through their
value and their orbital indices.

A typical FCI dump file may look like this:
\begin{lstlisting}
&FCI NORB= 8, NELEC= 10, MS2= 0,
ORBSYM= 1, 1, 1, 1, 1, 1, 1, 1,
ISYM= 1
&END
  8.2371169364e-01     1     1     1     1
  6.5011589197e-02     2     1     1     1
  6.5841092835e-01     2     2     1     1
  1.0997174278e-01     2     1     2     1
 -6.9437938103e-02     2     1     2     2
 ...
  7.7071181164e-01     8     6     0     0
 -6.6624329107e-16     8     7     0     0
 -6.2473727922e+00     8     8     0     0
\end{lstlisting}

\subsubsection{Example Input}
\begin{lstlisting}
 +task FCIDUMP
  act waterA
  env waterB
  onlyValenceOrbitals true
  +emb
      naddXcFunc pbe
      embeddingMode fermi
  -emb
 -task
\end{lstlisting}

\subsubsection{Basic Keywords}
\begin{description}
    \item [\texttt{name}]\hfill \\
    Aliases for this task are \ttt{FCIDUMP} and \ttt{FCIDUMPFILEWRITERTASK}.
    \item [\texttt{activeSystems}]\hfill \\
    Accepts a single active system for which the FCI dump file will be written.
    \item [\texttt{environmentSystems}]\hfill \\
    Accepts multiple environment systems. Their contribution will be included in the one electron integrals.
    \item [\texttt{outputFilePath}]\hfill \\
    The FCI dump output file path.
    \item [\texttt{orbitalRangeAlpha}]\hfill \\
    The orbital range for the alpha orbitals to be written to the FCI dump file.
    This is also the orbitals range, in the case restricted orbitals are used.
    \item [\texttt{orbitalRangeBeta}]\hfill \\
    The orbital range for the beta orbitals to be written to the FCI dump file.
    \item [\texttt{onlyValenceOrbitals}]\hfill \\
    If true, and no orbital ranges are provided, the orbital ranges are assumed to be the range of valence orbitals.
    \item [\texttt{calculateCoreEnergy}]\hfill \\
    If true, the energy of the core electrons is calculated and printed to the (FCI) output file.
    \item[\texttt{sub-blocks}]\hfill \\
    Possible sub-blocks are the embedding (\ttt{emb} with \ttt{naddXCFunc=BP86}).

\end{description}
\subsubsection{Advanced Keywords}
\begin{description}
    \item [\texttt{mullikenThreshold}]\hfill \\
    Prescreening threshold for the auxiliary function to orbital mapping. By default, \ttt{1.0e-4}.
    \item [\texttt{orbitalToShellThreshold}]\hfill \\
    Prescreening threshold for coefficient values. By default \ttt{1.0e-3}.
    \item [\texttt{valenceOrbitalsFromEnergyCutOff}]\hfill \\
    If true, valence orbitals are determined by an energy cut off. If false, tabulated values are used. By default, \ttt{false}.
    \item[\texttt{energyCutOff}]\hfill \\
    Energy cut off to determine core orbitals. By default, \ttt{-5.0}.
    \item[\texttt{virtualEnergyCutOff}]\hfill \\
    Energy cut off to determine virtual valence orbitals. By default, \ttt{+1.0}]
    \item[\texttt{integralSizeCutOff}]\hfill \\
    Integrals smaller than this value are not written to the output file. By default, \ttt{1e-9}.
    \item[\texttt{doiNetThreshold}]\hfill \\
    Prescreening threshold for the orbital coefficients during calculation of differential overlap integrals. By default, 1e-7.
    \item[\texttt{doiIntegralPrescreening}]\hfill \\
    All orbital combinations ik/jl in the integral (ik|jl) are ignored during the integral transformation that have a
    differential overlap smaller than this threshold. By default, \ttt{1e-7}.
\end{description}
\subsection{Task: Frozen-Density Embedding}
\label{sec:FDE}
This task performs frozen-density embedding (FDE) calculations, possibly using different embedding modes or local correlation settings.
\subsubsection{Example Input}
\begin{lstlisting}
+task FDE
  act acetone
  env water1
  env water2
  +emb
    embeddingMode NADDFUNC
    naddXCFunc PW91
    naddKinFunc PW91k
  -emb
-task
\end{lstlisting}
\subsubsection{Basic Keywords}
\begin{description}
	\item [\texttt{name}]\hfill \\
	Aliases for this task are \ttt{FDE} and \ttt{FDETask}.
	\item [\texttt{activeSystems}]\hfill \\
	Accepts a single active system that will be used in the actual FDE calculation.
	\item [\texttt{environmentSystems}]\hfill \\
	Accepts mutiple environment systems that are used in embedded calculation.
	\item [\texttt{sub-blocks}]\hfill \\
	The embedding (\ttt{emb}), local correlation (\ttt{lc}), localization(\ttt{loc}) and PCM (\ttt{pcm}) settings are added via sub-blocks in the task settings.
	Prominent settings in the embedding block that are relevant for this task, and their defaults are:
	\ttt{naddXCFunc=PW91}, \ttt{embeddingMode=NADD\_FUNC}, \ttt{PNOSETTINGS=TIGHT}, and \ttt{method=DLPNO\_MP2}.
	\item [\texttt{gridCutOff}] \hfill \\
    A distance cutoff for the integration grid used to calculate the non-additive  energy functional potentials. Negative values equal no cut off. By default \ttt{-1.0}
	\item[\texttt{mp2Type}]\hfill \\
	The MP2-type used for the evaluation of the correlation energy of double-hybrid functionals. By default \ttt{LOCAL} to use local MP2. Other options are \ttt{AO} to evaluate the full two-electron four-center integrals and \ttt{DF} which uses the density fitting approach specified with \ttt{densfitCorr} in the system block.
	\item [\texttt{maxResidual}] \hfill \\
	Convergence threshold for the local MP2 calculation. By default \ttt{1e-5}.
	\item [\texttt{maxCycles}] \hfill \\
	Maximum number of iterations before canceling the amplitude optimization in local MP2. By default \ttt{100}.
	\item [\texttt{calculateMP2Energy}] \hfill \\
	Calculate the MP2 contribution of double-hybrid functionals. By default \ttt{true}.
	\item [\texttt{calculateEnvironmentEnergy}] \hfill \\
	Calculate the energy of the environment systems. By default \ttt{false}.
	\item [\texttt{calculateSolvationEnergy}] \hfill \\
	Calculate the interaction of the active system with the joined environment and the energy of the active system. The environment is referenced
	as a continuum. By default \ttt{false}.
    \item[\texttt{calculateUnrelaxedMP2Density}]\hfill \\
    If true, the unrelaxed MP2 density of the active system is calulated. By default \ttt{false}.
	\item [\texttt{skipSCF}] \hfill \\
	Skip the SCF procedure and perform an energy evaluation only. By default \ttt{false}.
\end{description}

\subsubsection{Advanced Keywords}
\begin{description}
	\item [\texttt{smallSupersystemGrid}]\hfill \\
	If true will use the \ttt{smallGridAccuracy} of the given active system instead of the normal grid accuracy for the supersystem grid. By default \ttt{false}.
	\item [\texttt{finalGrid}]\hfill \\
	If true, non-additive xc and kin energies will only be evaluated once on a supersystem grid, at the end of the orbital optimization. By default \ttt{true}.
\end{description}

\subsection{Task: Finite Field}\label{task: ff}
\label{sec:ff}
This task performs Finite-Field (FF) calculations for a given active system, probably using embedding settings. 
Currently, it is possible to calculate (hyper) polarizabilites. Those properties can be calculated
either numerically, analytically or semi-numerically depending on the provided \ttt{frequency} keyword.
In general it is possible to calculate static and dynamic polarizabilities either numerically \ttt{frequency 0.0} or analytically \ttt{frequency 1e-9}. In the former case 12 SCF calculations will be performed varying the applied finite field.
In the latter case the \ttt{LRSCFTask} is called using the \ttt{frequency} which is numerically zero.
Static hyperpolarizabilities may be calculated numerically by choosing \ttt{frequency 0.0} and \ttt{hyperpolarizability true}.
The semi-numerical calculation of static hyperpolarizabilities may be triggered by choosing \ttt{frequency 1e-9} and \ttt{hyperpolarizability true}. Additionally, it is possible to obtain hyperpolarizabilities from the numerical differentiation
of dynamic polarizabilities with respect to a static electric field. This is possible by specifying any \ttt{frequency} and \ttt{hyperpolarizability true}.
\subsubsection{Example Input}
Isolated Input:
\begin{lstlisting}
+task ff
  act acetone
  finiteFieldStrength 0.1
  frequency 1.0
  hyperpolarizability true
-task
\end{lstlisting}
Embedded Input:
\begin{lstlisting}
+task ff
  act acetone
  env water
  finiteFieldStrength 0.1
  frequency 1.0
  hyperpolarizability true
-task
\end{lstlisting}
\subsubsection{Basic Keywords}
\begin{description}
	\item [\texttt{name}]\hfill \\
	Aliases for this task are \ttt{FF} and \ttt{FINITEFIELD}.
	\item [\texttt{activeSystems}]\hfill \\
	Accepts a single active system for which the desired properties are calculated.
	\item [\texttt{environmentSystems}]\hfill \\
	Accepts mutiple environment systems that are used in embedded calculations.
	\item [\texttt{sub-blocks}]\hfill \\
	The embedding (\ttt{emb}) settings are added via sub-blocks in the task settings.
	Prominent settings in the embedding block that are relevant for this task, and their defaults are:
	\ttt{naddXCFunc=NONE}, \ttt{naddKinFunc=NONE}, \ttt{embeddingMode=NONE}.
    \item [\texttt{finiteFieldStrength}]\hfill \\
    The step-with used for numerical differentiations. By default \ttt{1.0e-2}.
    \item [\texttt{frequency}]\hfill \\
    Specifies the frequency for that analytical dynamic polarizabilities are calculated. Analytical static polarizabilities
    can be calculated by specifying a frequency which is numerically zero (\emph{e.g.} \ttt{1e-9}). Numerical static polarizabilities
    are calculated by choosing \ttt{frequency 0.0}. By default \ttt{0}.
    \item [\texttt{hyperPolarizability}]\hfill \\
    A boolean that decides to calculate the hyperpolarizability by numerical differentiation, in the case of numerical static polarizabilities. In the case of analytical dynamic polarizabilities, those will be obtained by a semi-numerical approach. By default \ttt{false}.
\end{description}


\subsection{Task: Freeze-and-Thaw}
\label{sec:FAT}
This task performs freeze-and-thaw (FaT) calculations, for a list of given systems. Basically, this task
performs subsequent frozen-density embedding calculations for the given \ttt{activeSystems}, where in each
cycle the role of the active system changes. \ttt{environmentSystems} will never be set as active.
\subsubsection{Example Input}
\begin{lstlisting}
+task FAT
  act acetone1
  act acetone2
  env water1
  env water2
  +emb
    embeddingMode NADDFUNC
    naddXCFunc PW91
    naddKinFunc PW91k
  -emb
  maxcycles 6
-task
\end{lstlisting}
\subsubsection{Basic Keywords}
\begin{description}
	\item [\texttt{name}]\hfill \\
	Aliases for this task are \ttt{FAT} and \ttt{FaTTask}.
	\item [\texttt{activeSystems}]\hfill \\
	Accepts multiple active systems.
	\item [\texttt{environmentSystems}]\hfill \\
	Accepts multiple environment systems.
	\item [\texttt{sub-blocks}]\hfill \\
	The embedding (\ttt{emb}) and PCM (\ttt{pcm}) settings are added via sub-block in the task settings.
	Prominent settings in the embedding block that are relevant for this task, and their defaults are:
	\ttt{naddXCFunc=BP86}, \ttt{embeddingMode=LEVELSHIFT}.
	\item [\texttt{maxCycles}]\hfill \\
	The maximum number of FaT iterations. By default \ttt{50}.
	\item [\texttt{gridCutOff}]\hfill \\
	A distance cutoff for the integration grid used to calculate the non-additive  energy functional potentials. Negative values equal no cut off. By default \ttt{-1.0}.
	\item [\texttt{convThresh}]\hfill \\
	Convergence criterion for the absolute change of the density matrices w.r.t. FaT cycles. By default \ttt{1.0e-6}
	\item[\texttt{mp2Type}]\hfill \\
	The MP2-type used for the evaluation of the correlation energy of double-hybrid functionals. By default \ttt{DF} is chosen, which uses the density fitting approach specified with \ttt{densfitCorr} in the system block. Other options are \ttt{AO} to evaluate the full two-electron four-center integrals and \ttt{local} to use local MP2.
	\item [\texttt{calculateSolvationEnergy}]\hfill \\
	Calculate the interaction of the first active system with the joined environment and the energy of the first active system. The environment is referenced as a continuum. By default \ttt{false}.
	\item [\texttt{finalEnergyEvaluation}]\hfill \\
	Perform a final energy evaluation on the supersystem grid after converging the freeze-and-thaw. By default \ttt{true}.
  \item[\texttt{calculateUnrelaxedMP2Density}]\hfill \\
  This vector input specifies if unrelaxed MP2 densities are calculated for the active systems. Therefore, \ttt{calculateUnrelaxedMP2Density\{true false\}} will only calculate the unrelaxed MP2 density for the first active system. By default \ttt{\{\}}.
\end{description}
\subsubsection{Advanced Keywords}
\begin{description}
	\item [\texttt{smallSupersystemGrid}]\hfill \\
	If true will use the smallGridAccuracy of the given active system instead of the normal grid accuracy for the supersystem grid. By default \ttt{false}.
	\item [\texttt{extendBasis}]\hfill \\
	Extend the subsystem basis with basis functions centered in the other subsystems. By default \ttt{false}.
	\item [\texttt{basisExtThresh}]\hfill \\
	Overlap threshold for the extension of the subsystem basis. Needs \ttt{extendBasi = true}. By default \ttt{5.0e-2}.
	\item [\texttt{useConvAcceleration}]\hfill \\
	Turn the convergence acceleration (DIIS/Damping) on. By default \ttt{false}.
	\item [\texttt{diisStart}]\hfill \\
	Density RMSD threshold for the start of the DIIS. Needs \ttt{useConvAcceleration = true}. By default \ttt{5.0e-5}.
	\item [\texttt{diisEnd}]\hfill \\
	Density RMSD threshold for the end of the DIIS. Needs \ttt{useConvAcceleration = true}. By default \ttt{1.0e-4}.
	\item [\texttt{keepCoulombCache}]\hfill \\
	The Fock matrix contributions of the passive systems via their Coulomb interaction is not deleted and kept on disk. By default \ttt{false}.
\end{description}
\subsection{Task: FXD}
\label{sec:FXD}
This task performs fragment charge difference (FCD), fragment excitation difference (FED) and multistate FED-FCD diabatization calculations.
So far, this task assumes that the previous calculation was a TDA/CIS-type calculation. 
\subsubsection{Example Input}
\begin{lstlisting}
+task FXD
  act system
  donoratoms {start end}
  acceptoratoms {start end}
  fed true
-task
\end{lstlisting}
\subsubsection{Basic Keywords}
\begin{description}
	\item [\texttt{activeSystem}]\hfill \\
	The system for which the diabatization procedure is performed.
	\item [\texttt{donorAtoms}]\hfill \\
	Specifying the atoms belonging to the donor fragment \ttt{\{start end\}}. Starts with zero. 
	\item [\texttt{acceptorAtoms}]\hfill \\
	Specifying the atoms belonging to the donor fragment \ttt{\{start end\}}. 
	\item [\texttt{FED}]\hfill \\
	If \texttt{true} performs a fragment excitation difference calculation. The default is \ttt{false}.
	\item [\texttt{FCD}]\hfill \\
	If \texttt{true} performs a fragment charge difference calculation. The default is \ttt{false}.
	\item [\texttt{multistateFXD}]\hfill \\
	If \texttt{true} performs a multistate FED-FCD. The default is \ttt{false}.
	\item [\texttt{states}]\hfill \\
  Specifies the number for excited states used for the diabatization procedure. The default is \ttt{100}.
	\item [\texttt{loewdinPopulation}]\hfill \\
	Whether a Löwdin populations are used for the diabatization. The default is \ttt{true}. If it is set to \ttt{false}, Mulliken populations are used.
	\item [\texttt{writeTransformedExcitationVectors}]\hfill \\
	Writes the diabatic excitation vector to disk. The default is \ttt{false}. This is needed to plot the corresponding NTOs.
\end{description}

\subsection{Task: Generalized Direct Orbital Selection\label{task:gdos}}
This task performs a generalized orbital selection for multiple structures along a reaction coordinate.
This task does not localize any orbitals! A generalized variant of the algorithm of
the active space selection task (see Section~\ref{sec:activeSpaceTask}) is employed.
\subsubsection{Example Input}
\begin{lstlisting}
 +task gdos
   act reactant
   act product
   env r1
   env r2
   env r3
   env p1
   env p2
   env p3
   similarityLocThreshold {1e-1 1e-2}
   similarityKinEnergyThreshold {1e-1 1e-2}
 -task
\end{lstlisting}
\subsubsection{Basic Keywords}
\begin{description}
 \item [\texttt{Name}]\hfill \\
 Aliases for this task are \ttt{GENERALIZEDDOS} and \ttt{GDOS}.
 \item [\texttt{Supersystems}]\hfill \\
 Accepts multiple supersystems (\ttt{super}) which are used for the orbital comparison.
 \item [\texttt{ActiveSystems}]\hfill \\
 Accepts multiple active systems which are the supersystem for the orbital comparison.
 \item [\texttt{EnvironmentSystems}]\hfill \\
 Accepts multiple environment systems which are used as the subsystem into which the
 supersystems are partitioned. The oder of the subsystems is important. The
 first set of subsystems is used for the first active system, etc., as given in the
 example. Within each set the subsystems importance to the relative energies between
 the supersystems decreases with increasing subsystem index.
 \item [\texttt{similarityLocThreshold}]\hfill \\
 The threshold for the comparison of partial charges. The default is ${0.05}$.
 \item [\texttt{similarityKinEnergyThreshold}]\hfill \\
 The threshold for the comparison of the orbital kinetic energy. The default is ${0.05}$.
 \item [\texttt{populationAlgorithm}]\hfill \\
 The algorithm used to calculate partial charges. The default are shell-wise intrinsic atomic orbital (IAO)
 charges (\ttt{IAOShell}). Other useful options are Mulliken charges (\ttt{MULLIKEN}) and atom-wise IAO
 charges (\ttt{IAO}).
 \item [\texttt{mapVirtuals}]\hfill \\
 If true, the virtual orbitals are considered in the orbital mapping. By default \ttt{false}.
 \item [\texttt{bestMatchMapping}]\hfill \\
 If true, the selection thresholds are optimized to provide a qualitative orbital map, \emph{i.e.},
 the thresholds are chosen such that they minimize the number of unmappable orbitals
 under the constraint that they are smaller than \texttt{scoreStart}. By default \ttt{false}.
\end{description}
\subsubsection{Advanced Keywords}
\begin{description}
    \item [\texttt{localizationThreshold}]\hfill \\
    The atom-wise Mulliken population threshold for the assignment of atoms to the subsystems. This is done
    after the supersystem orbital set has been separated into active and environment orbitals. If the total
    population of the active orbitals on a given atoms exceeds this threshold it is assigned to the active
    system. The default is $0.8$ au.
    \item [\texttt{usePiBias}]\hfill \\
    Use a selection threshold scaling based on the number of significant, orbital-wise partial charges for
    each orbital. This is used to reduce the over-selection of $\pi$-orbitals. The default is \ttt{false}.
    \item [\texttt{biasThreshold}]\hfill \\
    The threshold for the evaluation of the number of significant partial charges for. The default is
    \ttt{0.01}.
    \item [\texttt{biasAverage}]\hfill \\
    The averaging-parameter for the threshold scaling used for \ttt{usePiBias true}.
    The default is $12.0$.
    \item [\texttt{prioFirst}]\hfill \\
    Prioritize the first subsystem of each subsystem set for the atom-assignment after
    the orbital partitioning.
    \item [\texttt{writeScores}]\hfill \\
    If true, the scores at which each orbital is selected is written to file.
    The orbitals are not partitioned into subsystems.
    This is achieved with a large number of on a logarithmic scale tightly packed DOS thresholds.
    By default \ttt{false}.
    \item [\texttt{scoreStart}]\hfill \\
    The start of the DOS-thresholds to be scanned for \texttt{writeScores}.
    By default $0.1$.
    \item [\texttt{scoreEnd}]\hfill \\
    The end of the DOS-thresholds to be scanned for \texttt{writeScores}.
    By default 1e-4.
    \item [\texttt{nTest}]\hfill \\
    The number of thresholds scanned for \texttt{writeScores}. By default $1000$.
    \item [\texttt{writeGroupsToFile}]\hfill \\
    If true, a file is created containing the orbital set map between structures. By default \ttt{false}.
    \item [\texttt{checkDegeneracies}]\hfill \\
    If true, the orbitals are checked if they are very similar with respect to the comparison criteria. By default \ttt{true}.
    \item [\texttt{degeneracyFactor}]\hfill \\
    Threshold scaling for the degeneracy check. By default \ttt{1.0}.
 \end{description}

\subsection{Task: Geometry Optimization}
This task performs geometry optimizations for a given structure. This task can perform both a subystem-based (Freeze-and-Thaw) and a supermolecular geometry optimization.
\subsubsection{Example Input}
\begin{lstlisting}
 +task Opt
   system water
 -task
\end{lstlisting}
\begin{lstlisting}
 +task Opt
  system water
  optAlgorithm SQNM
  +SQNM
   historyLength 20
   alpha 0.5
  -SQNM
 -task
\end{lstlisting}
\subsubsection{Basic Keywords}
\begin{description}
 \item [\texttt{name}]\hfill \\
   Aliases for this task are \ttt{GeometryOptimizationTask}, \ttt{GeoOpt} and \ttt{Opt}.
 \item [\texttt{activeSystems}]\hfill \\
   Optimizes all active systems. If more than one system is given, they are coupled via Freeze-and-Thaw calculations.
 \item [\texttt{environmentSystems}]\hfill \\
   Environment systems are added to Freeze-and-Thaw or Frozen-Density Embedding calculations of the active system(s) but remain electronically frozen and are not optimized.
 \item [\texttt{sub-blocks}]\hfill \\
   Embedding (\ttt{emb}) settings are added via sub-blocks in the task settings.
   Prominent settings in the embedding block that are relevant for this task, and their defaults are:
   \ttt{naddXCFunc=PW91}, \ttt{embeddingMode=naddfunc}, \ttt{naddkinfunc=PW91K}.
   \\
   Stabilized Quasi-Newton Method (\ttt{SQNM}) settings are added via sub-block in the task settings.
   Available options are:\\
   \ttt{historyLength} to determine the maximum number of previous cycles used in the algorithm. The default is \ttt{10}. \\
   \ttt{epsilon} to set the threshold for significant contributions of the overlap eigenvalues. The default is \ttt{1.0e-4}. \\
   \ttt{alpha} to set the initial step length. The default is \ttt{1.0}. \\
   \ttt{energyThreshold} to set the energy threshold to determine whether a step is accepted. The default is \ttt{1.0e-6}. \\
   \ttt{trustRadius} to set the trust radius - if the maximum displacement exceeds the trust radius, the whole step will be scaled down so that the new maximum displacement is the trust radius. The default is \ttt{0.1}. \\
 \item [\texttt{gradType}]\hfill \\
   The type of gradients used for geometry optimization. Possible options are analytic evaluation with the keyword \ttt{ANALYTICAL} or numerically (3 pt. scheme) with the keyword \ttt{NUMERICAL}. The default is \ttt{ANALYTICAL}.
 \item [\texttt{maxCycles}]\hfill \\
   Maximum number of geometry optimization cycles. The default is \ttt{100}.
 \item [\texttt{rmsgradThresh}]\hfill \\
   RMS convergence criterion for the gradient. The default is \ttt{1.0e-4}. 
 \item [\texttt{energyChangeThresh}]\hfill \\
   Convergence criterion for the energy change. The default is \ttt{5.0e-6}. 
 \item [\texttt{maxGradThresh}]\hfill \\ 
   Convergence criterion for the gradient maximum. The default is \ttt{3.0e-4}. 
 \item [\texttt{stepThresh}]\hfill \\
   Convergence criterion for the step threshold. The default is \ttt{2.0e-3}. 
 \item [\texttt{maxStepThresh}]\hfill \\
   Convergence criterion for the step threshold maximum. The default is \ttt{4.0e-3}. 
 \item [\texttt{numGradStepSize}]\hfill \\
   Step size for numerical gradients. The default is \ttt{1.0e-3}. 
 \item [\texttt{printLevel}]\hfill \\
   A value to regulate the amount of output the user is provided during each FaT iteration:
   \ttt{0}: No output; \ttt{1}: Print SCF results and grid information; \ttt{2}: Print SCF cycle info, SCF results and grid information. The default is \ttt{1}
 \item [\texttt{transInvar}]\hfill \\
   Make gradients translationally invariant. The default is \ttt{false}
 \item [\texttt{FaTmaxCycles}]\hfill \\
   The maximum number of FaT iterations. The default is \ttt{50}.
 \item [\texttt{FaTgridCutOff}]\hfill \\
   A distance cutoff for the integration grid used to calculate the non-additive energy functional potentials. Negative values correspond to no cutoff used. The default is \ttt{-1.0}.
 \item [\texttt{FaTenergyConvThresh}]\hfill \\
 Convergence criterion for the density w.r.t. Freeze-and-Thaw. The default is \ttt{1.0e-6}.
 \item [\texttt{optAlgorithm}]\hfill \\
  Algorithm to be used in the optimization. Options are \ttt{BFGS} (default) and \ttt{SQNM}.
\end{description}

\subsection{Task: Gradient}
This task performs a gradient calculation for a given structure. In case of a subsystem gradient calculation, this task will use all active systems for the gradient calculation. Furthermore, all active systems will be included in the Freeze-and-Thaw procedure. When additional environment systems are given, Frozen-Density Embedding gradients are calculated for the active subsystems.
\subsubsection{Example Input}
\begin{lstlisting}
 +task GradientTask
   system water
 -task
\end{lstlisting}
\subsubsection{Basic Keywords}
\begin{description}
\item [\texttt{name}]\hfill \\
  Aliases for this task are \ttt{GradientTask}, \ttt{Gradient} and \ttt{Grad}.
\item [\texttt{activeSystems}]\hfill \\
  Electronically optimizes all active systems. If more than one system is given, they are coupled via Freeze-and-Thaw calculations.
\item [\texttt{environmentSystems}]\hfill \\
  Environment systems are added to Freeze-and-Thaw or Frozen-Density Embedding calculations of the active system(s) but remain electronically frozen and are not optimized.
\item [\texttt{sub-blocks}]\hfill \\
  Embedding (\ttt{emb}) settings are added via sub-blocks in the task settings.
  Prominent settings in the embedding block that are relevant for this task, and their defaults are:
  \ttt{naddXCFunc=PW91}, \ttt{embeddingMode=naddfunc}, \ttt{naddkinfunc=PW91K}.
\item [\texttt{gradType}]\hfill \\
  The type of gradients used. Possible options are analytic evaluation with the keyword \ttt{ANALYTICAL} or numerically (3 pt. scheme) with the keyword \ttt{NUMERICAL}. The default is \ttt{ANALYTICAL}.
\item [\texttt{numGradStepSize}]\hfill \\
  Step size for numerical gradients. The default is \ttt{1.0e-3}. 
\item [\texttt{transInvar}]\hfill \\
  Make gradients translationally invariant. The default is \ttt{false}
\item [\texttt{FDEgridCutOff}]\hfill \\
  A distance cutoff for the integration grid used to calculate the non-additive energy functional potentials. Negative values correspond to no cutoff used. The default is \ttt{-1.0}.
\item [\texttt{FaTmaxCycles}]\hfill \\
  The maximum number of FaT iterations. The default is \ttt{50}.
\item [\texttt{FaTenergyConvThresh}]\hfill \\
  Convergence criterion for the density w.r.t. Freeze-and-Thaw. The default is \ttt{1.0e-6}.
\item [\texttt{print}]\hfill \\
  If \ttt{false} the system-wise gradients are not printed to the output. The default is \ttt{true}.
\item [\texttt{printTotal}]\hfill \\
  Enables the printing of the gradients for all atoms of all active systems. The default is \ttt{false}.
  Set this to \ttt{true} for \textsc{SNF}~\cite{SNF2002} calculations.
\end{description}

\subsection{Task: GW}
This task performs Many-Body-Perturbtaion Theory (MBPT) calculations. Either for the total energy via the direct Random-Phase-Approximation (dRPA) or for quasi-particle energies via the GW method. 
\subsubsection{Example Input}
\begin{lstlisting}
 +task GW
   act water
 -task
\end{lstlisting}
\subsubsection{Basic Keywords}
\begin{description}
  \item [\texttt{name}]\hfill \\
  Aliases for this task are \ttt{GWTask} and \ttt{GW}.
  \item [\texttt{activeSystem}]\hfill \\
  The system for which the calculation should be performed.
  \item [\texttt{environmentSystems}]\hfill \\
  Environment systems are included in a subsystem MBPT calculation via their screning contribution to the screened Coulomb interaction of the active subsystem. This is only valid for \ttt{gwtype: AC, CD} and for \ttt{mbpttype: RPA}.  
  \item [\texttt{mbpttype}]\hfill \\
  The type of MBPT calculation. Options are \ttt{GW} or \ttt{RPA}. The default is \ttt{GW}.
  \item [\texttt{gwtype}]\hfill \\
  The GW algorithm used in the GW calculation. The options are \ttt{Analytic}, \ttt{CD} (Contour-Deformation) or \ttt{AC} (Analytic-Continuation). The default is \ttt{CD}. Note: In case of \ttt{Analytic} a previous calculation of excitation energies is required (for exact results: All RPA excitation energies for the system need to be calculated via a previous LRSCFTask). In case of RPA screening this means the LRSCFTask needs to be used with \ttt{func HARTREE}.
  \item [\texttt{linearized}]\hfill \\
  Whether the quasi-particle energies are linearized. The default is \ttt{false}.
  \item [\texttt{qpIterations}]\hfill \\
  The number of quasi-particle iterations for a G0W0 calculation. The default is \ttt{0}.
  \item [\texttt{eta}]\hfill \\
  Imaginary shift parameter. The default is \ttt{0.001}.
  \item [\texttt{nVirt}]\hfill \\
  The number of virtual orbitals included in GW calculation (starting from the LUMO). The default is \ttt{10}.
  \item [\texttt{nOcc}]\hfill \\
  The number of occupied orbitals included in GW calculation (starting from the HOMO). The default is \ttt{10}.
  \item [\texttt{evGW}]\hfill \\
  Whether an evGW calculation is performed. The default is \ttt{false}.
  \item [\texttt{evGWcycles}]\hfill \\
  The number of evGW cycles. The default is \ttt{5}.
  \item [\texttt{ConvergenceThreshold}]\hfill \\
  The HOMO-LUMO gap convergence threshold for qpiterations/evGW cycles. The default is \ttt{1e-6} a.u..
  \item [\texttt{densFitCache}]\hfill \\
  The type of density fitting used for the four center integrals (\ttt{RI},\ttt{ACD},\ttt{ACCD}). The default is \ttt{RI}.
  \item [\texttt{ltconv}]\hfill \\
  Convergence parameter for the Laplace transformation if LT-(AC)-GW is used. By default \ttt{0}, which implies that the LT is not used.
  \item [\texttt{frozenCore}]\hfill \\
  Removes core orbitals from the reference orbitals (tabulated number for each atom type). The default is \ttt{false}.
  \end{description}
\subsubsection{Advanced Keywords}
\begin{description}
  \item [\texttt{gridCutOff}]\hfill \\
  A distance cutoff for the integration grid used to calculate the energy functional and potentials. Negative values correspond to no cutoff used. The default is \ttt{-1.0}.
  \item [\texttt{integrationPoints}]\hfill \\
  The number of integration points used for CD-GW and dRPA. The default is \ttt{128}.
  \item [\texttt{padePoints}]\hfill \\
  The number of points used in the pade approximation for analytic continuation. The default is \ttt{16}.
  \item [\texttt{fermiShift}]\hfill \\
  The initial Fermi shift of the HOMO/LUMO for analytic continuation. The default is \ttt{0.0} eV.  
  \item [\texttt{derivativeShift}]\hfill \\
  The shift for the evaluation of the numerical derviation of the self-energy (for linearization). The default is \ttt{0.002} Ha.
  \item [\texttt{imagShift}]\hfill \\
  Additional imaginary shift for numerical derivative if the derivative is larger than one. The default is \ttt{0.001} Ha.
  \item [\texttt{diis}]\hfill \\
  Whether a DIIS is used for convergence acceleration in qpIterations or evGW cycles. The default is \ttt{true}.
  \item [\texttt{diisMaxStore}]\hfill \\
  The numbers of DIIS vectors to be stored. The default is \ttt{10}.
  \item [\texttt{nafThresh}]\hfill \\
  The threshold for the naf functions. The default is \ttt{0}. NAFs are used if this threshold is != 0.
  \item [\texttt{subsystemAuxillaryBasisOnly}]\hfill \\
  Whether subsystem screening contributions should be calculated with subsystem auxiliary basis only. The default is \ttt{false}.
  \item [\texttt{damping}]\hfill \\
  Damping factor for convergence acceleration. The default is \ttt{0.2}.
  \item [\texttt{freq}]\hfill \\
  Start, end, stepsize for real axes frequency of the self-energy (only working for GW-Analytic). The structure is \ttt{start end stepsize}. The default is \ttt{$\{\}$}.
  \item [\texttt{gap}]\hfill \\
  Whether to shift occupied and virtual orbitals not included in the GW caclulation by the gap of change of the highest/lowest included occupied/virtual orbital. The default is \ttt{false}.
  \item [\texttt{environmentScreening}]\hfill \\
  Whether environmental screening is included in an embedded calculation if environmental subsystems are set. The default ist \ttt{true}.
  \item [\texttt{coreOnly}]\hfill \\
  Removes all but core orbitals from the reference orbitals. The default is \ttt{false}.
\end{description}

\subsection{Task: Hessian}
This task performs a Hessian calculation for a given structure. In case of a subsystem gradient calculation, this task will use all active systems for the gradient calculation. Furthermore, all active systems will be included in the Freeze-and-Thaw procedure. When additional environment systems are given, Frozen-Density Embedding Hessians are calculated for the active subsystems.
\subsubsection{Example Input}
\begin{lstlisting}
 +task HESS
   system water
 -task
\end{lstlisting}
\subsubsection{Basic Keywords}
\begin{description}
\item [\texttt{name}]\hfill \\
 Aliases for this task are \ttt{HessianTask}, \ttt{Hessian} and \ttt{Hess}.
\item [\texttt{activeSystems}]\hfill \\
 Optimizes all active systems. If more than one system is given, they are coupled via Freeze-and-Thaw calculations.
\item [\texttt{environmentSystems}]\hfill \\
 Environment systems are added to Freeze-and-Thaw or Frozen-Density Embedding calculations of the active system(s) but remain electronically frozen and are not optimized.
\item [\texttt{sub-blocks}]\hfill \\
 Embedding (\ttt{emb}) settings are added via sub-blocks in the task settings.
 Prominent settings in the embedding block that are relevant for this task, and their defaults are:
 \ttt{naddXCFunc=PW91}, \ttt{embeddingMode=naddfunc}, \ttt{naddkinfunc=PW91K}.
\item{hessType}\hfill \\
 The type of the Hessian. Possible options are analytic evaluation with the keyword \ttt{ANALYTICAL} (not implemented yet) or numerically (3 pt. scheme) with the keyword \ttt{NUMERICAL}. The default is \ttt{NUMERICAL}. Note that full numerical Hessians (gradient and Hessian) are numerically unstable. 
\item [\texttt{gradType}]\hfill \\
 The type of the gradient. Possible options are analytic evaluation with the keyword \ttt{ANALYTICAL} or numerically (3 pt. scheme) with the keyword \ttt{NUMERICAL}. The default is \ttt{ANALYTICAL}. 
\item [\texttt{numGradStepSize}]\hfill \\
 Step size for numerical gradients. The default is \ttt{1.0e-3}. 
\item [\texttt{numHessStepSize}]\hfill \\
 Step size for numerical Hessian. The default is \ttt{1.0e-3}. 
\item [\texttt{FaTgridCutOff}]\hfill \\
  A distance cutoff for the integration grid used to calculate the non-additive energy functional potentials. Negative values correspond to no cutoff used. The default is \ttt{-1.0}.
\item [\texttt{FaTmaxCycles}]\hfill \\
  The maximum number of FaT iterations. The default is \ttt{50}.
\item [\texttt{FaTenergyConvThresh}]\hfill \\
  Convergence criterion for the density w.r.t. Freeze-and-Thaw. The default is \ttt{1.0e-6}.
\item [\texttt{printToFile}]\hfill \\
  Prints the Hessian to HDF5. The default is \ttt{true}.
\end{description}

\subsection{Task: Import Cavity}\label{sec:tasks:ImportCavityTask}
This task loads a molecular surface of a given system with all necessary quantities in .h5 files.
This will load the CavityData.h5 and a PCMChargesData.h5 file of a previous calculation, and in case \ttt{cavityFormation} in the PCM block is set to \ttt{true}, a VDWCavityData.h5 file will also need to be loaded.
\subsubsection{Example Input}
\begin{lstlisting}
 +task IMPORTCAVITYTASK
  act water
  cavityPath ..
  vdwcavityPath ..
  fdecavity false
 -task
\end{lstlisting}

\subsubsection{Basic Keywords}
\begin{description}
\item [\texttt{name}]\hfill \\
  Aliases for this task are \ttt{IMPORTCAVITYTASK} and \ttt{IMPORTCAVITY}.
\item [\texttt{activeSystems}]\hfill \\
  The system to be equipped with a molecular surface.
\item [\texttt{cavityPath}]\hfill \\
  The file path of the CavityData.h5 file to be imported.
\item [\texttt{vdwcavityPath}]\hfill \\
  The file path of the vdwcavityData.h5 file to be imported. Needed in case you have \ttt{cavityFormation} set to \ttt{true} in the PCM block.
\item [\texttt{fdecavity}]\hfill \\
  Manual switch to toggle between importing a cavity for one system or a supersystem cavity for several subsystems. A supersystem cavity can be loaded for one system within this task, but will be applied to all systems as a supersystem cavityin subsequent relevant tasks such as Freeze-And-Thaw. \ttt{false} by default.
\end{description}

\subsection{Task: Local Correlation}\label{task:localCorrelation}
This tasks performs a local correlation (DLPNO-coupled cluster or MP2) calculation. This task only
calls other tasks. It serves as an input helper. It calls the localization task
(see Section~\ref{task:localization}) and the MP2 (see Section~\ref{task: mp2}) or Coupled Cluster task
(see Section~\ref{sec:coupledClusterTask}).
\subsubsection{Example Input}
\begin{lstlisting}
 +task LC
   system water
   +lc
     method DLPNO-CCSD(T0)
   -lc
 -task
\end{lstlisting}
\subsubsection{Basic Keywords}
\begin{description}
 \item [\texttt{name}]\hfill \\
   Aliases for this task are \ttt{LOCALCORRELATIONTASK}, \ttt{LC}.
 \item [\texttt{activeSystems}]\hfill \\
   Accepts a single active system which will be used in the calculation.
 \item [\texttt{environmentSystems}]\hfill \\
   Accepts a list of environment systems that are included in the Fock operator.
 \item [\texttt{sub-blocks}]\hfill \\
  Embedding (\ttt{\hyperref[sec:scb:emb]{emb}}) settings, local correlation (\ttt{\hyperref[sec:scb:lc]{lc}}) settings and
  orbital localization task (\ttt{\hyperref[task:localization]{loc}}) settings are added via sub-blocks in the task settings.
  By default, \ttt{splitValenceAndCore = true} is set for the localization task settings.
\end{description}
\subsubsection{Advanced Keywords}
\begin{description}
  \item [\texttt{maxCycles}]\hfill \\
  The maximum number of cycles allowed until convergence is expected by the coupled cluster iterations.
  By default, 100 cycles are allowed.
  \item [\texttt{normThreshold}]\hfill \\
  The threshold at which the coupled cluster/MP2 iterations are considered converged. By default \ttt{1.0e-5}.
  \item [\texttt{writePairEnergies}]\hfill \\
  Write the pair energies to a file with name: systemName\_pairEnergies\_CCSD.dat or systemName\_pairEnergies\_MP2.dat
\end{description}

\subsection{Task: Localization}\label{task:localization}
This task performs an orbital localization for the given active system. If the orbitals of
the active system are supposed to be aligned to the orbitals of a template system, the latter
has to be given as an environment system.

\subsubsection{Example Input}
\begin{lstlisting}
 +task LOC
   system water
   locType IBO
   splitValenceAndCore true
 -task
\end{lstlisting}
\subsubsection{Basic Keywords}
\begin{description}
 \item [\texttt{name}]\hfill \\
   Aliases for this task are \ttt{LocalizationTask}, \ttt{Localization} and \ttt{LOC}.
 \item [\texttt{activeSystems}]\hfill \\
   Accepts a single active system which will be used in the orbital localization
   (the orbitals are changed in place).
 \item [\texttt{environmentSystems}]\hfill \\
   Accepts a single environment systems that may be used as a template system for an orbital alignment.
   This is only relevant for \ttt{locType=ALIGN}.
 \item [\texttt{locType}]\hfill \\
   The orbital localization scheme to be used. The default is the intrinsic bond orbital scheme \ttt{IBO}.
   Other options are the  Pipek-Mezey (\ttt{PM}) scheme, the Foster--Boys (\ttt{FB}) scheme, the localization
   by Edminston and Ruedenberg (\ttt{ER}), a localization that does not retain orbital orthogonality
   (\ttt{NO}) and an orbital alignment procedure that aligns the orbitals to a given template system \ttt{ALIGN}\cite{Bensberg2020}.
 \item [\texttt{splitValenceAndCore}]\hfill \\
   If \ttt{true}, the valence and core orbitals are localized independently (no mixing). This is useful
   for local correlation methods. Furthermore, core like orbitals will resemble e.g. their $1s$ character
   more closely. The default is \ttt{false}. If set to true and \texttt{useEnergyCutOff = false}, core
   orbitals will be determined using tabulated  numbers of core orbitals for each element, selecting always
   the lowest orbitals according to their energy eigenvalue. If desired an energy cut-off
   (see \texttt{useEnergyCutOff}) can be used instead or a number of core orbitals may be specified manually.
   Note that the information about core orbitals is kept between tasks.
 \item [\texttt{localizeVirtuals}]\hfill \\
   If true, the virtual orbitals are localized as well. By default \ttt{false}. This is only supported for the IBO and ALIGN schemes.
\end{description}
\subsubsection{Advanced Keywords}
\begin{description}
 \item [\texttt{maxSweeps}]\hfill \\
   Most orbital localization schemes are iterative, this is the maximum number of cycles allowed.
   The default is $100$.
 \item [\texttt{useEnergyCutOff}]\hfill \\
   If true, core orbitals are determined using an energy cut-off. By default \ttt{true}.
   Needs \texttt{splitValenceAndCore = true}.
 \item [\texttt{energyCutOff}]\hfill \\
    The energy cut-off used for the core orbital selection. The default is $-5.0$ au.
 \item [\texttt{nCoreOrbitals}]\hfill \\
    Use a predefined number of core orbitals. Needs \texttt{useEnergyCutOff = false} and \texttt{splitValenceAndCore = true}.
    By default this is not used, \emph{i.e.}, the number of orbitals is set to \ttt{infinity}.
 \item [\texttt{useKineticAlign}]\hfill \\
   If \ttt{true}, the orbital kinetic energy is used as an additional criterion in the orbital alignment.
   The default is \ttt{false}.
 \item [\texttt{alignExponent}]\hfill \\
   The exponent for the penalty function used in the orbital alignment. This has to be an even integer
   larger or equal to $2$. The default is $4$. Exponents of $8$ or larger are not recommended since they
   may lead to numerical instabilities.
 \item [\texttt{splitVirtuals}]\hfill \\
   If true, the valence virtuals and diffuse virtuals are localized separately. By default \ttt{true}.
 \item [\texttt{virtualEnergyCutOff}]\hfill \\
   Orbital eigenvalue threshold to select diffuse virtual orbitals. By default $1.0$.
 \item [\texttt{nRydbergOrbitals}]\hfill \\
   Use a predefined number of diffuse virtual orbitals. Not used by default.
 \item [\texttt{replaceVirtuals}]\hfill \\
   Reconstruct the virtual orbitals before localization by projecting all occupied orbitals
   and cleanly separating valence virtuals from diffuse virtuals. This should be used if the
   \ttt{IBO} or \ttt{ALIGN} approaches are chosen and the same orbital set was not already reconstructed
   in this manner before. By default \ttt{false}.
\end{description}

% \clearpage
\subsection{Task: LRSCF (TDDFT/CC2/BSE)}
This task performs linear-response TDDFT/TDA calculations. In case of an uncoupled embedding calculation additional environment subsystems need to be added. In order to perform a coupled calculation more than one active system needs to be specified. Note that a coupled calculation requires a previous uncoupled calculation for the respective subsystem. These capabilities exist for excitation energies, oscillator strengths, rotatory strengths and both damped and regular linear-response properties (polarizabilties, optical rotation).\\

\noindent
In Serenity version 1.4.0 the approximate second-order coupled cluster model CC2 was added. Furthermore, CIS(D), its iterative variant CIS(D$_\infty$) and
ADC(2) are available. For these methods, also transition moments from the ground state can be computed (and thus oscillator and rotatory strengths).
Additionally, Bethe--Salpeter equation (BSE) calculations with and without the TDA can be performed.

\subsubsection{Example Input}
\begin{lstlisting}
+task LRSCF
  act water
  nEigen 10
  method tddft
  frequencies { 2 3 4 }
  frozenCore true
  +grid
    accuracy 7
    smallgridaccuracy 7
  -grid
-task
\end{lstlisting}
\subsubsection{Basic Keywords}
\begin{description}
    \item [\texttt{name}]\hfill \\
    Aliases for this task are \ttt{LRSCFTask} and \ttt{LRSCF}.
    \item [\texttt{activeSystems}]\hfill \\
    The systems for the LRSCF calculation. If only one system without an environment system is used, no additional kernel contributions due to the embedding procedure are taken into account. If more than one active system is used, a coupled calculation is performed. This requires a previous uncoupled calculation for the individual active subsystems.
    \item [\texttt{environmentSystems}]\hfill \\
    Additional kernel contributions are taken into account in case environmentSystems are used in the LRSCF calculation.
    \item [\texttt{sub-blocks}]\hfill \\
    Embedding (\ttt{emb}) settings are added via sub-blocks in the task settings.
    Prominent settings in the embedding block that are relevant for this task, and their defaults are:
    \ttt{naddXCFunc=NONE}, \ttt{embeddingMode=NONE}, \ttt{naddkinfunc=NONE}.\\
    Grid (\ttt{grid}) settings are added via sub-blocks in the task settings.
    Prominent settings in the grid block that are relevant for this task, and their defaults are:
    \ttt{accuracy=4}, \ttt{smallGridAccuracy=4}.
    Custom density functional settings (\ttt{customfunc}) settings are added via a sub-block in the task settings. This allows to define a custom functional for the evaluation of the kernel.
    \item [\texttt{nEigen}]\hfill \\
    Number of roots to be determined. The default is \ttt{4}.
    \item [\texttt{conv}]\hfill \\
    Convergence criterion for the iterative eigenvalue and response solvers. The default is \ttt{1.0e-5}.
    \item [\texttt{method}]\hfill \\
    Determines the method to be used. The default is \ttt{tddft}. \\ Also available are: \ttt{tda, cc2, cisdinf, cisd, adc2}.
    \item [\texttt{analysis}]\hfill \\
    If \ttt{false}, dominant contribution analysis and absorption/CD spectra will be suppressed. The default is \ttt{true}.
    \item [\texttt{cctrdens}]\hfill \\
    If \ttt{false}, no transition moments will be calculated for CC2, CIS(D$_\infty$) and ADC(2). The default is \ttt{false}.
    \item [\texttt{ccexdens}]\hfill \\
    If \ttt{false}, no excited state densities (and dipole moments) will be calculated for CC2, CIS(D$_\infty$) and ADC(2). The default is \ttt{false}.
    \item [\texttt{frequencies}]\hfill \\
    Frequencies for which dynamic polarizabilities and optical rotation are calculated for (in eV). More than one frequency can be given as~\ttt{$\{$0.1 0.2$\}$}. Default is an empty vector.
    \item [\texttt{frequencyRange}]\hfill \\
    Calculation of linear-response properties for a certain frequency range (in eV). Expects three arguments \ttt{$\{$start stop stepwidth$\}$}. Default is an empty vector.
    \item [\texttt{damping}]\hfill \\
    Damping parameter for response properties (finite lifetime effects, in eV), e.g. broadens the absorption spectrum. The default value is \ttt{0.0}.
    \item [\texttt{gauge}]\hfill \\
    The gauge chosen for the response property calculation. The options are \ttt{LENGTH} and \ttt{VELOCITY}. \ttt{LENGTH} gauge is chosen as default.
    \item [\texttt{rpaScreening}]\hfill \\
    Performs the exchange integral evaluation with static RPA screening. This keyword needs to be set to perform a BSE calculation in combination with \texttt{riCache}=\ttt{true}. The default is \ttt{false}. Note: If environmental subsystems are specified their screening contribution is included approximately. 
    \item [\texttt{restart}]\hfill \\
    Tries restarting from (preferably converged) eigenpairs that the task
    looks for in the system folder. The default is \ttt{false}.
    \item [\texttt{triplet}]\hfill \\
    Used for triplet excitations. The default ist \ttt{false}.
    \item [\texttt{scfstab}]\hfill \\
    Used for SCF wavefunction stability analysis. Defaults to \ttt{NONE}. Available are
    \begin{itemize}
      \item Real RHF $\rightarrow$ Real RHF    : $(A+B)$, singlet \ttt{scfstab real} (internal)
      \item Real RHF $\rightarrow$ Real UHF    : $(A+B)$, triplet \ttt{scfstab real} and \ttt{triplet true} (external)
      \item Real RHF $\rightarrow$ Complex RHF : $(A-B)$, singlet \ttt{scfstab nonreal} (external)
      \item Real UHF $\rightarrow$ Real UHF    : $(A+B)$ \ttt{scfstab real} (internal)
      \item Real UHF $\rightarrow$ Complex UHF : $(A-B)$ \ttt{scfstab nonreal} (external)
      \item Spin-Flip TD-HF/DFT \ttt{scfstab spinflip} (performed within the TDA)
    \end{itemize}
    The goal of an SCF is to minimize the energy with respect to orbital variation. At convergence of an SCF, the electronic gradient vanishes,
    i.e. $F_{ia} = 0$ corresponding to a stationary point. A sufficient condition for this stationary point to be a true minimum is that the matrix of 
    second derivatives with respect to to orbital variation (called the stability matrix) is positive definite. If it is not, there exist orbital rotations
    between occupied and virtual orbitals that further minimze the energy. A stability analysis diagonalizes the stability matrix and can be used
    to follow specific eigenvectors pointing towards the corresponding lower energy solution.
    One differentiates between internal (keeping the constraints of the original SCF wavefunction) and external (lifting certain constraints of the original SCF wavefunction) stability analyses.
    The [Real RHF $\rightarrow$ Real UHF] stability analysis (arguably most common), for example, can be used to identify triplet instabilities, i.e. finds
    if there exists a lower triplet (UHF) wavefunction than the found singlet (RHF) one. Spin-Flip TDDFT is not a stability analysis, but was
    implemented alongside them so it appears here for the sake of computational simplicity.
    \item [\texttt{stabroot}]\hfill \\
    Instruct to rotate the orbitals along the direction devised by the instability indexed by stabroot (usually 1, i.e. the lowest). 
    Another SCF task needs to be run. This is only possible for internal stability analysis.
    Defaults to \ttt{0}, which means that no root following will be done.
    \item [\texttt{stabscal}]\hfill \\
    Mixing parameter for the new orbitals. Defaults to \ttt{0.5}.\\
    $C_\mathrm{new} = C_\mathrm{old} \cdot U $ where $U=\mathrm{exp}(\ttt{stabscal} F)$ and $F$ contains the orbital rotation parameters.
    \item[\texttt{excGradList}]\hfill \\
    A 1-based list of excited states for which gradients are to be calculated. Excited-State gradients are calculated when this list is not empty.
    Calculating TDDFT gradients with Libxc (instead of XCFun) requires compilation of Serenity with the CMake flag \texttt{-DDISABLE\_KXC=OFF}.
    \end{description}
\subsubsection{Advanced Keywords}
\begin{description}
    \item [\texttt{maxCycles}]\hfill \\
    Maximum number of iterations for the iterative eigenvalue solver. The default is \ttt{100}.
    \item [\texttt{maxSubspaceDimension}]\hfill \\
    Maximum dimension of subspace of the iterative eigenvalue solver. The default is \ttt{1e9}.
    \item [\texttt{dominantThresh}]\hfill \\
    Orbital transitions with squared coefficients starting from the largest summed up to \ttt{dominantThresh} are considered dominant and their contribution is written into the output. The default value is \ttt{0.85}.
    \item [\texttt{func}]\hfill \\
    Exchange--correlation functional for the kernel evaluation. The default is to use the exchange--correlation functional as defined in the system settings. This can be customized by specifying a \ttt{+customfunc} block (see Sec.~\ref{sec:system:customfunc}) in the \ttt{LRSCFTask} input. The priority order is \ttt{func} from the \ttt{LRSCFTask} before the system settings, and within those categories a customized functional beats the composite variant.
    \item [\texttt{gaugeOrigin}]\hfill \\
    The gauge origin for the dipole integrals. The default is the center of mass of the molecule. Can be changed via \ttt{$\{x~y~z\}$}.
    \item [\texttt{besleyAtoms}]\hfill \\
    Number of atoms included in the orbital restriction according to Besley (the first $n$ atoms will be taken from the xyz file). The default is \ttt{0}.
    \item [\texttt{besleyCutoff}]\hfill \\
    Besley cut off for occupied and virtual orbitals. This needs to be specified in a vector \ttt{$\{$OCC VIRT$\}$}.
    \item [\texttt{excludeProjection}]\hfill \\
    Exclude all artificially shifted virtual orbitals from the set of reference orbitals in a projection-based embedding calculation. The default is \ttt{false}. This keyword needs to be handled carefully if chosen in combination with basis-set restriction.
    \item [\texttt{uncoupledSubspace}]\hfill \\
    Uncoupled subspace for the FDEc-LRSCF problem. Given a set of active subsystems, a subspace of excitations vectors can be defined by: \\  \ttt{$\{$ 2 1 2} \\
    \ttt{3 4 8 10 $\}$}, \\ where the first number $n$ gives the number of states, which are going to be included of that subsystem and the following $n$ numbers define the respective uncoupled eigenvectors (where the first excitation is labeled as 1). For this example, vectors 1 and 2 are taken from subsystem one, and vectors 4, 8 and 10 are chosen from active subsystem 2. If uncoupledSubspace is not set, all uncoupled vectors will be used to span the subspace.
    \item [\texttt{fullFDEc}]\hfill \\
    Converge the \emph{approximate} FDEc roots obtained from the uncoupled initial guess to the accuracy given by \ttt{convThresh}. In order to ensure convergence to the correct roots (and not any lower lying ones), a root-following procedure is employed. The default is \ttt{false}.
    \item [\texttt{loadType}]\hfill \\
    Reference states used to build transformation matrix for FDEc calculation. The options are \ttt{ISOLATED}, \ttt{UNCOUPLED} and \ttt{COUPLED}. The default is \ttt{UNCOUPLED}.
    \item[\texttt{couplingPattern}]\hfill\\
    Sets a special coupling pattern, e.g. for coupled / uncoupled coupling.
    \item[\texttt{diis}]\hfill\\
    Specifies whether the nonlinear eigenvalue solver uses a DIIS after preoptimization. If false, the quasi-linear Davidson algorithm will be used until \ttt{conv}
    is reached.
    \item[\texttt{diisStore}]\hfill\\
    Specifies how many diis vectors can be stored (for CC2 ground state and nonlinear
    eigenvalue solver). Default is 50.
    \item[\texttt{preopt}]\hfill\\
    Convergence threshold for the preoptimization of eigenvectors in nonlinear
    eigenvalue solvers for CC2/ADC(2). Up to this threshold, a quasi-linear
    Davidson algorithm will be used, after this a DIIS eigenvalue solver 
    is turned on and converged to the parameter given by \ttt{conv}. The default is \ttt{1e-3}.
    \item[\texttt{sss}]\hfill\\
    Scaling parameter for same-spin contributions (CC2/ADC(2)). The default is \ttt{1.0}.
    \item[\texttt{oss}]\hfill\\
    Scaling parameter for opposite-spin contributions (CC2/ADC(2)). The default is \ttt{1.0}.
    \item[\texttt{nafThresh}]\hfill\\
    Truncates the three-center MO integral basis using the natural auxiliary function technique.
    Threshold for truncation. The smaller, the fewer NAFs are truncated. NAFs are used if this 
    threshold is != 0. The default is \ttt{0}.
    \item[\texttt{sameDensity}]\hfill\\
    If two subsystems are used in the calculation with the same occupied but different virtual orbital spaces,
    the kernel must be evaluated with only one density. Because the \ttt{LRSCFTask} cannot find out so itself,
    this keyword is used to tell it which subsystems have the same density, e.g.~\ttt{$\{1\quad 2\}$} (start counting at 1).
    \item[\texttt{subsystemgrid}]\hfill\\
    Only includes the grid points associated with atoms of the specified subsystems in the kernel evaluation.
    \item[\texttt{partialResponseConstruction}]\hfill\\
    Invokes a partial response-matrix construction for (not full) FDEc calculations.
    Exploits symmetry of the response matrix and is therefore substantially faster than
    a regular FDEc calculation.
    The default is \ttt{false}.
    \item[\texttt{densFitJ}]\hfill\\
    Invokes density fitting for the Coulomb sigma vector. Uses the auxiliary basis defined in the system settings as \ttt{auxCLabel}.
    Options are \ttt{NONE}, \ttt{RI}, \ttt{ACD}, and \ttt{ACCD}. The default is \ttt{RI}.
    \item[\texttt{densFitK}]\hfill\\
    Invokes density fitting for the exchange sigma vector. Uses the auxiliary basis defined in the system settings as \ttt{auxCLabel}.
    Options are \ttt{NONE}, \ttt{RI}, \ttt{ACD}, and \ttt{ACCD}. The default is \ttt{NONE}.
    \item[\texttt{densFitLRK}]\hfill\\
    Invokes density fitting for the long-range exchange sigma vector. Uses the auxiliary basis defined in the system settings as \ttt{auxCLabel}.
    Options are \ttt{NONE}, \ttt{RI}, \ttt{ACD}, and \ttt{ACCD}. The default is \ttt{NONE}.
    \item[\texttt{densFitCache}]\hfill\\
    Invokes density fitting for RIIntegrals, used for ADC(2)/CC2. Uses the auxiliary basis defined in the system settings as \ttt{auxCLabel}.
    Options are \ttt{RI}, \ttt{ACD}, and \ttt{ACCD}. The default is \ttt{RI}.
    \item[\texttt{transitionCharges}]\hfill\\
    Calculates transition charges and stores them on disk. The default is \ttt{false}. Also calculates particle and hole populations, prints them and stores the hole-particle-correlation on disk.
    \item[\texttt{grimme}]\hfill\\
    Invokes Grimme's simplified TDA/TDDFT. The default is \ttt{false}.
    \item[\texttt{adaptivePrescreening}]\hfill\\
    Uses looser prescreening thresholds in the beginning of the Davidson iterations. Increases performance. The default is \ttt{true}.
    \item[\texttt{frozenCore}]\hfill\\
    Removes core orbitals from the reference orbitals (tabulated number for each atom type). The default is \ttt{false}.
    \item[\texttt{frozenVirtual}]\hfill\\
    Removes virtual orbitals from the reference orbitals that are higher lying than E(HOMO) + \ttt{frozenVirtual}. The default is \ttt{0.0} which does not remove any virtuals..
    \item[\texttt{coreOnly}]\hfill\\
    Removes all but core orbitals from the reference orbitals (can be used to perform X-ray calculations). The default is \ttt{false}.
    \item[\texttt{ltconv}]\hfill \\
	  Convergence parameter for the Laplace transformation if LT-SOS-CC2/ADC(2) is used. By default \ttt{0}. Note that this requires compilation with \ttt{-DSERENITY\_USE\_LAPLACE\_MINIMAX=ON}.
    \item[\texttt{aocache}]\hfill \\
	  Triggers that three center integrals are kept in memory for CC2/ADC(2) calculations. By default \ttt{true}.
    \item[\texttt{noKernel}]\hfill \\ 
	  Can be used to turn off the numerical integration of the XC and kinetic kernels. This keyword combined with
    \texttt{auxclabel MINPARKER\_S/SP} in the basis block of the system settings can be used to do TDDFT-ris calculations 
    \cite{zhou2023minimal}. By default \ttt{false}.
    \item[\texttt{approxCoulomb}]\hfill \\
	  This keyword accepts up to two distance thresholds (doubles) \{$t_1$ $t_2$\}, where $t_1$ must be smaller than $t_2$.
    If used in an LRSCFTask with two or more active subsystems, the inter-subsystem blocks of the CoulombSigmavector
    can be calculated at different levels of approximation for the $((ia)_I|(jb)_J)$-type integrals. Let $R$ denote the
    inter-subsystem distance, which is calculated as the closest pair of atoms of the two geometries, then the conditions are as follows:\\
    $R < t_1$: Four-center integrals (NORI) or RI using the union of the two basis sets (depending on the \texttt{densfitJ} keyword).\\
    $t_1 < R < t_2$: ``Monomer-RI''-type approximation where for the three-center integrals only fitting functions of the subsystem belonging to the orbital transition $ia$ are used. \\
    $t_2 < R$: Simplified TDA. \\
    If only one threshold is given, the second one is set to infinity. By default empty (nothing done \emph{approximately}).
    \item[\texttt{noCoupling}]\hfill \\
	  Skips off-diagonal blocks of the response matrix in the formation of $\sigma$-vectors no matter what. By default \ttt{false}.
    \item[\texttt{hypthresh}]\hfill \\
    Density threshold for TDDFT gradient calculations. If the density at a point falls below this threshold, the third derivatives of the xc functional (sometimes called the hyperkernel) are set to zero at this point. 
\end{description}

% \clearpage
\subsection{Task: MP2}\label{task: mp2}
\label{sec:MP2}
This task performs MP2 calculations for a given active system, probably using embedding or local correlation
settings. Effects of an optional PCM are taken only into account on the level of the orbitals/Fock matrix.

{\color{red}IMPORTANT:} The orbital localization step for local MP2 is not carried out
within this task! Please make sure that the orbitals have been localized before using the local
correlation version of this task (see Sec.~\ref{task:localization}).
\subsubsection{Example Input}
\begin{lstlisting}
+task MP2
  system acetone
  mp2Type RI
-task
\end{lstlisting}
\subsubsection{Basic Keywords}
\begin{description}
	\item [\texttt{name}]\hfill \\
	Aliases for this task are \ttt{MP2} and \ttt{MP2Task}.
	\item [\texttt{activeSystems}]\hfill \\
	Accepts a single active system for whose orbitals an MP2 correction is calculated.
	\item [\texttt{environmentSystems}]\hfill \\
	Accepts mutiple environment systems that are used in embedded calculations.
	\item [\texttt{sub-blocks}]\hfill \\
	The embedding (\ttt{emb}), local correlation (\ttt{lc}) and PCM (\ttt{pcm}) settings are added via sub-blocks in the task settings.
	Prominent settings in the embedding block that are relevant for this task, and their defaults are:
	\ttt{naddXCFunc=BP86}, \ttt{embeddingMode=LEVELSHIFT}.
	Similarly, \ttt{pnoSettings=NORMAL} is a commonly changed default in the local correlation settings.
	\item [\texttt{mp2Type}] \hfill \\ 
	MP2 type used. Full four center integral MP2 (\ttt{AO}), DF-MP2, using density fitting approach specified with \ttt{densfitCorr} in the system block (\ttt{DF}), local MP2 (\ttt{LOCAL}), Laplace-transform SOS-MP2 (\ttt{LT}). By default \ttt{DF}.
  \item[\texttt{maxResidual}]\hfill \\
	Convergence threshold for the local MP2 calculation. By default \ttt{1.0e-5}.
  \item[\texttt{sss}]\hfill \\
	Scaling parameter for same-spin contributions. By default \ttt{1.0}.
  \item[\texttt{oss}]\hfill \\
	Scaling parameter for opposite-spin contributions. By default \ttt{1.0}.
  \item[\texttt{ltconv}]\hfill \\
	Convergence parameter for the Laplace transformation if LT-SOS-MP2 is used. By default \ttt{1e-5}.
  \item[\texttt{maxCycles}]\hfill \\
  Maximum number of iterations before cancelling the amplitude optimization in local MP2. By default \ttt{100}.
  \item[\texttt{unrelaxedDensity}]\hfill \\
  If true, the unrelaxed MP2 density is calculated. By default \ttt{false}. 
\end{description}
\subsubsection{Advanced Keywords}
\begin{description}
  \item [\texttt{writePairEnergies}]\hfill \\
  Write the pair energies to a file with name: systemName\_pairEnergies\_MP2.dat.
  This only works for DLPNO-MP2.
\end{description}

\subsection{Task: Multipole Moment}
This task calculates the multipole moments of a given system, analytically or numerically.
\subsubsection{Example Input}
\begin{lstlisting}
 +task MULTIPOLEMOMENTTASK
  act water
  highestOrder 1
  origin COM
 -task
\end{lstlisting}

\subsubsection{Basic Keywords}
\begin{description}
\item [\texttt{name}]\hfill \\
  Aliases for this task are \ttt{MULTIPOLEMOMENTTASK} and \ttt{MULTI}.
\item [\texttt{activeSystems}]\hfill \\
  The systems whose multipole moments are to be calculated.
\item [\texttt{highestOrder}]\hfill \\
Calculate multipole moments up to this order (1=dipole, 2=quadrupole). 2 by default.
\item [\texttt{numerical}]\hfill \\
Multipole moments are calculated analytically using basis functions
and integrals per default (corresponding to \ttt{false}). If this is switched on,
they are calculated numerically using the density on an integration grid (corresponding to \ttt{true}).
\item [\texttt{origin}]\hfill \\
The spatial origin for the multipole calculation.
The origin of the Cartesian coordinates (\ttt{ORIGIN}, default) or the center
of mass of the given system (\ttt{COM}).
\item [\texttt{printTotal}]\hfill \\
Calculate and print the total multipole moment of all systems. The default is \ttt{false}.
Set this to \ttt{true} for \textsc{SNF}\cite{SNF2002} calculations.
\item [\texttt{printFragments}]\hfill \\
If set to \ttt{false} this suppresses the output of the multipole moment the individual subsystems.
The default is \ttt{true}.
\end{description}

\subsection{Task: Read/Write Orbitals to/from Disk}

This task allows to read and write orbitals from and to (non)-Serenity output files.\newline
Reading orbitals:
\begin{itemize}
  \item The ASCII-format for Turbomole is supported up to g functions.
  \item The xml format for Molpro is supported up to g functions.
  \item Molcas orbitals may be read up to arbitrary angular momentum.
  \item Furthermore, the task may read orbital files from Serenity.
\end{itemize}
Writing orbitals:
\begin{itemize}
  \item The ASCII-format for Turbomole is supported up to g functions. If a non-standard energy convergence threshold is used in Turbomole, this has to be changed in the written files afterwards.
  \item Molden files are supported up to g functions. Supports both spherical and cartesian harmonics.
\end{itemize}
Note that Serenity does not account for symmetry. Hence, the orbitals must be defined without any symmetry.
\subsubsection{Example Input}
\begin{lstlisting}
 +task READ
  act water
  fileformat turbomole
  path path/to/a/directory/containing/a/turbomole/ascii-mo-file
 -task
\end{lstlisting}

\subsubsection{Basic Keywords}
\begin{description}
 \item [\texttt{name}]\hfill \\
  Aliases for this task are \ttt{OrbitalsIOTask}, \ttt{ReadOrbs}, \ttt{Read}, \ttt{WriteOrbs}, \ttt{Write} and \ttt{IO}.
 \item [\texttt{activeSystems}]\hfill \\
  The system to which the orbitals should be read.
 \item [\texttt{fileFormat}]\hfill \\
  The file format to be expected by the task. Currently supported are the Turbomole ASCII-MO file format (\ttt{TURBOMOLE})
  Molpro's xml format (\ttt{MOLPRO}), the native \textsc{Serenity} HDF5 file format (\ttt{Serenity}), the Molcas-HDF5 file
  format (\ttt{Molcas}) and the \ttt{MOLDEN} file format. By default \ttt{TURBOMOLE}.
 \item [\texttt{resetCoreOrbitals}]\hfill \\
  If \texttt{fileFormat = SERENITY} and \ttt{true}, the core orbitals are reset to the core orbitals tabulated for the geometry. Otherwise, the core orbital
  definition on disk is used. Note that this is always done for \texttt{fileFormat = TURBOMOLE}.
 \item [\texttt{path}]\hfill \\
  The path to the orbital files. Default is \ttt{"."}.
 \item [\texttt{replaceInFile}]\hfill \\
  If true, copies the original orbital file and replaces the coefficients in this file by Serenity's coefficients. (Only implemented for Molcas HDF5 files).
  By default \ttt{false}.
 \item [\texttt{write}]\hfill \\
  If true, a file will be written instead of being read. By default \ttt{false}.
\end{description}

\subsection{Task: Orthogonalization}\label{task: ortho}
This task orthogonalizes orbitals.
\subsubsection{Example Input}
\begin{lstlisting}
	+task Ortho
	act water
	orthogonalizationScheme Loewdin
	-task
\end{lstlisting}

\subsubsection{Basic Keywords}
\begin{description}
	\item [\texttt{name}]\hfill \\
	Aliases for this task are \ttt{OrthogonalizationTask} and \ttt{Ortho}.
	\item [\texttt{Supersystems}]\hfill \\
	The first Supersystem is the system holding the orbitals resulting from the orthogonalization, if not specified the orthogonal orbitals are stored in a system
	called by the name of the first active system + "Ortho".
	\item [\texttt{activeSystem}]\hfill \\
	The active systems listed here are combined to a supersystem (see \ref{task: addition}) and the orbitals
	of this supersystem are orthogonalized and stored in the super system.
	\item [\texttt{orthogonalizationScheme}]\hfill \\
	The orthogonalization scheme used. Possible options are:\\
	\ttt{NONE}: No orthogonalization is performed.\\
	\ttt{LOEWDIN}: L\"owdin's symmetric orthogonalization\cite{lowdin1950, lowdin1970}.\\
	\ttt{PIPEK}: Pipek's iterative orthogonalization scheme\cite{pipek1985}.\\
	\ttt{BROER}: Broer's orthogonalization scheme based on corresponding orbitals\cite{broer1993}.\\
	\item [\texttt{maxIterations}]\hfill \\
	Maximum number of iterations performed in the Pipek scheme.
\end{description}

\subsection{Task: PCM Interaction Energy}
A task that allows the calculation of the dielectric energy correction from PCM for a given system (with associated density). Note that the apparent surface charges are not updated self-consistently!
\subsubsection{Example Input}
\begin{lstlisting}
+task PCM
  system water
-task
\end{lstlisting}

\subsubsection{Basic Keywords}
\begin{description}
    \item [\texttt{name}]\hfill \\
    Aliases for this task are \ttt{PCMInteractionEnergyTask} and \ttt{PCM}.
    \item [\texttt{activeSystem}]\hfill \\
    This task accepts a single active system for which the dielectric energy correction from PCM will be calculated.
    \item [\texttt{sub-blocks}]\hfill \\
    PCM (\ttt{\hyperref[sec:system:pcm]{pcm}}) settings are added via a sub-block in the task settings.    
\end{description}
\subsection{Task: Plot}
A task to export properties on 3D (cube) or 2D (planar) grid files. The .dat file containing the property on
a 2D grid contains the values as \ttt{x, y, z, value} of the property.

\subsubsection{Example Input (3D)}
\begin{lstlisting}
  +task Plot
    act water
    orbitals {3 4 5}
  -task
\end{lstlisting}

\subsubsection{Example Input (2D)}
\begin{lstlisting}
  +task Plot
    act water
    orbitals {3 4 5}
    atom1 1
    atom2 2
    atom3 3
  -task
\end{lstlisting}

\subsubsection{Basic Keywords}
\begin{description}
   \item [\texttt{name}]\hfill \\
  Aliases for this task are \ttt{PlotTask}, \ttt{Plot}, \ttt{CubeFileTask}, \ttt{CubeFile} and \ttt{Cube}.
   \item [\texttt{activeSystems}]\hfill \\
    Combines all active systems into one supersystem and plots its properties
    if available.
   \item [\texttt{environmentSystems}]\hfill \\
    Any environment system will extend the geometry and grid size
    only, thus subsystem properties can be displayed on a supersystem grid.
    \item [\texttt{gridSpacing}]\hfill \\
    The grid spacing in \AA{} along the unitvectors. Note that for a 2D plot the third entry is
    ignored. By default \ttt{\{$0.12, 0.12, 0.12$\}} (3D) or \ttt{\{$0.08, 0.08, 0.0$\}} (2D).
    \item [\texttt{borderWidth}]\hfill \\
    The border width in \AA{} around the geometry for which the grid is created. By default \ttt{2.0}.
\end{description}

\paragraph{Grid and plot settings for 3D plot}
\begin{description}
    \item [\texttt{xUnitVector}]\hfill \\
    The first unitvector defining the orientation of the cube grid. By default \ttt{\{$1.0, 0.0, 0.0$\}}.
    \item [\texttt{yUnitVector}]\hfill \\
    The second unitvector defining the orientation of the cube grid. By default \ttt{\{$0.0, 1.0, 0.0$\}}.
    \item [\texttt{zUnitVector}]\hfill \\
    The third unitvector defining the orientation of the cube grid. By default \ttt{\{$0.0, 0.0, 1.0$\}}.
    \item [\texttt{cavity}]\hfill \\
    Plot values on the molecular cavity. By default \ttt{False}.
    \item [\texttt{gridCoordinates}]\hfill \\
    Write grid coordinates to file. By default \ttt{False}.
\end{description}
\paragraph{Grid and plot settings for 2D plot}
\begin{description}
    \item [\texttt{p1}]\hfill \\
    First point which defines the plane in a 2D plot, this is also the origin of the plane.
    \item [\texttt{p2}]\hfill \\
    Second point which defines the plane in a 2D plot.
    \item [\texttt{p3}]\hfill \\
    Third point which defines the plane in a 2D plot.
    \item [\texttt{atom1}]\hfill \\
    The atom which coordinates should be used as p1. The counting starts at 1.
    \item [\texttt{atom2}]\hfill \\
    The atom which coordinates should be used as p2. The counting starts at 1.
    \item [\texttt{atom3}]\hfill \\
    The atom which coordinates should be used as p3. The counting starts at 1.
    \item [\texttt{projectCutOffRadius}]\hfill \\
    The maximal distance in \AA{} an atom in a 2D plot is allowed to have to be projected on the plane for the generation of the grid. By default \ttt{3.0}.
    \item [\texttt{xyHeatmap}]\hfill \\
    Plots the heatmap which is rotated to the xy plane and saves it as \_XYPLANE.dat. The geometry rotated with the same matrix is saved as \_MOLECULE\_ROTATED\_TO\_XYPLANE.
    This heatmap can be plotted with the \texttt{2DheatMap.py} script in the \texttt{tools} folder. By default \ttt{False}.
\end{description}

\paragraph{Properties which can be plotted}
\begin{description}
    \item [\texttt{density}]\hfill \\
    Plots the electron density. By default \ttt{False}.
    \item [\texttt{signedDensity}]\hfill \\
    Plots the 'signed density' defined as: $\mathrm{sign}(\nabla^2\rho(r))\rho(r)$ . By default \ttt{False}.
    \item [\texttt{orbitals}]\hfill \\
    Plots the orbitals from the list.
    \item [\texttt{allOrbitals}]\hfill \\
    Plots all orbitals. By default \ttt{False}.
    \item [\texttt{occOrbitals}]\hfill \\
    Plots all occupied orbitals. By default \ttt{False}.
    \item [\texttt{electrostaticPot}]\hfill \\
    Plots the electrostatic potential. By default \ttt{False}.
    \item [\texttt{sedd}]\hfill \\
    Plots the Single Exponential Decay Detector (SEDD). By default \ttt{False}.
    \item [\texttt{dori}]\hfill \\
    Plots the Density Overlap Regions Indicator (DORI). By default \ttt{False}.
    \item [\texttt{elf}]\hfill \\
    Plots the Electron Localization Function (ELF). By default \ttt{False}.
    \item [\texttt{elfts}]\hfill \\
    Plots the approximate ELF (ELF$_\text{TS}$). By default \ttt{False}.
    \item [\texttt{ntos}]\hfill \\
    Plots the natural transition orbitals, the transition density, the particle density and the hole density for the excitations given in \texttt{excitations}. Can only be done if an \texttt{LRSCFTask} was carried out before manually. By default \ttt{False}.
    \item [\texttt{nros}]\hfill \\
    Plots the natural response orbitals, which can only be done if an \texttt{LRSCFTask} with a \texttt{frequencies} entry has been carried out before manually. Thus far only available for an isolated system. By default \ttt{False}.
    \item [\texttt{cctrdens}]\hfill \\
    Plots the CC2 transition densities (left and right) for those excitations given in \texttt{excitations}. An appropriate \ttt{LRSCFTask} (with \ttt{cctrdens True} and the same active and environment systems) needs to be run before. By default \ttt{False}.
    \item [\texttt{ccexdens}]\hfill \\
    Plots the CC2 state densities (ground and excited states) for those excitations given in \texttt{excitations}. An appropriate \ttt{LRSCFTask} (with \ttt{ccexdens True} and the same active and environment systems) needs to be run before. By default \ttt{False}.
\end{description}

\subsubsection{Advanced Keywords}
\begin{description}
    \item [\texttt{maxGridPoints}]\hfill \\
Maximum number of grid points before stopping the cube file generation (1.6 GB). By default \ttt{1e+8}.
    \item [\texttt{ntoPlotThreshold}]\hfill \\
    Only the NTOs with a contribution to an investigated excitation higher than this threshold are plotted. By default \ttt{0.1}.
\item [\texttt{excitations}]\hfill \\
A vector that defines the excitations of interest. Necessary in order to plot NTOs, transition densities or hole and particle densities. Note that counting starts at 1 and refers to ascending transition energies as calculated by the \texttt{LRSCTask}. By default an empty vector. Example input: \ttt{excitations \{1 2 4\}}
\item [\texttt{nrominimum}]\hfill \\
A value between 0 and 1 which defines the accumulated singular values of natural response orbitals that are plotted. NROs are plotted starting from the pair with the highest singular value until the accumulated singular values reach \texttt{nrominimum}. By default \ttt{0.75}.
\end{description}

\subsection{Task: Population Analysis}
This task calculates the atom-wise electron populations of a given system.
\subsubsection{Example Input}
\begin{lstlisting}
 +task POP
  system water
  algorithm IAO
 -task
\end{lstlisting}

\subsubsection{Basic Keywords}
\begin{description}
 \item [\texttt{name}]\hfill \\
  Aliases for this task are \ttt{PopulationAnalysisTask}, \ttt{Population} and \ttt{POP}.
 \item [\texttt{activeSystems}]\hfill \\
  The system for which to calculate the atomic populations.
 \item [\texttt{algorithm}]\hfill \\
  The algorithm used for the population analysis.
  Options are: \ttt{Mulliken}, \ttt{Hirshfeld}, \ttt{Becke}, \ttt{IAO} (intrinsic atomic orbital analysis),\ttt{IAOShell}, \ttt{CM5} (Charge Model 5), and \ttt{CHELPG} (shell-wise IAO analysis). The default algorithm is \ttt{Mulliken}.
\end{description}

\subsection{Task: Quasi-Restricted Orbitals}

This task allows the calculation of quasi-restricted orbitals (QROs)~\cite{Neese2006} from UHF or UKS orbitals.
\subsubsection{Example Input}
\begin{lstlisting}
 +task QRO
  act methyl
 -task
\end{lstlisting}

\subsubsection{Basic Keywords}
\begin{description}
 \item [\texttt{name}]\hfill \\
  Aliases for this task are \ttt{QRO} and \ttt{QUASIRESTRICTEDORBITALS}.
 \item [\texttt{activeSystems}]\hfill \\
  The system for which the QROs should be calculated.
 \item [\texttt{environmentSystems}]\hfill \\
  Environment systems may be supplied for an embedded energy/Fock matrix calculation
  for the re-canonicalization of the QROs.
 \item[\texttt{sub-blocks}]\hfill \\
  The embedding settings (\ttt{emb}) settings can be added by a sub-block. They are used for the embedded energy/Fock matrix
  calculation.
 \item [\texttt{canonicalize}]\hfill \\
  If \ttt{true} the QROs will be re-canonicalized. Doubly occupied orbitals are chosen to
  diagonalize the beta-Fock matrix, singly occupied orbitals to diagonalize (alpha-Fock + beta-Fock)/2
  and the purely virtual orbitals to diagonalize the alpha-Fock matrix.
\end{description}

\subsection{Task: SCF}
This task solves the Roothaan--Hall equation (Eq.~\ref{eq:roothaan-hall}) for one single system.
\begin{equation}\label{eq:roothaan-hall}
\textbf{F}\textbf{C} =  \textbf{S}\textbf{C}\varepsilon
\end{equation}
Here $\textbf{F}$ is the Fock matrix which is the matrix representation of the Fock operator $\hat{F}$:
\begin{equation}\label{eq:fockop}
\hat{F} = \hat{T}_{\text{e}} + \hat{V}_{\text{ne}} + \hat{J} + \hat{V}_{\text{xc}}~,
\end{equation}
$\textbf{S}$ is the atom orbital (AO) overlap matrix, with the elements:
\begin{equation}
S_{ij} = \langle \chi_i(r) | \chi_j(r) \rangle= \int \chi_i(r) \chi^*_j(r) ~\text{d}r,
\end{equation}
$\textbf{C}$ is the coefficient matrix containing the the linear combination of atomic orbitals (LCAO) of
each molecular orbital (MO):
\begin{equation}
\phi_i(r) = \sum_j C_{ij} \cdot \chi_j(r)~,
\end{equation}
and $\varepsilon$ is a diagonal matrix containing the orbital energies.
$\textbf{F}$ can be the (Hartree--)Fock matrix ($\hat{V}_{\text{xc}} = \hat{K}$) and also the Kohn--Sham Fock matrix ($V_{\text{xc}}[\rho(r),\nabla\rho(r),...]$), depending on the method chosen
in the \ttt{system} block of the given active system.
The iterative SCF procedure consists of three main steps:
\begin{itemize}
	\item Construction of $\textbf{F}$ from the electron density ($\textbf{P}$,$\rho(r)$)
	\item Generation of new orbitals by solving of Eq.~\ref{eq:roothaan-hall} for $\textbf{C}$
	\item The calculation of the electron density ($\textbf{D}$,$\rho(r)$) from the orbitals, using a given occupation.
\end{itemize}
Convergence of the SCF procedure is accelerated using DIIS\cite{pula1982} and damping by default.
In addition to the default settings it is also possible to switch the DIIS\cite{pula1982} for the ADIIS\cite{hu2010},
and it is possible to modify the way the damping is applied, and also its strength.
In the case of SCF calculations that do not converge easily it usually helps to simply increase the amount of damping that is
applied.
The ultimately best strategy may however vary with the given system.
For all options pertaining the convergence please see the \ttt{scf} block options of the systems, Section~\ref{sec:system:scf}.
\subsubsection{Example Input}
\begin{lstlisting}
 +task SCF
   system water
 -task
\end{lstlisting}
\subsubsection{Basic Keywords}
\begin{description}
	\item[\texttt{name}]\hfill \\
	Aliases for this task are \ttt{SCFTask} and \ttt{SCF}
	\item[\texttt{activeSystems}]\hfill \\
	Accepts a single active system that will be used in the SCF calculation.
	\item[\texttt{sub-blocks}]\hfill \\
	The local correlation (\ttt{lc}) settings can be added in a sub-block. A common setting (with its default value) in this sub-block is \ttt{pnoSettings=TIGHT}.	
	\item[\texttt{restart}]\hfill \\
	Will try to restart the SCF from loaded orbitals. By default this setting is \ttt{false}.
	\item[\texttt{mp2Type}]\hfill \\
	The MP2-type used for the evaluation of the correlation energy of double-hybrid functionals. By default \ttt{DF} is chosen, which uses the density fitting approach specified with \ttt{densfitCorr} in the system block. Other options are \ttt{AO} to evaluate the full two-electron four-center integrals and \ttt{local} to use local MP2.
	\item[\texttt{maxResidual}]\hfill \\
	Convergence threshold for the local MP2 calculation. The default value is \ttt{1.0e-5}.
	\item[\texttt{maxCycles}]\hfill \\
	Maximum number of iterations before cancelling the amplitude optimization in local MP2. The default number of cycles is \ttt{100}.
	\item[\texttt{skipSCF}]\hfill \\
	Skip the SCF procedure and perform an energy evaluation only. By default \ttt{false}.
	\item[\texttt{allowNotConverged}]\hfill \\
  If the maximum number of SCF cycles is reached Serenity will continue even with non-converged orbitals. By default \ttt{false}.
  \item[\texttt{calculateMP2Energy}]\hfill \\
  If the correlation energy of double-hybrid functionals is calculated. By default \ttt{true}.
  \item[\texttt{exca}]\hfill \\
  A vector of excitations of alpha-electrons to start a $\Delta$SCF MOM/IMOM~\cite{Gilbert2008,Barca2018} calculation. \ttt{\{0 0\}} gives a HOMO$\rightarrow$LUMO transition, \ttt{\{1 1\}} a HOMO-1$\rightarrow$LUMO+1 transition. By using \ttt{\{0\}} it is possible to do a MOM/IMOM procedure without giving a excitation. By default \ttt{\{\}}.
  \item[\texttt{excb}]\hfill \\
  The same as \texttt{exca} for beta electrons. By default \ttt{\{\}}.
  \item[\texttt{momCycles}]\hfill \\
  Number of MOM cycles before transitioning into the IMOM procedure. By default \texttt{0}.
\end{description}

\subsection{Task: System-Addition}\label{task: addition}
The System-Addition Task constructs a supersystem from subsystems. The supersystem can be constructed only from the atoms of the subsystems or from atoms and occupied orbitals. Basis-set changes are allowed. Charge, spin and geometry of the supersystem may be changed.
\subsubsection{Example Input}
\begin{lstlisting}
+task add
  act super
  env A
  env B
-task
\end{lstlisting}
\subsubsection{Basic Keywords}
\begin{description}
	\item[\texttt{name}]\hfill \\
	Aliases for this task are \ttt{ADDITIONTASK} and \ttt{ADD}
	\item[\texttt{activeSystems}]\hfill \\
	Accepts a single active system that will be used as the supersystem.
	\item[\texttt{environmentSystems}]\hfill \\
	A list of the environment systems used as the subsystems added together.
	\item[\texttt{checkSuperGeom}]\hfill \\
	Perform a cross-check to ensure that the supersystem geometry is not changed. By default \ttt{false}.
	\item[\texttt{checkSuperCharge}]\hfill \\
	Perform a cross-check to ensure that supersystem charge and spin are not changed. By default \ttt{false}.
	\item[\texttt{addOccupiedOrbitals}]\hfill \\
	Construct the supersystem occupied orbitals from the union of the subsystem orbitals. The orbital set may not be orthonormal! By default \ttt{false}.
\end{description}

\subsection{Task: System-Splitting}
The System-Splitting Task splits the occupied orbital set of a supersystem into subsets which are associated with fragments of the supersystem. The number of fragments is not limited, however, the union of all subsystem atoms have to give the supersystem atoms. Atoms are not allowed to be duplicated in the subsystems. The partitioning is based on the orbital localization. Basis-set changes are allowed. If the basis set of a sub- and the supersystem do not match, the density matrix of the subsystem is constructed by projecting the density matrix corresponding to the selected orbitals into the subsystem basis. Charge, spin and geometry of
the subsystems may be changed.
\subsubsection{Example Input}
\begin{lstlisting}
+task split
  act super
  env A
  env B
-task
\end{lstlisting}
\subsubsection{Basic Keywords}
\begin{description}
	\item[\texttt{name}]\hfill \\
	Aliases for this task are \ttt{SPLITTINGTASK} and \ttt{SPLIT}
	\item[\texttt{activeSystems}]\hfill \\
	Accepts a single active system that will be used as the supersystem which is split.
	\item[\texttt{environmentSystems}]\hfill \\
	A list of the environment systems the supersystem is split into.
	\item[\texttt{systemPartitioning}]\hfill \\
	The algorithm for the system partitioning. The default is \ttt{BESTMATCH} and other possibilities are \ttt{ENFORCECHARGES}, \ttt{POPULATION}, \ttt{SPADE} and  \ttt{SPADE\_ENFORCE\_CHARGES}.
	\item[\texttt{orbitalThreshold}]\hfill \\
	The population threshold for the \ttt{POPULATION} partitioning with a default of \ttt{0.4}.
\end{description}

\subsection{Task: Top-Down Embedding\label{sec:topDownTask}}
The Top-Down Embedding Task (also named the Projection-Based Embedding Task in prior versions),
is one of the two major embedding Tasks in \textsc{Serenity}. It only works with two subsystems, one being the active system (containing all atoms to be in the active region), the second one being the environment system (containing all other atoms). The 'top-down' labeling references the fact that at first a supersystem calculation is carried out. Subsequently, the two actual subsystems are identified by distribution of the orbital space into an active and an environment part. The active system is then relaxed using the specified active system method, while the environment system is kept frozen.\\
The initial supersystem calculation is carried out using the settings of the environment system, expecting these settings to result in a setup that is computationally cheaper, in general. Note that the basis set requested in the environment system is used for all atoms of the supersystem in this calculation. This ensures the evenly tempered distribution of the electron density across the entire supersystem, which is beneficial for a reasonable distribution of orbitals into active and environment regions. This distribution usually happens after a localization procedure. All of the settings for both, orbital localization and distribution into subsystems are given below. Furthermore, it is possible to truncate the basis of the active system in several different ways, which are also given below. This feature allows for the embedded active system calculation to be faster than a supersystem calculation with the same settings. \\
Note that the Top-Down Embedding Task is identical to calling the following input, with exception for
potential-reconstruction techniques:
\begin{lstlisting}
+system
name supersystem
...
-system

+system
name active
...
-system

+system
name environment
...
-system

+task ADD
act supersystem
env active
env environment
addOccupiedOrbitals false
#May be set to true to produce an initial guess for the 
#supersystem from subsystem densities.
-task

+task SCF
system supersystem
-task

+task LOC
system supersystem
locType <locType>
-task

+task SPLIT
act supersystem
env active
env environment
systemPartitioning <systemPartitioning>
orbitalThreshold <orbitalThreshold>
-task

+task BASISSETTASK
#May be skipped if no basis-set truncation is desired. Or a basis-set truncation
#for the environment could be done in addition.
system active
truncAlgorithm <truncAlgorithm>
netThreshold <netThreshold>
truncationFactor <truncationFactor>
-task

+task FDE
#A FATTask could be used here, too.
act active
env environment
+EMB
...
-EMB
-task
\end{lstlisting}
Thus, the Top-Down Embedding Task task can be restarted after or during any of the tasks above as described for the respective task. Note that in the input above, the settings of the supersystem are used for the supersystem
SCF.

\subsubsection{Example Input}
\begin{lstlisting}
+task TDEmbeddingTask
  act Active
  env Env
  +emb
    naddXcFunc PW91
    embeddingmode NADDFUNC
  -emb
-task
\end{lstlisting}
\subsubsection{Basic Keywords}
\begin{description}
	\item[\texttt{name}]\hfill \\
	Aliases for this task are \ttt{TDEmbeddingTask}, \ttt{PBE}, \ttt{ProjectionBasedEmbTask} and \ttt{TD}.
	\item[\texttt{activeSystems}]\hfill \\
	Will use this system as the active part of the supersystem.
	\item[\texttt{environmentSystems}]\hfill \\
	Will use the first environment system as the environment part of the supersystem.
	The initial supersystem calculation will be carried out implying the options
	of this system, including the application of the given basis label to all atoms,
	even those of the active system.
	\item[\texttt{sub-blocks}]\hfill \\
	Possible sub-blocks are the embedding (\ttt{emb} with \ttt{naddXCFunc=BP86} and \ttt{embeddingMode=LEVELSHIFT}), the local correlation (\ttt{lc} with \ttt{pnoSettings=TIGHT}) and the PCM (\ttt{pcm}) settings.
	\item[\texttt{locType}]\hfill \\
	The orbital localization method. The default is \ttt{IBO}. For other valid options see localization options in Section~\ref{task:localization}.
	\item[\texttt{orbitalThreshold}]\hfill \\
	Threshold for orbital populations on the active region to be considered active. By defalt \ttt{0.6}.
	\item[\texttt{systemPartitioning}]\hfill \\
	The algorithm used for the system partitioning. The default algorithm is \ttt{POPULATION\_THRESHOLD}.
	\item[\texttt{mp2Type}]\hfill \\
	The MP2-type used for the evaluation of the correlation energy of double-hybrid functionals. By the default \ttt{DF} is chosen. Other options are \ttt{AO} to evaluate the full two-electron four-center integrals and \ttt{local} to use local MP2.
	\item[\texttt{splitValenceAndCore}]\hfill \\
	Localize valence and core orbitals independently. By default \ttt{false}.
\end{description}
\subsubsection{Advanced Keywords}
\begin{description}
	\item[\texttt{truncAlgorithm}]\hfill \\
	The truncation algorithm for the active system basis. By default no truncation is performed (\ttt{NONE}). For further information see Section~\ref{task:truncation}.
	\item[\texttt{truncationFactor}]\hfill \\
	The truncation factor used in the "primitive" truncation schemes. Needs \ttt{truncAlgorithm=PRIMITIVENETPOP} and has a default value of \ttt{0.0}. For further information see Section~\ref{task:truncation}.
	\item[\texttt{netThreshold}]\hfill \\
	The Mulliken net population threshold for the Mulliken net population truncation. Needs \ttt{truncAlgorithm=NETPOPULATION} and has a default value of \ttt{1.0e-4}.  For further information see Section~\ref{task:truncation}.
	\item[\texttt{noSupRec}]\hfill \\
	Boolean (default \ttt{true}) to switch off/on double reconstruction. Needs \ttt{embeddingMode=RECONSTRUCTION}.
	\item[\texttt{useFermiLevel}]\hfill \\
	Boolean (default \ttt{true}) to switch off/on using the supersystem-Fermi level for the Fermi-shifted Huzinaga equation. Needs \ttt{embeddingMode=FERMI}.	
  \item[\texttt{load}]\hfill \\
  The path to load HDF5 files for the supersystem from. If this path and a name for the supersystem is given, that system is used for the calculation. Atoms of the supersystem have to be the union of active and environment atoms. No sorting required.  
  \item[\texttt{name}]\hfill \\
  The name of the supersystem calculation. If this and a load path is given, the given supersystem is used for the calculation.
  \item[\texttt{maxResidual}]\hfill \\
  Convergence threshold for the local MP2 calculation. By default \ttt{1.0e-5}.
  \item[\texttt{maxCycles}]\hfill \\
  Maximum number of iterations before cancelling the amplitude optimization in local MP2. The default number of cycles is \ttt{100}.   
  \item[\texttt{addOrbitals}]\hfill \\
  Add the orbitals of the subsystems up to form the supersystem and skip the supersystem SCF. By default \ttt{false}.
\end{description}
\subsection{Task: Top Down Static Embedding}\label{sec:tasks:TopDownStaticEmbedding}
This tasks runs a top-down embedding calculations without additional relaxation of the orbital subspaces after supersystem partitioning.

\subsubsection{Example Input}
\begin{lstlisting}
 +task TDSTATIC
  act waterA
  act waterB
  env waterC
  useQuasiRestrictedOrbitals true
  +emb
      naddXcFunc pbe
      embeddingMode fermi
  -emb
 -task
\end{lstlisting}

\subsubsection{Basic Keywords}
\begin{description}
    \item [\texttt{name}]\hfill \\
    Aliases for this task are \ttt{TDSTATIC}, \ttt{STATIC}, and \ttt{TOPDOWNSTATICEMBEDDINGTASK}.
    \item [\texttt{activeSystems}]\hfill \\
    Accepts multiple active systems.
    \item [\texttt{environmentSystems}]\hfill \\
    Accepts multiple environment systems.
    \item [\texttt{useQuasiRestrictedOrbitals}]\hfill \\
    If true, quasi restricted orbitals are generated und used for the subsystem partitioning.
    All orbital localization steps will be restricted to rotations within the quasi restricted orbital subspaces.
    \item[\texttt{sub-blocks}]\hfill \\
    Possible sub-blocks are the embedding (\ttt{emb} with \ttt{naddXCFunc=BP86} and \ttt{embeddingMode=FERMI}), the local correlation (\ttt{lc} with \ttt{pnoMethod=None}),
    the PCM (\ttt{pcm}) settings, the orbital localization settings (\ttt{loc} with \ttt{splitValenceAndCore=true}), the system splitting settings (\ttt{split}),
    and the system addition settings (\ttt{add}).

\end{description}
\subsubsection{Advanced Keywords}
\begin{description}
    \item [\texttt{maxCycles}]\hfill \\
    The maximum number of cycles allowed until convergence is expected by the coupled cluster iterations.
    By default, 100 cycles are allowed.
    \item [\texttt{normThreshold}]\hfill \\
    The threshold at which the coupled cluster/MP2 iterations are considered converged. By default \ttt{1.0e-5}.
    \item [\texttt{writePairEnergies}]\hfill \\
    Write the pair energies to a file with name: systemName\_pairEnergies\_CCSD.dat or systemName\_pairEnergies\_MP2.dat
\end{description}
\subsection{Task: Transition-State Geometry Optimization}
This task performs transition-state geometry optimizations. Has to be provided with one guess structure for the transition state (act) and can additionally be given structures of both minima on the potential-energy surface (env).
\subsubsection{Example Input}
\begin{lstlisting}
+task TS
  act water
-task
\end{lstlisting}
\subsubsection{Basic Keywords}
\begin{description}
  \item [\texttt{name}]\hfill \\
    Aliases for this task are \ttt{TSTask}, \ttt{TSOpt} and \ttt{TS}.
  \item [\texttt{activeSystem}]\hfill \\
    Transition state to be optimized.
  \item [\texttt{environmentSystems}]\hfill \\
    Structures of minima on the potential energy surface.
  \item [\texttt{lst}]\hfill \\
    Specifies if LST is performed before QST. Default is \ttt{false}.
  \item [\texttt{lstqstonly}]\hfill \\
    Specifies that only LST/QST should be done and NO Bofill TS-search afterwards. Default is \ttt{false}.
  \item [\texttt{normalmode}]\hfill \\
    Specifies the reaction coordinate along which the TS is searched. Default is \ttt{1}.
\end{description}
\subsection{Task: Virtual Orbital Space Selection}
\label{sec:VOSS}
Selects, manipulates or localizes a specific virtual orbital space for a subsystem.
\subsubsection{Example Input}
\begin{lstlisting}
+task VOSS
  act A
  act ALE
  env B
  excludeprojection true
  localizedVirtualOrbitals true
-task
\end{lstlisting}
\subsubsection{Basic Keywords}
\begin{description}
	\item[\texttt{name}]\hfill \\
    Aliases for this task are \ttt{VOSSTASK}, \ttt{VOSS} and \ttt{VIRTUALORBITALSPACESELECTIONTASK}.
	\item[\texttt{activeSystems}]\hfill \\
	A list of active systems.
	\item[\texttt{environmentSystems}]\hfill \\
	A list of the environment systems.
  \item[\texttt{excludeProjection}]\hfill\\
  Excludes occupied orbitals of environment subsystems in projection-based embedding calculations from the virtual orbital space of the active subsystem. The default is \ttt{false}. 
  \item[\texttt{localCanonicalVirtuals}]\hfill\\
  Select canonical virtual orbitals located on the active subsystem based on a modified overlap criterion. The default is \ttt{0.0}. This requires two active systems. The first active system is the system with the initial virtual orbital space and the second system is the system to which the new virtual orbitals are written to. This keyword should be used for projection-based embedding calculations.
  \item[\texttt{envCanonicalVirtuals}]\hfill\\
  Select canonical virtual orbitals located on the environment subsystems based on a modified overlap criterion. The default is \ttt{0.0}. This requires two active systems. The first active system is the system with the initial virtual orbital space and the second system is the system to which the new virtual orbitals are written to. This keyword should be used for projection-based embedding calculations.
  \item[\texttt{localizedVirtualOrbitals}]\hfill\\
  Performs a localization of the virtual orbitals and selects the virtual orbitals located on the subsystem. The default is \ttt{false}. The first active system is the system with the initial virtual orbital space and the second system is the system to which the new virtual orbitals are written to. This keyword should be used for projection-based embedding calculations.
  \item[\texttt{localizedEnvVirtualOrbitals}]\hfill\\
  Performs a localization of the virtual orbitals and selects the virtual orbitals located on the subsystem. The default is \ttt{false}. The first active system is the system with the initial virtual orbital space and the second system is the system to which the new virtual orbitals are written to. This keyword should be used for projection-based embedding calculations.
\end{description}
\subsubsection{Advanced Keywords}
\begin{description}
  \item[\texttt{recalculateFockMatrix}]\hfill\\
  Recalculates the Fock matrix. The default is \ttt{false}.
  \item[\texttt{identifier}]\hfill\\
  Additional identifier for the file in which the new orbitals and orbital energies are stored. The default is \ttt{""}
  \item[\texttt{mixingOccAndVirtualOrbitals}]\hfill\\
  Takes the occ orbitals of environment subsystem 1 and the virtuals of the environment subsystem 2. The structure of the supersystem needs to be given by the active system. The new orbital space is saved in the active system. The default is \ttt{false}.
  \item[\texttt{relaxation}]\hfill\\
  Performs an additional orbital space orthogonalization in case of \texttt{mixingOccAndVirtualOrbitals} is used. The default is \ttt{false}.
\end{description}

\subsection{Task: Wavefunction Embedding Task}
\label{sec:WFinWFTask}
Allows the DLPNO-based coupled-cluster-in-coupled-cluster embedding according to Ref.~\cite{Sparta2017}.
\subsubsection{Example Input}
You can start the embedding calculation either from subsystems with previously calculated orbitals
(\ttt{fromFragments true}), or by partitioning a supersystem into subsystems based on the localization
of the supersystem orbitals on the subsystems (\emph{e.g.}, see the Top-Down embedding task in
Section~\ref{sec:topDownTask}).

Embedding with already calculated subsystem orbitals:
\begin{lstlisting}
+task WFEMB
  act supersystem
  env frag1
  env frag2
  env frag3
  fromFragments true
  fullDecomposition false
  +lc0
    method DLPNO-CCSD(T0)
    pnoSettings normal
    useFrozenCore true
  -lc0
  +lc1
    method DLPNO-CCSD
    pnoSettings loose
    useFrozenCore true
  -lc1
  +lc2
    method NONE
    useFrozenCore true
  -lc2
-task
\end{lstlisting}

Embedding by partitioing of the supersystem orbitals:
\begin{lstlisting}
+task WFEMB
  act supersystem
  env frag1
  env frag2
  fromFragments false
  fullDecomposition false
  +lc0
    method DLPNO-CCSD(T0)
    pnoSettings normal
    useFrozenCore true
  -lc0
  +lc1
    method DLPNO-CCSD
    pnoSettings loose
    useFrozenCore true
  -lc1
  +loc
    locType IBO
  -loc
  +split
    systemPartitioning bestmatch
  -split
-task
\end{lstlisting}

\subsubsection{Basic Keywords}
\begin{description}
    \item [\texttt{name}]\hfill \\
    Aliases for this task are \ttt{WFEMB} and \ttt{WAVEFUNCTIONEMBEDDING}.
	\item[\texttt{activeSystems}]\hfill \\
	The supersystems to be used.
	\item[\texttt{environmentSystems}]\hfill \\
	The subsystems to be used.
    \item [\texttt{sub-blocks}]\hfill \\
    The embedding (\ttt{emb}), local correlation (\ttt{lc}), PCM (\ttt{pcm}), localization task settings (\ttt{loc}) and
    system splitting task settings (\ttt{split}) are added via sub-blocks in the task settings.
    The local correlation settings are added as a list of settings. For each fragment (environment) system there is
    a local correlation settings object that determines the settings employed for the orbitals of this system. The
    task assumes that systems/settings with a lower index are tighter.
    By default \ttt{splitValenceAndCore = true} is set for the localization task settings.
    \item [\texttt{fullDecomposition}]\hfill \\
    If true, a full energy decomposition is calculated in terms of energy of the fragments and interaction energy.
    By default \ttt{false}.
    \item [\texttt{fromFragments}]\hfill \\
    If true, the supersystem is ignored and overwritten with the union of the subsystems. By default \ttt{false}.
\end{description}
\subsubsection{Advanced Keywords}
\begin{description}
  \item[\texttt{accurateInteraction}]\hfill\\
  If true, the settings with the lower subsystem index are used for cross system pairs.
  Otherwise: the settings with the higher subsystem index are used. By default \ttt{true}.
  \item [\texttt{maxCycles}]\hfill \\
  The maximum number of cycles allowed until convergence is expected by the coupled cluster iterations.
  By default 100 cycles are allowed.
  \item [\texttt{normThreshold}]\hfill \\
  The threshold at which the coupled cluster iterations are considered converged. By default \ttt{1.0e-5}.
  \item[\texttt{writePairEnergies}] \hfill \\
  If true, write the pair energies to a file with the name: systemName\_pairEnergies\_MultiLevelCC.dat. \ttt{False} by default.
\end{description}

\subsection{Task: Write Integrals}
This task can write HCore integrals to disk in either HDF5 or ASCII file format.
\subsubsection{Example Input}
\begin{lstlisting}
+task INTEGRALS
  system water
  fileFormat HDF5
-task
\end{lstlisting}

\subsubsection{Basic Keywords}
\begin{description}
    \item [\texttt{name}]\hfill \\
    Aliases for this task are \ttt{WriteIntegralsTask}, \ttt{WriteInts}, \ttt{Integrals} and \ttt{Int}.
    \item [\texttt{activeSystems}]\hfill \\
    This task accepts a single active system for which the integrals will be written to disk.
    \item [\texttt{fileFormat}]\hfill \\
    File format used to write the orbitals to disk. Options are \ttt{HDF5} and \ttt{ASCII}. By default \ttt{HDF5}.
\end{description}
    
\subsubsection{Advanced Keywords}
\begin{description}
    \item [\texttt{hCoreIntegrals}]\hfill \\
    If \ttt{true}, the core hamiltonian integrals are written. By default \ttt{false}.
\end{description}
